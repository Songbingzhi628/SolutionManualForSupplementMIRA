\ChDecl{Ch0B}{0$\cdot$B}{\quad{\ANote $C,D$ are Dedekind cuts.\;\;Numbers used here are always rational.}}

\vspace{4pt}

%LOG: 尽管作者说这一节以后用不到,但这一节还是给我干出心理阴影了。我真是一点没学过实分析。

\BulletPointX Define $\tilde{q}=\Bra{a:a<q},$ and $-\tilde{q}=\widetilde{-q}=\Bra{a:a<-q}.$\TextB{}
Then $\tilde{0}=\Bra{a:a<0}=\Qbb\Backslash{\Qbs}\Rightarrow-\tilde{0}=\Bra{a:a<-b\leqslant0}=\tilde{0}.$\par\vspace{3pt}
\BulletPointX Define $-D=\Bra{a:a<-b,\,b\not\in D}=\Bra{{-a}:-a<-b\Longleftrightarrow a>b,\:b\not\in D}.$\TextB{}
$-\Par{{-D}}=-\Bra{a:a<-b,\,b\notin D}=\Bra{c:c<-a,\,a\geqslant -b,\,\forall b\notin D}=\Bra{c:c<b,\,\forall b\notin D}=D.$\TextB{}
{\FontSmall The last equa is becs (a) $d\notin D\Rightarrow\exists\,b\notin D,\,d\geqslant b,$ and (b) $d\in D\Rightarrow$ if $\exists\,b\notin D$ suth $d\geqslant b,$ then $b\in D,$ ctradic.}\vspace{-2pt}
\SepLine

%\Anchor{0BT1}\ProblemBX{\TipsN{1}}{
%	\TextA{Supp $D_1=\Bra{x\in\Qbb:x<a},\:D_2=\Bra{y\in\Qbb:y<b}.$}
%	\TextA{Prove $D_1+D_2=\Bra{q\in\Qbb:q<a+b}.$}
%}Becs $D_1+D_2=\Bra{x+y:x+y<x+b<a+b,\,x\in D_1,\,y\in D_2}.$\parSol{}
%Supp $x\in D_1.$ Let $\delta=\Par{a-x}\big/2\Rightarrow x<x'=x+\delta<a.$ Let $y=b-\delta\in D_2.$\parSol{}
%Then $x+b=x'+y<a+b.$ Let $\delta'=\Par{a-x'}\big/2\Rightarrow x+b<x''+y<a+b.$\PfEnd\vspace{3pt}\parSol{}
%\Or $D_1+D_2\subseteq\Bra{q\in\Qbb:q<a+b}.$ We show $\forall q\in RHS,\exists\,\Par{x,y}\in D_1\times D_2,\:x+y=q.$\parSol{}
%Let $\delta=\Par{a+b-q}\big/2.$ Let $x=a-\delta\in D_1,\,y=b-\delta\in D_2\Rightarrow x+y=a+b-\delta=q.$\PfEnd
%\SepLine

%\Anchor{0BT2}\ProblemBX{\TipsN{2}}{
%	\TextA{Supp $D$ is Ddkd cut. Prove $\forall b\in\Qbb\Backslash D,\:d<b,\forall d\in D.$}
%}Asum $\exists\,\Par{b,d}\in\Par{\Qbb\Backslash D}\times D,\:d\geqslant b.$ By def of Ddkd cut.\PfEnd
%\SepLine

\Anchor{0BT}\ProblemBX{\Tips}{
%	\PrePa\TextA{If $x\in\Qbb$ suth $\forall d\in D,\,d\leqslant x,$ then $x\in\Qbb\Backslash D.$\hfill\FontNorm\tgnr If not, by def, ctradic.}
%	\PrePb\TextA{If $\forall d\in D,\,\exists\,\varepsilon\in\Qbp,\,d+\varepsilon\in D.$\hfill\FontNorm\tgnr Supp $d<d'\in D\Rightarrow d'-d=\varepsilon\in\Qbp.$}
	\TextA{Prove $\forall\varepsilon>0,\,\exists\,b\not\in D$ \,suth\, $b-\varepsilon\in D.$}
}Asum $\exists\,\varepsilon>0$ suth $\nexists\,b\not\in D,\,b-\varepsilon\in D\Longleftrightarrow\forall b\not\in D,\,b-\varepsilon\not\in D.$\parSol{}
Then $\Par{b-\varepsilon}-\dots-\varepsilon=b-n\cdot\varepsilon\not\in D$ for any $n\in\Nbp.$\parSol{}
Now $\forall d\in D,\,\exists\,n\in\Nbp$ suth $b-n\cdot\varepsilon<d\Rightarrow b-n\cdot\varepsilon\in D,$ ctradic.\PfEnd
%\ACoro Since $D$ does not contain the largest elem, $\forall\varepsilon\in\Qbp,\,\exists\,b\in\Qbb\Backslash D,\;b-\varepsilon<d$ for some $d\in D.$
\SepLine

\Anchor{0B1}\ProblemN{1}{
	\TextA{Prove {\tgnr\large(a)} $D+\tilde{0}=D,$ \;{\tgnr\large(b)} $-D$ is Dedekind cut, and $D+\Par{{-D}}=\tilde{0}.$}
}(a) $\forall d\in D,\,\exists\,\varepsilon>0,\:d+\varepsilon\in D\Rightarrow\Par{d+\varepsilon}+\Par{{-\varepsilon}}\in D+\tilde{0}.$\vspace{2pt}\parSol{}
(b) Asum $x\in -D$ is the largest elem of $-D\Rightarrow\exists\,b\not\in D,\:x<-b\Rightarrow 0<-b-x.$\parSol{\Hb}
Let $\delta=\Par{{-b-x}}\big/2\Rightarrow0<\delta<-b-x\Rightarrow x<x+\delta<-b.$\parSol{\Hb}
Thus by def, $x+\delta\in -D,$ ctradic the max of $x\in -D.$ \,Hence $-D$ is Ddkd cut.\vspace{3pt}\parSol{\Hb}
$D+\Par{{-D}}=\Bra{x+y:x+y<x-b,\:x\in D,\,b\not\in D}.$\parSol{\Hb}
Supp $a\in\tilde{0}\Rightarrow-a>0.$ By \TIPS, $\exists\,b\not\in D$ \,suth\, $b+a\in D.$\parSol{\Hb}
Note that $b<b-a\not\in D\Rightarrow-b>-b+a\in -D.$ \,Then $\Par{{-b+a}}+\Par{b+a}=2\:\!a<0.$\parSol{\Hb}
Thus $\forall a\in\tilde{0},\,\exists\,b\not\in D,\;d=b+\frac{\:1\:}{2}a\in D,\,c=-b+\frac{\:1\:}{2}a\in-D\Rightarrow c+d=a\in D+\Par{{-D}}.$\PfEnd\vspace{3pt}
\ACoro $-\Par{{-D}}+\Par{{-D}}=\tilde{0}\Rightarrow -\Par{{-D}}=D,$ by the uniqnes.
%$\Bra{a_1+a_2:a_1,a_2\in D}=D+D=\tilde{0}=\Bra{a:a<0}\Longrightarrow\forall a\in D,\,a+a<0\Longleftrightarrow a<0.$
\SepLine

\Anchor{0B3}\ProblemN{3}{
	\TextA{Show $C\subsetneq D\Longleftrightarrow D-C${\FontNorm ${}=\Bra{d-y:d\in D,\:y>x,\:x\not\in C}$\:\!} posi.\vspace{1pt}}
}(a) $C\subsetneq D\Rightarrow\exists\,x\in D\Backslash C\Rightarrow\exists\,y\in D,\,y>x\Rightarrow\exists\,y'\in D,\,y'>y\Longleftrightarrow 0<y'-y\in D-C.$\vspace{2pt}\parSol{}
(b) $0\in D-C\Rightarrow\exists\,y>x\notin C,\;y\in D\Rightarrow\forall c\in C,\,c<x<y\in D\Rightarrow C\subseteq D.$ \,又 $D-C\neq\tilde{0}.$\PfEnd
\SepLine

%\Anchor{0B'1}\ProblemB{
%	\TextB{Prove $D=-D\Longleftrightarrow D=\tilde{0}.$}
%}$\Bra{a_1+a_2:a_1,a_2\in D}=D+D=\tilde{0}=\Bra{a:a<0}\Rightarrow\forall a\in D,\,a+a<0\Longleftrightarrow a<0.$\parSol{}
%\SepLine

\Anchor{0B5}\ProblemN{5}{
	\TextA{Prove {\tgnr\large(a)} $D$ posi $\Rightarrow -D$ not posi, \;{\tgnr\large(b)} non0 $-D$ not posi $\Rightarrow D$ posi.}
}(a) $0\notin\Bra{a:a<-b,\,b\notin D}\Longleftrightarrow\nexists\,b\notin D,\,0<-b\Longleftrightarrow\forall b\notin D,\,b\geqslant0\Longleftarrow 0\in D.$\vspace{2pt}\parSol{}
(b) Becs $\tilde{0}$ is the largest non posi cuts. Thus $-D\neq\tilde{0}\Rightarrow-D\subsetneq\tilde{0}\Rightarrow\tilde{0}-\Par{{-D}}=D$ posi.\parSol{\Hb}
\Or $\exists\,a<0,\,a\notin -D=\Bra{a:-a>b,\,b\notin D}\Longleftrightarrow\nexists\,b\notin D,\,-a>b\Longleftrightarrow\forall b\notin D,\,0<-a\leqslant b.$\PfEnd
\SepLine

\Anchor{0B4}\ProblemN{4}{
	\PrePa\TextA{Supp $D\neq\tilde{0}.$ Find a formula for $D^{-1}$ suth $D^{-1}$ is Dedekind cut and $DD^{-1}=\tilde{1}.$}
	\PrePb\TextA{Show assoc for multi and distr holds in $\mD.$}
}Let $D^{-1}=\Bra{a:a<1\big/d,\,\forall d\in D}\Rightarrow DD^{-1}=\Bra{a:a<rs,\:r\in D^\pm,\,s\in\Par{D^{-1}}{^\pm}}.$\parSol{}
$\Par{D^{-1}}{^+}=\Bra{a:0<a<1\big/d,\,\forall d\in D^+},\:\Par{D^{-1}}{^-}=\Bra{a:0\geqslant a\geqslant 1\big/d,\,\exists\,d\in D}.$
\SepLine\pagebreak

\BulletPointX Define $D^+=\Bra{d\in D:d>0}=D\cap\Qbp.$ \,Then $D^+\neq\emptySet\Longleftrightarrow\Qbb\Backslash\Qbp\subsetneq D\Longleftrightarrow 0\in D\Longleftrightarrow D$ posi.\vspace{2pt}\TextB{}
Define $D^-=\Bra{r\not\in D:r\leqslant 0}=\Par{\Qbb\Backslash D}\cap\Par{\Qbb\Backslash\Qbp}=\Qbb\Backslash\Par{D\cup\Qbp}.$\TextB{}
(a) $D^-=\zeroSubs\Longleftarrow D=\tilde{0}.$ \;Convly, $\Bra{r\notin D:r\leqslant0}=\zeroSubs\Longrightarrow\Qbb\Backslash D=\Qbs.$\TextB{}
(b) $D^-=\emptySet\Longleftrightarrow D\cup\Qbp=\Qbb\Longleftrightarrow\Qbb\Backslash\Qbp\subseteq D\Longleftrightarrow 0\in D\Longleftrightarrow D$ posi. \hfill\ACoro $D$ not posi $\Longleftrightarrow 0\in D^-.$\TextB{}
(c) $\Par{D^-}{^-}=\Bra{r\in D:r\leqslant0}=\Qbb\Backslash D^+.$ \hfill\ACoro $D$ not posi $\Longleftrightarrow\Par{{-D}}{^-}=D.$
\SepLine

\BulletPointX $\Par{{-D}}{^+}=\Par{{-D}}\cap\Qbp=\Bra{a:0<a<-b,\:b\not\in D\Longleftrightarrow b\in D^-\nonzero}.$\TextB{}
$\Par{{-D}}{^-}=\Par{\Qbb\Backslash{-D}}\cap\Par{\Qbb\Backslash\Qbp}=\Bra{a:0\geqslant a\geqslant-b,\:\forall b\not\in D}.$
%\vspace{3pt}\TextB{}
%Define $\tilde{1}=\Bra{a:a<1}\Rightarrow\tilde{1}{^+}=\Bra{a:0<a<1},\:\tilde{1}{^-}=\Bra{a:a\leqslant 0}=\tilde{0}\cup\zeroSubs.$\TextB{}
%Then $-\tilde{1}=\Bra{a:-a>1\Longleftrightarrow a<-1}$ not posi.
\SepLine

\BulletPointX For $C,D$ posi, \,define $CD=\Bra{a:a\leqslant cd,\,c\in C^+,\,d\in D^+}=\Bra{cd:c\in C^+,d\in D^+}\cup\Par{\Qbb\Backslash\Qbp}.$\vspace{2pt}\TextB{}
$\Bra{cd:c\in C^+,d\in D^+}=CD\cap\Qbp=\Par{CD}{^+}.$ \;Note that `$a\leqslant cd$' here is equiv to `$a<cd$'.\vspace{4pt}\par
\BulletPointX For $-C,-D$ posi, \,define $CD=\Par{{-C}}\Par{{-D}}=\Bra{cd:c\in\Par{{-C}}{^+},d\in\Par{{-D}}{^+}}\cup\Par{\Qbb\Backslash\Qbp}.$\TextB{}
$CD=\Bra{0<cd<\Par{{-r}}\Par{{-s}}:r\in C^-\nonzero,\,s\in D^-\nonzero}\cup\Par{\Qbb\Backslash\Qbp}=\Bra{a:a<rs,\,r\in C^-,\,s\in D^-}.$\vspace{2pt}\TextB{}
If $C,-C$ not posi $\Rightarrow C=\tilde{0},$ then with the asum $\tilde{0}D=\tilde{0},$ \,it still holds. \,Simlr for $D,-D$ not posi.\vspace{2pt}\par
\BulletPointX {\normalsize The intuitive key point is that the prod of cuts is the cut with the endpoint being the prod of endpoints of cuts.}
\SepLine

\BulletPointX For $C$ not posi and $D$ posi, we expect that $CD$ not posi. Consider $C$ and $-D$ both not posi.\TextB{}
$CD=-C\Par{{-D}}=-\Bra{a:a<rt,\:r\in C^-,\,t\in\Par{{-D}}{^-}}=\Bra{{-a}:a>b,\,b\geqslant rt,\:\forall r\in C^-,\,t\in\Par{{-D}}{^-}}$\TextB{}
\Blind{$CD$}${}=\Bra{{-a}:a>rt,\:\forall r\in C^-,\,0\geqslant t\geqslant -s,\,\forall s\notin D}=\Bra{a:a<ru,\:\forall r\in C^-,\,0\leqslant u\leqslant s,\,\forall s\notin D}.$\TextB{}
\Blind{$CD={}$}{\FontSmall \BigPar{$r\leqslant 0<s,\;rs\leqslant ru=-rt\leqslant0\leqslant rt\leqslant-rs.$}}\vspace{3pt}\par
\BulletPointX Note the `$0\leqslant u$'. Becs $C^-\neq\emptySet\Rightarrow 0\in C^-.$ If it is to be exactly $CD=\Bra{a:a<0}$, \,then $C^-=\zeroSubs,$\TextB{}
for if not, $\exists\,u>0,$ and $\exists\,r\in C^-\nonzero,$ such that $\exists\,a<ru<0.$ Hence `$0\leqslant u$' is actually `$0<u$'.\vspace{3pt}\par
\BulletPointX `$u\leqslant s$' cannot be abbreviated as in $\Bra{{-a}:a>-rs,\,\forall s\not\in D,\,r\in C^-}=\Bra{{a}:a<rs,\,\forall s\not\in D,\,r\in C^-}.$\TextB{}
`$u\leqslant s$' cannot be `$u<s$', becs here $rs<ru\Rightarrow\exists\,a=rs.$ \,Simlr for `$a<ru$' to be `$a\leqslant ru$' with `$u<s$'.\vspace{2pt}\par
\BulletPointX Note that $\Bra{u:0<u\leqslant s,\,\forall s\notin D}\supsetneq D^+.$
\SepLine

\BulletPointX Consider $-C$ and $D$ both posi. \;Omitting $C=\tilde{0}.$\vspace{1pt}\TextB{}
$CD=-\Sbra{\Par{{-C}}D}=-\Bra{a:a\leqslant cd,\,c\in\Par{{-C}}{^+},\,d\in D^+}=\Bra{{-a}:a>b,\,b>cd,\,\forall c\in\Par{{-C}}{^+},\,d\in D^+}$\TextB{}
\Blind{$CD$}${}=\Bra{{a}:a<-cd,\,\forall d\in D^+,\,\forall c\text{\;suth\;}$\uline{$0<c<-r,\,\exists\,r$}${}\in C^-}$\hfill{\FontSmall\BigPar{$rd<{}$\uline{$-cd<0$}}}\vspace{2pt}\TextB{}
%To make \uline{those} comfortable, notice the following:\vspace{1pt}\TextB{}
%(a) $A=\Bra{a:a<M},\,B=\Bra{b:b<a,\,\forall a\in A\Rightarrow b<a<M}\subseteq\Bra{b:b<M}=A.$\TextB{}
%\Ha \!又 $\forall b<M,\,\exists\,\epsilon>0,\:b+\epsilon<M\Rightarrow b\in B.$\vspace{3pt}\TextB{}
%(b) $\Bra{r:a<r,\,\forall a\in A}\supseteq\Bra{r:M\leqslant r}.$\TextB{}
%\Hb \!又 Supp $r$ suth $\forall a\in A,\,r>a.$ If $r<M,$ then $r\in A\Rightarrow\exists\,r'\in A,\,r'>r.$\vspace{3pt}\TextB{}
%(b) $A=\Bra{a:M<a},\,B=\Bra{b:b<a,\,\forall a\in A}\supseteq\Bra{b:b\leqslant M}$\TextB{}
%\Hb \!又 Supp $b$ suth $\forall a\in A,\,b<a.$ If $b>M,$ then $b\in A\Rightarrow\exists\,b'\in A,\,b'<b.$\vspace{4pt}\TextB{}
%$CD=\Bra{a:a<-cd,\:\forall d\in D^+,\,$\uline{$\forall c\text{\;suth\;}0<c<-c',\,\exists\,c'\notin C$}$}$\TextB{}
$\Blind{CD}\overset{*}{=}\Bra{a:a<rd,\;$\uline{$\forall r\in C^-,$}$\,\forall d\in D^+}.$\vspace{4pt}\TextB{}
Intuitively, \,$0<d_1<\cdots<d_n<\sup D,$ and $\sup C\leqslant r_{\:\!\!m}<\cdots<r_{\:\!\!1}<0.$ \,Becs $a<r_{\:\!\!j}\,d_k,\,\forall j,k.$\TextB{}
又 $r_{\:\!\!j+1}<r_{\:\!\!j}\,d_k<r_{\:\!\!j}\,d_{k-1}\Rightarrow a<r_{\:\!\!m}\,d_n<r_{\:\!\!j}\,d_k.$ \;Thus `$a<rd$' cannot be `$a\leqslant rd$'.%, becs \,$r_{\:\!\!m}\,d_{n+1}<r_{\:\!\!m}\,d_n$
\vspace{3pt}\TextB{}
($*$.a) Supp $a<-cd<0,\forall c\in\Par{{-C}}{^+},\,d\in D^+.$ Asum \,$-cd>a\geqslant rd',\,\exists\,r\in C^-,\,d'\in D^+.$\TextB{}
\Blind{($*$.a)} Let \,$d=d'\Rightarrow r\leqslant-c\Longleftrightarrow c\leqslant-r.$ \;Now $r<0.$\TextB{}
If $\exists$ min upper bound in $\Qbb,$ \,then $r=\sup C,$\TextB{}
othws $r>\sup C\Rightarrow -r<-\sup C=\Sup\Par{{-C}}\Rightarrow\exists\,c>-r,\,c<\Sup\Par{{-C}},$ simlr if $\sup C\notin\Qbb.$\TextB{}
\TextB{}
($*$.b) Supp $a\leqslant c'd,\,\forall c'\!,\:\!d$ \Par{$\Rightarrow a<0$}. Asum \,$a\geqslant-cd,\,\exists\,c'\!,\:\!d'.$\TextB{}
\Blind{($*$.b)} \vspace{-2pt}
\SepLine

\BulletPointX Let $\,LHS=\Bra{a:a<ru,\:\forall r\in C^-,\,0<u\leqslant s,\,\forall s\notin D},\;RHS=\Bra{a:a\leqslant rd,\;\forall r\in C^-,\,\forall d\in D^+}.$\vspace{1pt}\TextB{}
Where $\tilde{0}\neq C$ is not posi, $D$ posi. We show `$rd$' < `$ru$', \,so that\, $LHS=RHS.$\TextB{}
\textbf{Seems} equiv to \,`$d$' $>$ `$u$' $\geqslant$ `$s$', \,while $d\in D,\,s\notin D,$ thus ctradic. \,{\NOTICE that \,`$r,d,u$'\, are \textbf{not certain}.}\TextB{}
Supp $a<ru,\,\forall r,u.$ \,Asum $a>r'd,\,\exists\,r'\!,\:\!d.$ Now $r'd<a<ru,\,\forall r,u.$ Let $r=r'\Rightarrow d>u,$ ctradic.\TextB{}
Supp $a\leqslant rd,\,\forall r,d.$ \,Asum $a\geqslant r'u,\,\exists\,r'\!,\:\!u.$ \Par{$r'\neq0$} Now $r'u\leqslant a\leqslant rd,\,\forall r,d\Rightarrow u\geqslant d,\,\forall d\in D^+.$\TextB{}
In fact, \,$u>d,\,\forall d\in D^+\Rightarrow u\notin D.$\TextB{}
If $\exists$ certain smallest elem $s_M$ in $\Qbb\Backslash D$, and if $u=s_M,$ then $LHS=\Bra{a:a<rs_M,\,\forall r\in C^-}.$\TextB{}
Othws, $\exists\,d\in D^+,\,0<u<d<s,\,\forall s\notin D,$ ctradic.\vspace{-2pt}
\SepLine

%\BulletPointX Now we show ${}^+D=\Bra{u:0<u<s,\,\forall s\notin D}=D^+.$ Immed, ${}^+D\supseteq D^+.$\TextB{}
%Becs $\forall u\in\Qbp,\,u\not\in D^+\Rightarrow u\geqslant b,\,\exists\,b\notin D\Rightarrow u\notin{}^+D.$
%\SepLine

%\BulletPointX For $C$ not posi and $D$ posi, to make $CD=DC,$ we define\TextB{}
%$CD=\Bra{a:a<rs,\,\forall s\not\in C,\,s\in D^-}=\Bra{a:a\leqslant cd,\;\forall c\in C^+,\,d\in D^-}.$
%\SepLine

%\BulletPointX Thus $CD=DC$ in all cases.
%\SepLine

\BulletPointX For a posi $D,$ \;$\tilde{1}D=\Bra{a:a\leqslant ij<j,\;0<i<1,\;j\in D^+}\subseteq D.$\TextB{}
Now $\BigPar{\tilde{1}D}{^+}\subseteq D^+.$ \;又 $\forall d\in D^+,\,\exists\,\varepsilon>0,\:d+\varepsilon\in D^+\Rightarrow d=\Par{d+\varepsilon}\:\!${\Large$\frac{d}{\:d\,+\,\varepsilon\:}$}${}\in\tilde{1}D.$
\SepLine

\ChEnd
\pagebreak

\ChDecl{Ch0C}{0$\cdot$C}{}

\vspace{2pt}

\Anchor{0C5}\ProblemN{5}{
	\TextA{Supp $a_1,a_2,\dots$ is a seq in $\Qbb,$ and $\Sup\Bra{a_1,a_2,\dots}=\sqrt{2}.$}
	\TextA{Prove $\Sup\Bra{a_{n},a_{n+1},\dots}=\sqrt{2}$ \;for all $n\in\Nbp.$}
}Becs the sup not in seq $\Rightarrow$ infily many disti elem.\parSol{}
$\forall a_i,\,\exists\,a_j,\:a_i<a_j<\sqrt{2}.$ \;For $a_{n+k},$ choose $a_i>a_{n+k}.$ If $i\in\;\!\!\Bra{1,\dots,n},$ then choose $a_j>a_i.$\parSol{}
After at most $\Par{n+1}$ steps, we have $a_{m}$ with $m>n.$ Thus $\forall a_{n+i},\,\exists\,a_{n+j},\:a_{n+i}<a_{n+j}<\sqrt{2}.$\PfEnd
\SepLine

\ProblemB[]{
	\TextB{Supp nonempty $A\subseteq\Rbb.$}
}
\Anchor{0CT1}\ProblemBX{\TipsN{1}}{
	\TextA{Define $-A=\Bra{-a:a\in A}\Rightarrow{-\Par{{-A}}}=A.$ \,Prove $\Sup\Par{{-A}}=-\inf A.$}
}$-b$ is an upper bound of $-A\Longleftrightarrow\forall a\in A,\,-a\leqslant-b\Longleftrightarrow a\geqslant b\Longleftrightarrow b$ is a lower bound of $A.$\parSol{}
Thus \,$-b_M=\Sup\Par{{-A}}\Longleftrightarrow-b_M\leqslant -b\Longleftrightarrow b_M\geqslant b\Longleftrightarrow b_M=\inf A.$\PfEnd
\SepLine

\Anchor{0CT2}\ProblemBX{\TipsN{2}}{
	\TextA{Show if $x\in\Rbb,$ {\tgnr\large(a)} $\sup A>x\Rightarrow\exists\,a\in A,\:a>x,$ \;{\tgnr\large(b)} $\inf A<x\Rightarrow\exists\,a\in A,\:a<x.$}
}(a) $\nexists\,a>x\Longleftrightarrow\forall a\in A,\,a\leqslant x.$ Then by def of sup.\parSol{\Ha}
\Or By (b), $\inf\Par{{-A}}=-\sup A<-x\Rightarrow\exists\,{-a}\in A,\:-a<-x.$ \hfill Simlr for (b).\Blind{\qquad}\PfEnd
%\vspace{4pt}\parSol{}
%(b) $\nexists\,a<x\Longleftrightarrow\forall a\in A,\,a\geqslant x.$ Then by def of inf.\parSol{\Hb}
%\Or By (a), $\sup\Par{{-A}}=-\inf A>-x\Rightarrow\exists\,{-a}\in A,\:-a>-x.$\PfEnd
\SepLine

\Anchor{0C6}\ProblemN{6}{
	\TextA{Supp $A,B\subseteq\Rbb$ has infily many disti elem, so has $A+B=\Bra{a+b:a\in A,b\in B}.$}
	\TextA{Prove $\Sup\Par{A+B}=\sup A+\sup B,\,$ and \,$\Inf\Par{A+B}=\inf A+\inf B.$}
}$\inf A+\inf B\leqslant a+b\leqslant\sup A+\sup B\Rightarrow\Sup\Par{A+B}\leqslant\sup A+\sup B,\,\inf A+\inf B\leqslant\Inf\Par{A+B}.$\vspace{2pt}\parSol{}
$\sup A+\sup B>\Sup\Par{A+B}\Longleftrightarrow\sup A>\Sup\Par{A+B}-\sup B$\parSol{}
$\Longleftrightarrow\exists\,a+\sup B>\Sup\Par{A+B}\Longleftrightarrow\sup B>\Sup\Par{A+B}-a\Longleftrightarrow\exists\,a+b>\Sup\Par{A+B}.$ Ctradic.\vspace{2pt}\parSol{}
%$\inf A<\Inf\Par{A+B}-\inf B\Longleftrightarrow\inf A+\inf B<\Inf\Par{A+B}$\parSol{}
%$\Longleftrightarrow\exists\,a+\inf B<\Inf\Par{A+B}\Longleftrightarrow\inf B<\Inf\Par{A+B}-a\Longleftrightarrow\exists\,b+a<\Inf\Par{A+B}.$ Ctradic.
Simlr for $\Inf\Par{A+B}\in A+B.$ \,\Or Apply to $-A-B,$ becs $\Sup\Par{{-A}}=-\inf A.$\PfEnd
%Then $\forall a\in A,b\in B,\,a+b\in A+B\Rightarrow\exists\,c+d\in A+B,\,c+d>a+b.$\parSol{}
%If $\sup A\in A.$ Then $\sup A+d>a+b.$
%Supp $A,B$ both have infily many elem. Supp sups and infs are not in there sets.\parSol{}
%$\forall a\in A,b\in B,\,$\parSol{}
\SepLine

%\Anchor{0C7}\ProblemN{7}{
%	\TextA{Supp $X\neq\emptySet,\,f,g:X\rightarrow\Rbb$ are functions. \,Prove $\sup_X\!\Par{\,f+g}\leqslant\sup_X f+\sup_X g.$}
%}$ $
%\SepLine

%\Anchor{0C9}\ProblemN{9}{
%	\TextA{Prove the ordered field of rational functions with coeffs in $\Rbb$ does not satisfy Archimedean Property.}
%}
%\SepLine

%\Anchor{0C14}\ProblemN{14}{
%	\TextA{Supp $A,B\subseteq\Interval{[}{]}{{-\infty,\infty}}$ are intervals, and $A\cap B\neq\emptySet.$ \,Prove $A\cup B$ are intervals.}
%}
%\SepLine

%\Anchor{0C15}\ProblemN{15}{
%	\TextA{Supp $I\subseteq\Rbb$ is an interval containing more than one number. Prove $\Inf\Par{I\cap\Qbb}=\inf I.$}
%}$\inf I<\Inf\Par{I\cap\Qbb}\Rightarrow\exists\,x\in I,\,x<\Inf\Par{I\cap\Qbb}.$ While $\exists\,n\in\Nbp,\,1\big/n<x.$\PfEnd
%\SepLine

\Anchor{0C16}\ProblemN{16}{
	\TextA{Supp $\Rbb{_1},\Rbb{_2}$ are complete ordered fields. Define $\varphi_1,\varphi_2$ as in [0.11].\vspace{2pt}}
	\TextA{Define $\mR_1\Par{a}=\Bra{q\in\Qbb:\varphi_1\Par{q}\leqslant a}\Rightarrow\sup_{\mR_1\SmallPar[0.5pt]{a}}\varphi_1=a.$ \;Define $\psi_1\Par{a}=\sup_{\mR_1\SmallPar[0.5pt]{a}}\varphi_2.$\vspace{4pt}}
	\PrePa\TextA{Show $\psi=\psi_1:\Rbb{_1}\rightarrow\Rbb{_2}$ is one-to-one. \;{\tgnr\large(b)} Show $\psi\Par{0}=0,\,\psi\Par{1}=1.$\vspace{2pt}}
	\PrePc\TextA{Show $\psi\Par{a\pm b}=\psi\Par{a}\pm\Par{b},$ and $\psi\Par{ab^{-1}}=\psi\Par{a}\psi\Par{b}{^{-1}}.$\vspace{2pt}}
	\PrePd\TextA{Supp $a\in\Rbb{_1}.$ Show $a>0\Longleftrightarrow\psi\Par{a}>0.$\vspace{2pt}}
}(a) Define $\mR_2\Par{c}=\Bra{q\in\Qbb:\varphi_2\Par{q}\leqslant c}\Rightarrow\sup_{\mR_2\SmallPar[0.5pt]{c}}\varphi_2=c.$\vspace{1pt}\parSol{\Ha}
Define $\psi_2\Par{c}=\sup_{\mR_2\SmallPar[0.5pt]{c}}\varphi_1.$ Then $\psi_2:\Rbb{_2}\rightarrow\Rbb{_1}$ well-defined.\vspace{1pt}\parSol{\Ha}
Note that $\mR_2\BigPar{\psi_1\Par{a}}=\Bra{q\in\Qbb:\varphi_2\Par{q}\leqslant\sup_{\mR_1\SmallPar[0.5pt]{a}}\varphi_2}=\mR_1\Par{a}.$\vspace{1pt}\parSol{\Ha}
Now $\psi_2\BigPar{\psi_1\Par{a}}=\sup_{\mR_1\SmallPar[0.5pt]{a}}\varphi_1=a.$ \;Rev the roles of $\Rbb_1,\Rbb_2.$\vspace{4pt}\parSol{}
(b) Note that $\varphi\Par{q}<0\Longleftrightarrow q<0,$ and $\varphi\Par{0}=0.$\parSol{\Hb}
$\forall q\in\mR_1\Par{1}=\Bra{1\big/m\in\Qbb:m\in\Nbp,\varphi_1\Par{1\big/m}=\Par{1+\dots+1}{^{-1}}\leqslant 1}\cup\Par{\Qbb\Backslash\Qbp},$ \,$\varphi_2\Par{q}\leqslant 1.$\vspace{2pt}\parSol{}
(c) $\mR_1\Par{a\pm b}=\Bra{p\pm q\in\Qbb:\varphi_1\Par{p}\pm\varphi_1\Par{q}\leqslant a\pm b}.$\parSol{\Hc}
$\mR_1\Par{ab^{-1}}=\Bra{pq^{-1}\in\Qbb:\varphi_1\Par{q}\cdot\varphi_1\Par{q}{^{-1}}\leqslant ab^{-1}}.$\vspace{3pt}\parSol{}
(d) $a>0\Longleftrightarrow\exists\,n\in\Nbp,\:1\big/n<a\Longleftrightarrow\psi_1\Par{a}>0.$\PfEnd
\SepLine

\ChEnd