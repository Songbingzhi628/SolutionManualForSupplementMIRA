\ChDecl{Ch0B}{0$\cdot$B}{\quad{\ANote $C,D$ are Dedekind cuts.\;\;Numbers used here are always rational.}}

\vspace{4pt}

\BulletPointX Define $\tilde{q}=\Bra{a:a<q},$ and $-\tilde{q}=\widetilde{-q}=\Bra{a:a<-q}.$\TextB{}
Then $\tilde{0}=\Bra{a:a<0}=\Qbb\Backslash{\Qbs}\Rightarrow-\tilde{0}=\Bra{a:a<-b\leqslant0}=\tilde{0}.$\par\vspace{3pt}
\BulletPointX Define $-D=\Bra{a:a<-b,\,b\not\in D}=\Bra{{-a}:-a<-b\Longleftrightarrow a>b,\:b\not\in D}.$\TextB{}
$-\Par{{-D}}=-\Bra{a:a<-b,\,b\notin D}=\Bra{c:c<-a,\,a\geqslant -b,\,\forall b\notin D}=\Bra{c:c<b,\,\forall b\notin D}=D.$\TextB{}
{\FontSmall The last equa is becs (a) $d\notin D\Rightarrow\exists\,b\notin D,\,d\geqslant b,$ and (b) $d\in D\Rightarrow$ if $\exists\,b\notin D$ suth $d\geqslant b,$ then $b\in D,$ ctradic.}\vspace{-2pt}
\SepLine

%\Anchor{0BT1}\ProblemBX{\TipsN{1}}{
%	\TextA{Supp $D_1=\Bra{x\in\Qbb:x<a},\:D_2=\Bra{y\in\Qbb:y<b}.$}
%	\TextA{Prove $D_1+D_2=\Bra{q\in\Qbb:q<a+b}.$}
%}Becs $D_1+D_2=\Bra{x+y:x+y<x+b<a+b,\,x\in D_1,\,y\in D_2}.$\parSol{}
%Supp $x\in D_1.$ Let $\delta=\Par{a-x}\big/2\Rightarrow x<x'=x+\delta<a.$ Let $y=b-\delta\in D_2.$\parSol{}
%Then $x+b=x'+y<a+b.$ Let $\delta'=\Par{a-x'}\big/2\Rightarrow x+b<x''+y<a+b.$\PfEnd\vspace{3pt}\parSol{}
%\Or $D_1+D_2\subseteq\Bra{q\in\Qbb:q<a+b}.$ We show $\forall q\in RHS,\exists\,\Par{x,y}\in D_1\times D_2,\:x+y=q.$\parSol{}
%Let $\delta=\Par{a+b-q}\big/2.$ Let $x=a-\delta\in D_1,\,y=b-\delta\in D_2\Rightarrow x+y=a+b-\delta=q.$\PfEnd
%\SepLine

%\Anchor{0BT2}\ProblemBX{\TipsN{2}}{
%	\TextA{Supp $D$ is Ddkd cut. Prove $\forall b\in\Qbb\Backslash D,\:d<b,\forall d\in D.$}
%}Asum $\exists\,\Par{b,d}\in\Par{\Qbb\Backslash D}\times D,\:d\geqslant b.$ By def of Ddkd cut.\PfEnd
%\SepLine

\Anchor{0BT}\ProblemBX{\Tips}{
%	\PrePa\TextA{If $x\in\Qbb$ suth $\forall d\in D,\,d\leqslant x,$ then $x\in\Qbb\Backslash D.$\hfill\FontNorm\tgnr If not, by def, ctradic.}
%	\PrePb\TextA{If $\forall d\in D,\,\exists\,\varepsilon\in\Qbp,\,d+\varepsilon\in D.$\hfill\FontNorm\tgnr Supp $d<d'\in D\Rightarrow d'-d=\varepsilon\in\Qbp.$}
	\TextA{Prove $\forall\varepsilon>0,\,\exists\,b\not\in D$ \,suth\, $b-\varepsilon\in D.$}
%	\PrePb\TextA{Prove $\forall q\in\tilde{0}\cap\Qbp,\,\exists\,d$}
}Asum $\exists\,\varepsilon>0$ suth $\nexists\,b\not\in D,\,b-\varepsilon\in D\Longleftrightarrow\forall b\not\in D,\,b-\varepsilon\not\in D.$\parSol{}
Then $\Par{b-\varepsilon}-\dots-\varepsilon=b-n\cdot\varepsilon\not\in D$ for any $n\in\Nbp.$\parSol{}
Now $\forall d\in D,\,\exists\,n\in\Nbp$ suth $b-n\cdot\varepsilon<d\Rightarrow b-n\cdot\varepsilon\in D,$ ctradic.\PfEnd
%\ACoro Since $D$ does not contain the largest elem, $\forall\varepsilon\in\Qbp,\,\exists\,b\in\Qbb\Backslash D,\;b-\varepsilon<d$ for some $d\in D.$
\SepLine

\Anchor{0B1}\ProblemN{1}{
	\TextA{Prove {\tgnr\large(a)} $D+\tilde{0}=D,$ \;{\tgnr\large(b)} $-D$ is Dedekind cut, and $D+\Par{{-D}}=\tilde{0}.$}
}(a) $\forall d\in D,\,\exists\,\varepsilon>0,\:d+\varepsilon\in D\Rightarrow\Par{d+\varepsilon}+\Par{{-\varepsilon}}\in D+\tilde{0}.$\vspace{2pt}\parSol{}
(b) Asum $x\in -D$ is the largest elem of $-D\Rightarrow\exists\,b\not\in D,\:x<-b\Rightarrow 0<-b-x.$\parSol{\Hb}
Let $\delta=\Par{{-b-x}}\big/2\Rightarrow0<\delta<-b-x\Rightarrow x<x+\delta<-b.$\parSol{\Hb}
Thus by def, $x+\delta\in -D,$ ctradic the max of $x\in -D.$ \,Hence $-D$ is Ddkd cut.\vspace{3pt}\parSol{\Hb}
$D+\Par{{-D}}=\Bra{x+y:x+y<x-b,\:x\in D,\,b\not\in D}.$\parSol{\Hb}
Supp $a\in\tilde{0}\Rightarrow-a>0.$ By \TIPS, $\exists\,b\not\in D$ \,suth\, $b+a\in D.$\parSol{\Hb}
Note that $b<b-a\not\in D\Rightarrow-b>-b+a\in -D.$ \,Then $\Par{{-b+a}}+\Par{b+a}=2\:\!a<0.$\parSol{\Hb}
Thus $\forall a\in\tilde{0},\,\exists\,b\not\in D,\;d=b+\frac{\:1\:}{2}a\in D,\,c=-b+\frac{\:1\:}{2}a\in-D\Rightarrow c+d=a\in D+\Par{{-D}}.$\PfEnd
%\ACoro $-\Par{{-D}}+\Par{{-D}}=\tilde{0}\Rightarrow -\Par{{-D}}=D,$ by the uniqnes.
%$\Bra{a_1+a_2:a_1,a_2\in D}=D+D=\tilde{0}=\Bra{a:a<0}\Longrightarrow\forall a\in D,\,a+a<0\Longleftrightarrow a<0.$
\SepLine

\Anchor{0B3}\ProblemN{3}{
	\TextA{Show $C\subsetneq D\Longleftrightarrow D-C${\FontNorm ${}=\Bra{d-y:d\in D,\:y>x,\:x\not\in C}$\:\!} posi.\vspace{1pt}}
}(a) $C\subsetneq D\Rightarrow\exists\,x\in D\Backslash C\Rightarrow\exists\,y\in D,\,y>x\Rightarrow\exists\,d\in D,\,d>y\Longleftrightarrow 0<d-y\in D-C.$\vspace{2pt}\parSol{}
(b) $0\in D-C\Rightarrow\exists\,y>x\notin C,\;y\in D\Rightarrow\forall c\in C,\,c<x<y\in D\Rightarrow C\subseteq D.$ \,又 $D-C\neq\tilde{0}.$\PfEnd
\SepLine

%\Anchor{0B'1}\ProblemB{
%	\TextB{Prove $D=-D\Longleftrightarrow D=\tilde{0}.$}
%}$\Bra{a_1+a_2:a_1,a_2\in D}=D+D=\tilde{0}=\Bra{a:a<0}\Rightarrow\forall a\in D,\,a+a<0\Longleftrightarrow a<0.$\parSol{}
%\SepLine

\Anchor{0B5}\ProblemN{5}{
	\TextA{Prove {\tgnr\large(a)} $D$ posi $\Rightarrow -D$ not posi, \;{\tgnr\large(b)} non0 $-D$ not posi $\Rightarrow D$ posi.}
}(a) $0\notin\Bra{a:a<-b,\,b\notin D}\Longleftrightarrow\nexists\,b\notin D,\,0<-b\Longleftrightarrow\forall b\notin D,\,b\geqslant0\Longleftarrow 0\in D.$\vspace{2pt}\parSol{}
(b) Becs $\tilde{0}$ is the largest non posi cuts. Thus $-D\neq\tilde{0}\Rightarrow-D\subsetneq\tilde{0}\Rightarrow\tilde{0}-\Par{{-D}}=D$ posi.\parSol{\Hb}
\Or $\exists\,a<0,\,a\notin -D=\Bra{a:-a>b,\,b\notin D}\Longleftrightarrow\nexists\,b\notin D,\,-a>b\Longleftrightarrow\forall b\notin D,\,0<-a\leqslant b.$\PfEnd
\SepLine

\BulletPointX Define $D^+=\Bra{d\in D:d>0}=D\cap\Qbp.$ \,Then $D^+\neq\emptySet\Longleftrightarrow\Qbb\Backslash\Qbp\subsetneq D\Longleftrightarrow 0\in D\Longleftrightarrow D$ posi.\vspace{2pt}\TextB{}
Define $D^-=\Bra{r\not\in D:r\leqslant 0}=\Par{\Qbb\Backslash D}\cap\Par{\Qbb\Backslash\Qbp}=\Qbb\Backslash\Par{D\cup\Qbp}.$\TextB{}
%\BulletPointX Define $D^\circ=\Bra{a:a\leqslant b,\,\forall b\notin D}.$ Now $D^\circ=D\Longleftrightarrow\nexists\,\min\Qbb\Backslash D.$\TextB{}
(a) $D^-=\zeroSubs\Longleftarrow D=\tilde{0}.$ \;Convly, $\Bra{r\notin D:r\leqslant0}=\zeroSubs\Longrightarrow\Qbb\Backslash D=\Qbs.$\TextB{}
(b) $D^-=\emptySet\Longleftrightarrow D\cup\Qbp=\Qbb\Longleftrightarrow\Qbb\Backslash\Qbp\subseteq D\Longleftrightarrow 0\in D\Longleftrightarrow D$ posi. \hfill\ACoro $D$ not posi $\Longleftrightarrow 0\in D^-.$\TextB{}
(c) $\Par{D^-}{^-}=\Bra{r\in D:r\leqslant0}=\Qbb\Backslash D^+.$ \hfill\ACoro $D$ not posi $\Longleftrightarrow\Par{D^-}{^-}=D.$
\SepLine

\BulletPointX $\Par{{-D}}{^+}=\Par{{-D}}\cap\Qbp=\Bra{a:0<a<-b,\:b\not\in D\Longleftrightarrow b\in D^-\nonzero}.$\TextB{}
$\Par{{-D}}{^-}=\Par{\Qbb\Backslash{-D}}\cap\Par{\Qbb\Backslash\Qbp}=\Bra{a:0\geqslant a\geqslant-b,\:\forall b\not\in D}.$
%\vspace{3pt}\TextB{}
%Define $\tilde{1}=\Bra{a:a<1}\Rightarrow\tilde{1}{^+}=\Bra{a:0<a<1},\:\tilde{1}{^-}=\Bra{a:a\leqslant 0}=\tilde{0}\cup\zeroSubs.$\TextB{}
%Then $-\tilde{1}=\Bra{a:-a>1\Longleftrightarrow a<-1}$ not posi.
\SepLine

\BulletPointX For $C,D$ posi, \,define $CD=\Bra{a:a\leqslant cd,\,c\in C^+,\,d\in D^+}=\Bra{cd:c\in C^+,d\in D^+}\cup\Par{\Qbb\Backslash\Qbp}.$\vspace{2pt}\TextB{}
$\Bra{cd:c\in C^+,d\in D^+}=CD\cap\Qbp=\Par{CD}{^+}.$ \;Note that `$a\leqslant cd$' here is equiv to `$a<cd$'.%\vspace{4pt}\par
\SepLine

\BulletPointX For $-C,-D$ posi, \,define $CD=\Par{{-C}}\Par{{-D}}=\Bra{cd:c\in\Par{{-C}}{^+},d\in\Par{{-D}}{^+}}\cup\Par{\Qbb\Backslash\Qbp}.$\TextB{}
$CD=\Bra{0<cd<\Par{{-r}}\Par{{-s}}:r\in C^-\nonzero,\,s\in D^-\nonzero}\cup\Par{\Qbb\Backslash\Qbp}=\Bra{a:a<rs,\,r\in C^-,\,s\in D^-}.$\vspace{2pt}\TextB{}
If $C,-C$ not posi $\Rightarrow C=\tilde{0},$ then with the asum $\tilde{0}D=\tilde{0},$ \,it still holds. \,Simlr for $D.$
\SepLine

\BulletPointX For $D$ posi, \,define $D^{-1}=\Bra{a:a<1\big/b,\;b\notin D}\Rightarrow DD^{-1}=\Bra{a:a\leqslant d\big/b<1,\;b\notin D,\,d\in D^+}=\tilde{1}.$\TextB{}
The last equa holds becs $\forall a\in\tilde{1}\cap\Qbp,\:\exists\,d\in D^+$ suth $b=d\big/a\notin D\Rightarrow d\big/b=a\in DD^{-1}.$\vspace{3pt}\par
\BulletPointX For non0 $D$ not posi, \,define $D^{-1}=-\Par{{-D}}{^{-1}}=-\Bra{a:a<1\big/b,\:\exists\,b\notin -D\Longleftrightarrow\exists\,b\geqslant -s,\,\forall s\notin D}$\TextB{}
$=\Bra{a:a<-x,\,\exists\,x\geqslant 1\big/b,\,\forall b\text{\;suth\;}b\geqslant -s,\,\forall s\notin D}=\Bra{a:a<-1\big/b,\,\forall b\text{\;suth\;}b\geqslant -s,\,\forall s\notin D}$\TextB{}
$=\Bra{a:a<1\big/b,\,\forall b\text{\;suth\;}b\leqslant s,\,\forall s\notin D}\neq\Bra{a:a<1\big/s,\,\forall s\notin D}.$\vspace{2pt}\TextB{}
Let \,$b_1<\cdots<b_m<\cdots\leqslant s,\,\forall s\notin D\Longleftrightarrow\forall s\in D^-\Rightarrow$ each \:\!$s_j,b_k<0\Rightarrow 1\big/s_j\leqslant\cdots<1\big/b_m<\cdots<1\big/b_1.$\TextB{}
Thus `$a<1\big/b$' is equiv to `$a<1\big/s,\,\exists\,s\in D^-$'. Hence $D^{-1}=\Bra{a:a<1\big/b,\;b\in D^-}.$\vspace{4pt}\TextB{}
$DD^{-1}=\Bra{a:a<rs,\,\exists\,r\in D^-,\,\exists\,s\text{\;suth\;}0\geqslant s\geqslant1\big/b,\,\forall b\in D^-}\subseteq\tilde{1}.$\TextB{}
Asum $\exists\,x$\;suth\;$rs\leqslant x<1,\,\forall r,s.$ \,Let $D\not\ni\cdots\leqslant b_m<\cdots<b_1\leqslant0,$\;and\;$D\not\ni\cdots\leqslant r_{\:\!\!m}<\cdots<r_{\:\!\!1}\leqslant0.$\TextB{}
$1\big/b_1<\cdots<1\big/b_m\leqslant\cdots\leqslant\cdots\leqslant s_n<\cdots<s_1\leqslant0,$\;and \,$r_{\:\!\!m}\big/b_m\geqslant\cdots\geqslant\cdots\geqslant r_{\:\!\!m}\,s_n>\cdots>r_{\:\!\!j}\,s_k>\cdots>r_{\:\!\!1}\,s_1.$\vspace{3pt}\TextB{}
Let $r_{\:\!\!m}=b_m.$ Ctradic. \hfill\Or $DD^{-1}=D\Sbra{\Par{{-\tilde{1}}}\BigPar{{-D}}{^{-1}}}=\Sbra{D\Par{{-\tilde{1}}}}\BigPar{{-D}}{^{-1}}=\Par{{-D}}\Par{{-D}}{^{-1}}=\tilde{1}.$
\SepLine

\BulletPointX For $C$ not posi and $D$ posi, we expect that $CD$ not posi. Consider $C$ and $-D$ both not posi.\TextB{}
$CD=-\Sbra{C\Par{{-D}}}=-\Bra{a:a<rt,\:r\in C^-,\,t\in\Par{{-D}}{^-}}=\Bra{{-a}:a>b,\,b\geqslant rt,\:\forall r\in C^-,\,t\in\Par{{-D}}{^-}}$\TextB{}
\Blind{$CD$}${}=\Bra{{-a}:a>rt,\:\forall r\in C^-,\,\forall t\text{\;suth\;}0\geqslant t\geqslant -s,\,\forall s\notin D}$\TextB{}
\Blind{$CD$}${}=\Bra{a:a<ru,\:\forall r\in C^-,\,\forall u\text{\;suth\;}0\leqslant u\leqslant s,\,\forall s\notin D}.$\hfill{\FontSmall \BigPar{$r\leqslant 0<s,\;rs\leqslant ru=-rt\leqslant0\leqslant rt\leqslant-rs.$}}\vspace{3pt}\par
\BulletPointX Note the `$0\leqslant u$'. Becs $C^-\neq\emptySet\Rightarrow 0\in C^-.$ If it is to be exactly $CD=\Bra{a:a<0}$, \,then $C^-=\zeroSubs,$\TextB{}
for if not, $\exists\,u>0,$ and $\exists\,r\in C^-\nonzero,$ suth $\exists\,a<ru<0.$ Hence `$0\leqslant u$' is actually `$0<u$'.\vspace{3pt}\par
\BulletPointX `$u\leqslant s$' cannot be abbreviated as in $\Bra{{-a}:a>-rs,\,\forall s\not\in D,\,r\in C^-}=\Bra{{a}:a<rs,\,\forall s\not\in D,\,r\in C^-}.$\TextB{}
`$u\leqslant s$' cannot be `$u<s$', becs here $rs<ru\Rightarrow\exists\,a=rs.$\vspace{4pt}\par
\BulletPointX Note that \,$\Bra{u:0<u\leqslant s,\,\forall s\notin D}=\MathLeftBrace{l}{\!D^+\cup\Bra{{\min\Qbb\Backslash D}},\,\,\text{if it exis,}\\\!D^+,\hfill\text{othws.}}$ \quad Denote it by $D^\oplus=D^\otimes\nonzero.$\vspace{8pt}
\SepLine

\BulletPointX For $C$ not posi and $D$ posi. If $C=\tilde{0},$  then $CD=-\Sbra{C\Par{{-D}}}=-\tilde{0}.$ \,Now consider $-C$ and $D$ both posi.\vspace{1pt}\TextB{}
But\, $CD=-\Par{{-C}}D=-\Bra{a:a\leqslant cd,\,\,c\in\Par{{-C}}{^+},\,d\in D^+}\neq\Bra{a:a<-cd,\;\forall c\in\Par{{-C}}{^+},\,d\in D^+}.$\TextB{}
Altho ``$a\leqslant cd$'' is equiv to ``$a<cd$'' so that \,$b\notin\Par{{-C}}D\Rightarrow b\geqslant cd,$ which is actually $b>cd,\,\forall c,d.$\TextB{}
And\, $a<-b<-cd,\,\forall c,d\Rightarrow\forall a,\,\exists\,x$ suth $a<x<-cd,\,\forall c,d.$ \,While {\;\!$a$\:\!} can be the `boundary' in $RHS$.
\SepLine

%\BulletPointX For $C$ posi and $D$ not posi. If $D=\tilde{0},$ then $CD=-\Par{{-C}}\,\tilde{0}=-\tilde{0}.$ \,Now consider $C$ and $-D$ both posi.\vspace{1pt}\TextB{}
%$CD=-\Sbra{C\Par{{-D}}}=-\Bra{a:a<cd,\,c\in C^+,\,d\in\Par{{-D}}{^+}}=\Bra{{-a}:a>b,\,b\geqslant cd,\,\forall c\in C^+,\,d\in\Par{{-D}}{^+}}$\TextB{}
%\Blind{$CD$}${}=\Bra{{a}:a<-cd,\,\forall c\in C^+,\,\forall d\text{\;suth\;}$\uline{$0<d<-s,\,\exists\,s$}${}\in D^-}$\hfill{\FontSmall\BigPar{$sd<{}$\uline{$-cd<0$}}}\TextB{}
%$\Blind{CD}=\Bra{a:a<ct,\,\forall c\in C^+,\,\forall t\text{\;suth\;}s<t<0,\,\exists\,s\in D^-}$\vspace{4pt}\par
%\BulletPointX Note that \,$\Bra{t:s<t<0,\,s\in D^-}=\MathLeftBrace{l}{\!D^-\Backslash\Bra{0,\min\Qbb\Backslash D},\,\,\text{if it exis,}\\\!D^-\nonzero,\hfill\text{othws.}}$\vspace{8pt}
%\SepLine

\BulletPointX $LHS=\Bra{a:a<ru,\;\forall r\in C^-\!,\;\forall u\in D^\oplus},\;\Bra{cs:c\in C,\,s\notin D}=RHS.$\TextB{}
Becs \,$cs\leqslant cu<ru.$ \,We show $LHS\subseteq RHS.$ \,Let $c_1<\cdots<c_n<\cdots\in C$, and $s_1>\cdots>s_m\geqslant\cdots\notin D.$\TextB{}
Then \uline{$c_1s_1<\cdots<c_n\:\!s_m<\cdots<ru,\,\forall r,u$ as in $LHS.$ \;Thus $\,a\in LHS\Rightarrow\exists\,a<c_j\:\!s_k.$}\PfEnd\vspace{2pt}\TextB{}
\Or Note that in $LHS$, \,`$a<ru$' is equiv to `$a<rs,\,\exists\,s\notin D.$' Now $LHS=\Bra{a:a\big/s\in C,\,\exists\,s\notin D}.$\PfEnd
\SepLine

\BulletPointX For $C$ posi and $D$ not posi. If $D=\tilde{0},$ then $CD=-\Sbra{\Par{{-C}}D}=-\tilde{0}.$\TextB{}
%Now consider $-C$ not posi and $-D$ posi. \TextB{}
$CD=\Par{{-C}}\Par{{-D}}=\Bra{a:a<ru,\;\forall r\in\Par{{-C}}{^-},\;\forall u\in\Par{{-D}}{^\oplus}}$\hfill $\Qbb\Backslash{{-D}}=\Bra{s:s\geqslant -y,\;\forall y\notin D}$\TextB{}
$\Blind{CD}=\Bra{a:a<ru,\;\forall r\text{\;suth\;}\forall x\notin C,\,0\geqslant r\geqslant -x,\;\forall u\text{\;suth\;}0<u\leqslant s,\,\forall s\geqslant -y,\,\forall y\notin D}$\TextB{}
$\Blind{CD}=\Bra{a:a<\Par{{-r}}\Par{{-u}},\;\forall r\text{\;suth\;}\forall x\notin C,\,0\leqslant -r\leqslant x,\;\forall u\text{\;suth\;}y\leqslant -u<0,\,\forall y\notin D}$\TextB{}
$\Blind{CD}=\Bra{a:a<ru,\;\forall r\text{\;suth\;}\forall x\notin C,\,0\leqslant r\leqslant x,\;\forall u\in D^-}=\Bra{a:a<ru,\,\forall r\in C^\oplus,\,\forall u\in D^-},$ simlr.
%Becs $D^-\neq\emptySet\Rightarrow 0\in D^-.$ If it is to be exactly $CD=\Bra{a:a<0},$ then $D^-=\Bra{0},$\TextB{}
%for if not, $\exists\,r>0,$ and $\exists\,u\in D^-\nonzero,$ suth $\exists\,a<ru<0.$ Hence `$0\leqslant r$' is actually `$0<r$'.
\SepLine

\BulletPointX We show $-D=\Bra{a:a<-b,\,b\notin D}=\Par{{-\tilde{1}}}D.$\TextB{}
For $D$ posi, $RHS=\Bra{a:a<ru,\,\forall r\text{\;suth\;}{-1}\leqslant r\leqslant0,\,\forall u\in D^\oplus}=\Bra{a:a<-u,\,\forall u\in D^\oplus}\supseteq -D.$\TextB{}
Supp $x$ suth $-b\leqslant x<-u,\,\forall b\notin D,\,\forall u\text{\;suth\;}{-s}\leqslant-u<0,\:\!\forall s\notin D\Longrightarrow -s\leqslant x<-u.$ Let $-u=x.$\TextB{}
For $D$ not posi, $RHS=\Bra{a:a<rb,\,\exists\,r\text{\;suth\;}{-1}\leqslant r\leqslant0,\,b\in D^-}=\Bra{a:a<-b,\;0\geqslant b\notin D}=-D.$
\SepLine

\BulletPointX We show $\tilde{1}\:\!D=D.$ \;For $D$ not posi, immed. \,Othws, \;$\tilde{1}\:\!D=\Bra{a:a\leqslant ij<j,\;0<i<1,\;j\in D^+}\subseteq D.$\vspace{3pt}\TextB{}
Now $\BigPar{\tilde{1}\:\!D}{^+}\subseteq D^+.$ \;又 $\forall d\in D^+,\,\exists\,\varepsilon>0,\:d+\varepsilon\in D^+\Rightarrow d=\Par{d+\varepsilon}\:\!${\Large$\frac{d}{\:d\,+\,\varepsilon\:}$}${}\in\BigPar{\tilde{1}\:\!D}{^+}.$
\SepLine

\ProblemN{4}{
	\TextA{Supp $B,C,D$ non0 Dedekind cuts. Show $\Par{BC}D=B\Par{CD},$ $B\Par{C+D}=BC+BD.$}
}We discuss in cases.\vspace{8pt}\par\quad
{\small\begin{tabular}{|c|c|c|c|c|c|c|}
	\hline
	\Backslash & 1 & 2 & 3 & 4 & 5 & 6 \\
	\hline
	$B$  & + & + & + & - & - & -\\
	\hline
	$C$  & + & + & - & - & + & +\\
	\hline
	$D$  & + & - & - & - & - & +\\
	\hline
\end{tabular}}
\vspace{8pt}\par\quad
(1) $\Sbra{\Par{BC}D}{^+}=\Bra{\Par{bc}d:bc\in\Par{BC}{^+},\,d\in D^+}=\Bra{b\Par{cd}:b\in B^+,\,cd\in\Par{CD}{^+}}=\Sbra{B\Par{CD}}{^+}.$\vspace{2pt}\par\quad
\Blind{(1)} $B\Par{C+D}=\Bra{a:a\leqslant bc+bd,\;\exists\,b\in B^+,\,0<c+d\in C+D}.$\par\quad
\Blind{(1)} $BC+BD=\Bra{x:x\leqslant uc,\;u\in B^+,\,c\in C^+}+\Bra{y:y\leqslant vd,\;v\in B^+,\,d\in D^+}$\par\quad
\Blind{(1)} $\Blind{BC+BD}=\Bra{a:a\leqslant uc+vd,\;\exists\,u,v\in B^+,\,c\in C^+,\,d\in D^+}.$ \;Done.\vspace{4pt}\par\quad
(3) $\Par{BC}{^-}=\Bra{r:0\geqslant r\geqslant bc,\,\exists\,c\in C^-,\,\exists\,b\in B^\oplus}.$\par\quad
\Blind{(3)} $\Par{BC}D=\Bra{a:a<rs,\:r\in\Par{BC}{^-},\,s\in D^-}$\par\quad
\Blind{(3)} $\Blind{\Par{BC}D}=\Bra{a:a<bcs,\,\exists\,s\in D^-,\,c\in C^-,\,\exists\,b\in B^\oplus}.$\par\quad
\Blind{(3)} $B\Par{CD}=\Bra{a:a\leqslant bx<bcs,\,\exists\,s\in D^-,\,c\in C^-,\,\exists\,b\in B^+,\text{ and }cs>x\in\Par{CD}{^+}}$\par\quad
\Blind{(3)} $\Blind{B\Par{CD}}=\Bra{a:a<bcs,\,\exists\,s\in D^-,\,c\in C^-,\,\exists\,b\in B^+}.$\vspace{2pt}\par\quad
\Blind{(3)} Note that \,$\Bra{q:q<b,\,b\in B^+}=\Bra{q:q<b,\,\exists\,b\in B^\oplus}.$ \;Done.\vspace{6pt}\par\quad
\Blind{(3)} $B\Par{C+D}=\Bra{a:a<ru,\,\forall r\text{\;suth\;}\forall x\notin B,\,0<r\leqslant x,\;\forall u\text{\;suth\;}0\geqslant u>c+d,\,\forall c\in C,\,d\in D}$\par\quad
\Blind{(3)} $\Blind{B\Par{C+D}}=\Bra{a:a<ru,\,\forall r\in B^\oplus,\;\forall u\text{\;suth\;}0\geqslant u\geqslant c+d,\,\exists\,c\in C^-,\,d\in D^-}$\par\quad
\Blind{(3)} $\Blind{B\Par{C+D}}=\Bra{a:a<r\Par{c+d},\,\forall r\in B^\oplus,\;\forall c\in C^-,\,d\in D^-}.$\par\quad
\Blind{(3)} $BC+BD=\Bra{x:x<pc,\,\forall p\in B^\oplus,\;\forall c\in C^-}+\Bra{y:y<qd,\,\forall q\in B^\oplus,\;\forall d\in D^-}$\par\quad
\Blind{(3)} $\Blind{BC+BD}=\Bra{a:a<pc+qd,\,\forall p,q\in B^\oplus,\;\forall c\in C^-,\,d\in D^-}.$\hfill Done immed.\vspace{6pt}\par\quad
(5) $\Par{BC}{^-}=\Bra{r:0\geqslant r\geqslant bc,\,\exists\,b\in B^-,\,\exists\,c\in C^\oplus}.$ \,$\Par{CD}{^-}=\Bra{r:0\geqslant r\geqslant cd,\,\exists\,d\in D^-,\,\exists\,c\in C^\oplus}.$\par\quad
\Blind{(5)} $\Par{BC}D=\Bra{a:a<rd\leqslant bcd,\,\exists\,d\in D^-,\,b\in B^-,\,\exists\,c\in C^\oplus}.$\par\quad
%\Blind{(6)}\par\quad
\Blind{(5)} $B\Par{CD}=\Bra{a:a<br\leqslant bcd,\,\exists\,b\in B^-,\,d\in D^-,\,\exists\,c\in C^\oplus}.$ \;Done.\vspace{2pt}\par\quad
\Blind{(5)} \Or By commu and (3), $\Par{BC}D=\Par{CB}D=C\Par{BD}=C\Par{DB}=\Par{CD}B=B\Par{CD}.$\vspace{6pt}\par\quad
\Blind{(5)} $BC+BD=\Bra{x:x<bc,\,\forall b\in B^-,\,\forall c\in C^\oplus}+\Bra{y:y<bd,\,\exists\,b\in B^-,\,d\in D^-}$\par\quad
\Blind{(5)} $\Blind{BC+BD}=\Bra{a:a<pc+qd,\,\forall p\in B^-,\,\forall c\in C^\oplus,\text{\;and\;}\exists\,q\in B^-,\,d\in D^-\Rightarrow q\geqslant p}$\par\quad
\Blind{(5)} $\Blind{BC+BD}=\Bra{a:a<pc+y,\,\forall p\in B^-,\,\forall c\in C^\oplus,\text{\;and\;}\forall y\text{\;suth\;}y\geqslant qd,\,\forall q\in B^-,\,d\in D^-}$\par\quad
\Blind{(5)} $\Blind{BC+BD}=\Bra{a:a<p\Par{c+d'},\,\forall p\in B^-,\,\forall c\in C^\oplus,\text{\;and\;}\forall d'\text{\;suth\;}d'\leqslant d,\,\forall d\in D^-}.$\vspace{4pt}\par\quad
%$\Blind{BC+BD}=\Bra{bc:b\in B,\,c\notin C}+\Bra{a:a<bd,\,\exists\,b\in B^-,\,d\in D^-}$\vspace{-18pt}\par\quad
%\qquad\Or\par\quad
%$\Blind{BC+BD}=\Bra{pc+a:a<qd,\;\exists\,p\in B,\,q\in B^-,\,c\notin C,\,d\in D^-}$\par\quad
%$\Blind{BC+BD}=\Bra{x=pc+\Par{x-pc}\,:\,x-pc<qd,\;\exists\,p\in B,\,q\in B^-,\,c\notin C,\,d\in D^-}$\par\quad
%$\Blind{BC+BD}=\Bra{a:a<pc+qd,\;\exists\,c\notin C,\,d\in D^-,\,p\in B,\,q\in B^-\Rightarrow p<q\leqslant0,\text{\,and\;}c\neq0}.$\vspace{4pt}\par\quad
%(I) If $C=-D.$ \,Then $B\Par{C+D}=\tilde{0}.$ \,Note that \,$c\in\Qbb\Backslash C=\Qbb\Backslash{{-D}}=\Bra{c:c\geqslant -d,\:\forall d\in D^-}.$\par\quad\HI
%Thus \,$pc+qd=pc+\Par{qd-pd}+pd\leqslant0,$ \,becs\, $qd-pd\leqslant0$ \,and\: $pc+pd\leqslant0.$ \,Now immed done.\vspace{3pt}\par\quad
\Blind{(5)} (I) If $C+D$ not posi. Then $B\Par{C+D}=\Bra{a:a<bx,\,\exists\,b\in B^-,\,0\geqslant x>c+d,\,\forall\Par{c,d}\in C\times D}.$\par\quad\HI
\Blind{(5)} Rewrite as $\Bra{a:a<t,\,\forall t\text{\;suth\;}t\geqslant bx,\,\forall b\in B^-,\,x\in\Par{C+D}{^-}}.$ \;Done.\vspace{4pt}\par\quad
\Blind{(5)} (II) If $C+D$ posi. Then $B\Par{C+D}=\Bra{a:a<bx,\,\forall b\in B^-,\,x\in\Par{C+D}{^\oplus}}.$\par\quad\HII
\Blind{(5)} If $\Par{C+D}{^\oplus}=\Par{C+D}{^+}.$ \,Then $B\Par{C+D}=\Bra{a:a<bc+bd,\,\forall b\in B^-,\,\forall\Par{c,d}\in C^\oplus\times D^\oplus}.$\par\quad\HII
\Blind{(5)} Othws, $\exists\,s=\min\Qbb\Backslash\Par{C+D}\Rightarrow B\Par{C+D}=\Bra{a:a<bs,\,\forall b\in B^-}.$ \;Done.\pagebreak
\par\quad
(4) $\Par{BC}D=\Bra{xd:d\in D,\:x\geqslant bc,\,\forall b\in B^-,\,c\in C^-}$\par\quad
\Blind{(4)} $\Blind{\Par{BC}D}=\Bra{a:a<bcd,\,\forall d\in D^-,\,\forall b\in B^-,\,c\in C^-}$\par\quad
\Blind{(4)} $\Blind{B\Par{CD}}=\Bra{by:b\in B,\,y\geqslant cd,\,\forall c\in C^-,\,d\in D^-}=B\Par{CD}.$\vspace{3pt}\par\quad
\Blind{(4)} $B\Par{C+D}=\Bra{a:a<t,\,\forall t\text{\;suth\;}t\geqslant b\Par{c+d},\,\forall b\in B^-,\,\Par{c,d}\in C^-\times D^-}$\par\quad
\Blind{(4)} $\Blind{B\Par{C+D}}=\Bra{a:a<t_1,\,\forall t_1\text{\;suth\;}t_1\geqslant bc,\,\forall\Par{b,c}\in B^-\times C^-}$\par\quad
\Blind{(4)} $\Blind{B\Par{C+D}=}+\Bra{a:a<t_2,\,\forall t_2\text{\;suth\;}t_2\geqslant bd,\,\forall\Par{b,d}\in B^-\times D^-}=BC+BD.$ \;Done.\par\vspace{6pt}
\ANote Supp for any $B$ posi, $C$ posi, $D$ not posi, assoc holds.\parNot
Supp $B$ posi, $C$ not posi, $D$ posi. Then $\Par{\overset{+}{B}\overset{-}{C}}\overset{+}{D}=\overset{+}{D}\Par{\overset{+}{B}\overset{-}{C}}=\Par{\overset{+}{D}\overset{+}{B}}\overset{-}{C}=\overset{+}{B}\Par{\overset{+}{D}\overset{-}{C}}.$ \;Convly true.\parNot
Simlr for the case $B$ not posi, $C$ posi, $D$ not posi, equiv to the case $B$ not posi, $C$ not posi, $D$ posi.\vspace{2pt}\par
(2) holds $\Rightarrow\Par{\overset{-}{B}\overset{+}{C}}\overset{+}{D}=\overset{+}{D}\Par{\overset{+}{C}\overset{-}{B}}=\Par{\overset{+}{C}\overset{+}{D}}\overset{-}{B},$ (6) holds. Convly true. Simlr to (3) with (5) in assoc.\par
(5) holds $\Rightarrow\overset{+}{B}\Par{\overset{+}{C}+\overset{-}{D}}=\overset{-}{\Par{{-B}}}\Sbra{{-\Par{\overset{+}{C}+\overset{-}{D}}}}\xlongequal{\text{(5)}}\overset{-}{\Par{{-B}}}\Sbra{\overset{-}{\Par{{-C}}}+\overset{+}{\Par{{-D}}}}\xlongequal{\text{(5)}}\overset{+}{B}\overset{+}{C}+\overset{+}{B}\overset{-}{D},$ by def of multi.\parNot
Thus (5) $\Rightarrow$ (2) in distr. Convly as well.\par\vspace{4pt}\quad
(6) $\Par{BC}{^-}=\Bra{r:0\geqslant r\geqslant bc,\,\exists\,b\in B^-,\,c\in C^\oplus}.$\par\quad
\Blind{(6)} $\Par{BC}D=\Bra{a:a<ru,\,\forall r\in\Par{BC}{^-},\,u\in D^\oplus}=\Bra{a:a<bcd,\,\forall b\in B^-,\,c\in C^\oplus,\,d\in D^\oplus}.$\par\quad
\Blind{(6)} $\Par{CD}{^\oplus}=\Bra{u:0<u\leqslant s,\,\forall s>cd,\,\forall c\in C^+,\,d\in D^+}.$\par\quad
\Blind{(6)} $B\Par{CD}=\Bra{a:a<ru,\,\forall r\in B^-,\,u\in\Par{CD}{^\oplus}}.$ \;Done.\par\vspace{4pt}\quad
\Blind{(6)} $B\Par{C+D}=\Bra{a:a<b\Par{c+d},\,\forall b\in B^-,\,\forall\Par{c,d}\in C^\oplus\times D^\oplus}.$\par\quad
\Blind{(6)} $BC+BD=\Bra{a:a<bc,\,\forall b\in B^-,\,c\in C^\oplus}+\Bra{a:a<bd,\,\forall b\in B^-,\,d\in D^\oplus}.$ \;Done.\vspace{2pt}\par\quad
\Or (2) instead of (6) ? \,\NOTICE that the distr in (6) cannot be shown without (6).\PfEnd
\SepLine
\ChEnd

\vfill\ChDecl{Ch0C}{0$\cdot$C}{}

\vspace{4pt}

\def\sup{{\textup{\tgnr sup}}\;\!}
\def\inf{{\textup{\tgnr inf}}\;\!}

\def\supU{\sup\!\!\;U}
\def\supV{\sup\!\!\;V}

\Anchor{0C2}\ProblemN{2}{
	\TextA{Supp nonempty $U\subseteq V\subseteq\Rbb.$\;Show\;$\supU\leqslant\supV.$
%	and {\tgnr\large(b)} $\inf U\geqslant\inf V.$
	}
}Asum $\supU>\supV\Rightarrow\exists\,t\in U\cap\Interval{(}{]}{{\supV,\supU}}\Rightarrow \supV<t\in V,$ ctradic.\PfEnd
%(b) Asum $\inf U<\inf V\Rightarrow\exists\,t\in U\cap\Interval{[}{)}{{\inf U,\inf V}}\Rightarrow\inf V>t\in V,$ ctradic.\PfEnd
\SepLine

\Anchor{0C5}\ProblemN{5}{
	\TextA{Supp $\Sup\Bra{a_1,a_2,\dots}=\sqrt{2}$ and each $a_k\in\Qbb.$ Prove each $\Sup\Bra{a_{n},a_{n+1},\dots}=\sqrt{2}.$}
}Becs the sup not in seq $\Rightarrow$ infily many disti elem.\parSol{}
$\forall a_i,\,\exists\,a_j,\:a_i<a_j<\sqrt{2}.$ \;For $a_{n+k},$ choose $a_i>a_{n+k}.$ If $i\in\;\!\!\Bra{1,\dots,n},$ then choose $a_j>a_i.$\parSol{}
After at most $n$ steps, we must have $a_{m}$ with $m>n.$ Thus $\forall a_{n+i},\,\exists\,a_{n+j},\:a_{n+i}<a_{n+j}<\sqrt{2}.$\PfEnd
\SepLine

\ProblemB[]{
	\TextB{Supp nonempty $A\subseteq\Rbb.$}
}
\Anchor{0CT1}\ProblemBX{\TipsN{1}}{
	\TextA{Define $-A=\Bra{{-a}:a\in A}\Rightarrow{-\Par{{-A}}}=A.$ \,Prove $\Sup\Par{{-A}}=-\inf A.$}
}$-b$ is an up-bound of $-A\Longleftrightarrow\forall a\in A,\,-a\leqslant-b\Longleftrightarrow a\geqslant b\Longleftrightarrow b$ is a low-bound of $A.$\parSol{}
Thus \,$-b_M=\Sup\Par{{-A}}\Longleftrightarrow-b_M\leqslant -b\Longleftrightarrow b_M\geqslant b\Longleftrightarrow b_M=\inf A.$\PfEnd
\SepLine

\Anchor{0CT2}\ProblemBX{\TipsN{2}}{
	\TextA{Show if $x\in\Rbb,$ {\tgnr\large(a)} $\sup A>x\Rightarrow\exists\,a\in A,\:a>x,$ \;{\tgnr\large(b)} $\inf A<x\Rightarrow\exists\,a\in A,\:a<x.$}
}(a) $\nexists\:a>x\Longleftrightarrow\forall a\in A,\,a\leqslant x\Longrightarrow\sup A\leqslant x.$\parSol{\Ha}
\Or By (b), $\inf\Par{{-A}}=-\sup A<-x\Rightarrow\exists\,{-a}\in A,\:-a<-x.$ \hfill Simlr for (b).\Blind{\qquad}\PfEnd
%\vspace{4pt}\parSol{}
%(b) $\nexists\,a<x\Longleftrightarrow\forall a\in A,\,a\geqslant x.$ Then by def of inf.\parSol{\Hb}
%\Or By (a), $\sup\Par{{-A}}=-\inf A>-x\Rightarrow\exists\,{-a}\in A,\:-a>-x.$\PfEnd
\SepLine
\pagebreak

\Anchor{0C6}\ProblemN{6}{
	\TextA{Supp nonempty $A,B\subseteq\Rbb.$ Prove $\Sup\Par{A+B}=\sup A+\sup B,$\FontNorm\;and\;$\Inf\Par{A+B}=\inf A+\inf B.$}
}$\inf A+\inf B\leqslant a+b\leqslant\sup A+\sup B\Rightarrow\Sup\Par{A+B}\leqslant\sup A+\sup B,\,\inf A+\inf B\leqslant\Inf\Par{A+B}.$\vspace{2pt}\parSol{}
$\sup A+\sup B>\Sup\Par{A+B}\Longleftrightarrow\sup A>\Sup\Par{A+B}-\sup B$\parSol{}
$\Longleftrightarrow\exists\,a+\sup B>\Sup\Par{A+B}\Longleftrightarrow\sup B>\Sup\Par{A+B}-a\Longleftrightarrow\exists\,a+b>\Sup\Par{A+B}.$ Ctradic.\vspace{2pt}\parSol{}
%$\inf A<\Inf\Par{A+B}-\inf B\Longleftrightarrow\inf A+\inf B<\Inf\Par{A+B}$\parSol{}
%$\Longleftrightarrow\exists\,a+\inf B<\Inf\Par{A+B}\Longleftrightarrow\inf B<\Inf\Par{A+B}-a\Longleftrightarrow\exists\,b+a<\Inf\Par{A+B}.$ Ctradic.
Simlr for $\Inf\Par{A+B}\in A+B.$ \,\Or Apply to $-A-B,$ becs $\Sup\Par{{-A}}=-\inf A.$\PfEnd
%Then $\forall a\in A,b\in B,\,a+b\in A+B\Rightarrow\exists\,c+d\in A+B,\,c+d>a+b.$\parSol{}
%If $\sup A\in A.$ Then $\sup A+d>a+b.$
%Supp $A,B$ both have infily many elem. Supp sups and infs are not in there sets.\parSol{}
%$\forall a\in A,b\in B,\,$\parSol{}
\SepLine

%\Anchor{0C7}\ProblemN{7}{
%	\TextA{Supp $X\neq\emptySet,\,f,g:X\rightarrow\Rbb$ are functions. \,Prove $\sup_X\!\Par{\,f+g}\leqslant\sup_X f+\sup_X g.$}
%}$ $
%\SepLine

%\Anchor{0C9}\ProblemN{9}{
%	\TextA{Prove the ordered field of rational functions with coeffs in $\Rbb$ does not satisfy Archimedean Property.}
%}
%\SepLine

%\Anchor{0C14}\ProblemN{14}{
%	\TextA{Supp $A,B\subseteq\Interval{[}{]}{{-\infty,\infty}}$ are intervals, and $A\cap B\neq\emptySet.$ \,Prove $A\cup B$ are intervals.}
%}
%\SepLine

%\Anchor{0C15}\ProblemN{15}{
%	\TextA{Supp $I\subseteq\Rbb$ is an interval containing more than one number. Prove $\Inf\Par{I\cap\Qbb}=\inf I.$}
%}$\inf I<\Inf\Par{I\cap\Qbb}\Rightarrow\exists\,x\in I,\,x<\Inf\Par{I\cap\Qbb}.$ While $\exists\,n\in\Nbp,\,1\big/n<x.$\PfEnd
%\SepLine

\Anchor{0C16}\ProblemN{16}{
	\TextA{Supp $\Rbb{_1},\Rbb{_2}$ are complete ordered fields. Define $\varphi_1,\varphi_2$ as in [0.11].\vspace{2pt}}
	\TextA{Define $\psi:\Rbb{_1}\rightarrow\Rbb{_2}$ by $\psi\Par{a}=\Sup\Bra{\varphi_2\Par{q}:q\in\Qbb,\,\varphi_1\Par{q}\leqslant a}.$ \,Show\vspace{2pt}}
	\PrePa\TextA{$\psi:\Rbb{_1}\rightarrow\Rbb{_2}$ is one-to-one. \;{\tgnr\large(b)} $\psi\Par{0}=0,\,\psi\Par{1}=1.$\vspace{2pt}}
	\PrePc\TextA{$\psi\Par{a+b}=\psi\Par{a}+\psi\Par{b}.$ \;{\tgnr\large(d)} $\psi\Par{ab}=\psi\Par{a}\:\!\psi\Par{b}.$\vspace{2pt}}
}(a) Define $\mR_1\Par{a}=\Bra{q\in\Qbb:\varphi_1\Par{q}\leqslant a}\Rightarrow\sup_{\!\mR_1\SmallPar[0.5pt]{a}}\varphi_1=a.$ \;Let $\psi_1\Par{a}=\sup_{\!\mR_1\SmallPar[0.5pt]{a}}\varphi_2=\psi\Par{a}.$\vspace{2pt}\parSol{\Ha}
Define $\mR_2\Par{c}=\Bra{q\in\Qbb:\varphi_2\Par{q}\leqslant c}\Rightarrow\sup_{\!\mR_2\SmallPar[0.5pt]{c}}\varphi_2=c.$\vspace{1pt}\parSol{\Ha}
Define $\psi_2\Par{c}=\sup_{\!\mR_2\SmallPar[0.5pt]{c}}\varphi_1.$ Then $\psi_2:\Rbb{_2}\rightarrow\Rbb{_1}$ well-defined.\vspace{1pt}\parSol{\Ha}
Note that $\mR_2\BigPar{\psi_1\Par{a}}=\Bra{q\in\Qbb:\varphi_2\Par{q}\leqslant\sup_{\!\mR_1\SmallPar[0.5pt]{a}}\varphi_2}=\mR_1\Par{a}.$\vspace{1pt}\parSol{\Ha}
Now $\psi_2\BigPar{\psi_1\Par{a}}=\sup_{\!\mR_1\SmallPar[0.5pt]{a}}\varphi_1=a.$ \;Rev the roles of $\Rbb_1,\Rbb_2.$\vspace{3pt}\parSol{}
(b) Becs $\varphi_1\Par{q}\leqslant\varphi_1\Par{0}\Longleftrightarrow q\leqslant0\Longleftrightarrow\varphi_2\Par{q}\leqslant\varphi_2\Par{0}.$ \,Simlr for $\psi\Par{1}=1.$\vspace{3pt}\parSol{}
(c) $S=\Bra{\varphi_2\Par{t}:t\in\Qbb,\,\varphi_1\Par{t}\leqslant a+b}\supseteq\Bra{\varphi_2\Par{p+q}:p,q\in\Qbb,\,\varphi_1\Par{p}\leqslant a,\,\varphi_1\Par{q}\leqslant b}=T$\parSol{\Hc}
$\Rightarrow\sup S\geqslant\sup T.$ \,Asum it is `$>$'. Now $\exists\,t\in\Qbb$ suth\;$\sup S\geqslant\varphi_2\Par{t}>\sup T=\psi\Par{a}+\psi\Par{b}.$\parSol{\Hc}
Which means\;$\varphi_2\Par{t}>\varphi_2\Par{p+q},\,\forall p,q\in\Qbb$ suth\;$\varphi_1\Par{p}\leqslant a$ and $\varphi_1\Par{q}\leqslant b.$\parSol{\Hc}
Now $a+b\geqslant\varphi_1\Par{t}>\varphi_1\Sbra{\Par{t+p+q}\big/2}>\varphi_1\Par{p+q},\forall p,q.$ \,Ctradic.\vspace{4pt}\parSol{}
(d) We show it for (I) $a,b>0,$ (II) $a>0>b.$\parSol{\Hd}
$LHS=\Sup\Bra{\varphi_2\Par{t}:t\in\Qbb,\,\varphi_1\Par{t}\leqslant ab},\,\Sup\Bra{\psi\Par{a}\:\!\varphi_2\Par{q}:q\in\Qbb,\,\varphi_1\Par{q}\leqslant b}=RHS.$\parSol{\Hd}
(I) $RHS=\Sup\Bra{\varphi_2\Par{p}\:\!\varphi_2\Par{q}:p,q\in\Qbb,\,0<\varphi_1\Par{p}\leqslant a,\;\varphi_1\Par{q}\leqslant b\Longrightarrow\varphi_1\Par{pq}\leqslant ab}.$\parSol{\Hd}
(II) $RHS=\Sup\Bra{\varphi_2\Par{s}\:\!\varphi_2\Par{q}:s,q\in\Qbb,\,\varphi_1\Par{q}\leqslant b,\,\text{and\;}s\geqslant p,\forall p\text{\;suth\;}0<\varphi_1\Par{p}\leqslant a}.$\parSol{\Hd\HII}
Note that\;$\varphi_1\Par{s}\geqslant\varphi_1\Par{p},\forall p\Rightarrow\varphi_1\Par{s}\geqslant a,$ for if not, $\exists\,p'\in\Qbb$ suth\;$\varphi_1\Par{s}<\varphi_1\Par{p'}\leqslant a.$\parSol{\Hd\HII}
So that $\varphi_1\Par{sq}\leqslant a\:\!\varphi_1\Par{q}\leqslant ab.$\vspace{2pt}\parSol{\Hd}
Now $LHS\geqslant RHS.$ Asum it is `$>$'. Then $\exists\,t\in\Qbb$ suth\;$LHS\geqslant\varphi_2\Par{t}>RHS.$\parSol{\Hd}
Which means\;$\varphi_2\Par{t}>\psi\Par{a}\:\!\varphi_2\Par{q},\,\forall q\in\Qbb$ suth\;$\varphi_1\Par{q}\leqslant b.$\parSol{\Hd}
(I) $\Longrightarrow\varphi_2\Par{t}>\varphi_2\Par{s}\:\!\varphi_2\Par{q},\,\forall q$ and $\forall s\in\Qbb$ suth\;$0<\varphi_1\Par{s}\leqslant a.$\parSol{\Hd}
(II) $\Longrightarrow\varphi_2\Par{t}>\varphi_2\Par{s}\:\!\varphi_2\Par{q},\,\forall q$ and $\forall s\in\Qbb$ suth $\varphi_2\Par{s}\geqslant\varphi_2\Par{p},\forall p$ suth\;$0<\varphi_1\Par{p}\leqslant a.$\vspace{1pt}\parSol{\Hd}
Thus\;$ab\geqslant\varphi_1\Par{t}>\varphi_1\Sbra{\Par{t+sq}\big/2}>\varphi_1\Par{sq},\forall s,q.$ \,Ctradic.\PfEnd\SepLine
\ChEnd

\vfill\ChDecl{Ch0D}{0$\cdot$D}{}

\vspace{4pt}

\Anchor{0DN065}\ProblemB{
	\TextA{Supp nonempty $A$ is closed and open subsets of $\Rbbn.$ Prove $A=\Rbbn.$}
}Asum $A\neq\Rbbn.$ \,Let\;$a\in A,\,b\in\Rbbn\Backslash A.$ \,Define \,$f\Par{t}=\Par{1-t}\:\!a+t\:\!b.$\par\quad
If $\;\!f\Par{t_1}=f\Par{t_2}\Rightarrow\Par{t_1-t_2}\Par{b-a}=0\Rightarrow t_1=t_2.$ \,Inje.\par\quad
Let $\mA=\Bra{t\in\Interval{[}{)}{0,1}:\,f\Par{t}\in A},$ and \,$\sup\mA=t_M\in\Interval{[}{]}{0,1}.$ \,Let $c=\;\!f\Par{t_M}.$\vspace{2pt}\par\quad
(I) If $c\in A\Rightarrow\exists\,\delta>0,\,B\Par{c,\delta}\subseteq A.$ \,Let \:\!$t\neq t_M$ \:\!be suth \;\!$f\Par{t}\in B\Par{c,\delta}\Rightarrow\BigDvert{\Par{t-t_M}\Par{b-a}}{_\infty}<\delta.$\par\quad\HI
Let $\varepsilon=\aMid{t-t_M}>0\Rightarrow f\Par{t_M+\varepsilon}\in B\Par{c,\delta}\subseteq A\Rightarrow t_M\geqslant t_M+\varepsilon,$ ctradic.\vspace{2pt}\par\quad
(II) If $c\in\Rbbn\Backslash A.$ \,Simlr. \,$\forall t'\in\Interval{[}{)}{t_M-\varepsilon,\:t_M},\,t'\notin\mA.$ \,Now $\forall t\in\mA,\,t<t_M\Rightarrow t<t_M-\varepsilon.$\PfEnd
\SepLine\pagebreak

%\Anchor{0D12}\ProblemN{12}{
%	\TextA{Supp $a_1,a_2,\dots\in\Rbbn$ convg of lim $L.$ \,Prove $F=\Bra{L,a_1,a_2,\dots}$ is closed.}
%}We show $\Rbbn\Backslash F$ is open. Asum $\exists\,b\in\Rbbn\Backslash F$ suth $\forall\delta>0,\,B\Par{b,\delta}\not\subseteq\Rbbn\Backslash F\Longleftrightarrow\exists\,x\in B\Par{b,\delta}\cap F.$\parSol{}
%Then $\forall\delta>0,\exists\,x\in F$ suth $\Dvert{b-x}{_\infty}<\delta.$ \,Thus $\exists$ a seq $b_1,b_2,\dots\in F$ with lim $b\notin F.$\parSol{}
%Now $F=F\cup\Bra{b_1,b_2,\dots},$ while $F=\Bra{L,a_1,a_2,\dots}\Rightarrow$ each $b_j=a_k,\exists\,k\in\Nbp.$ \,Ctradic.\PfEnd
%\SepLine

\Anchor{0D15}\ProblemN{15}{
	\TextA{Supp $F$ is a closed subset of $\Rbb.$ Prove $S=\Bra{a^2:a\in F}$ is closed.}
}Supp $S$ not closed $\Rightarrow\exists\,a_1^2,a_2^2,\dots\in S$ convg to $L^2$ suth $\pm L\notin F.$ \,Let $\varepsilon_1>\varepsilon_2>\dots\in\Rbp.$\parSol{}
Becs $\forall\varepsilon_p,\exists\,m\in\Nbp,\forall k\geqslant m,\:\Dvert{a_k^2-L^2}{_\infty}=\aMid{a_k-L}\cdot\aMid{a_k+L}<\varepsilon_p,$\parSol{}
if $\aMid{a_k-L}<\varepsilon_p,$ then let $b_p=a_k,$ \:if $\aMid{a_k+L}<\varepsilon_p,$ then let $c_p=a_k.$\parSol{}
Then we have at least one seq in $F$ with lim $\pm L\notin F\Rightarrow F$ not closed.\PfEnd
\SepLine

\Anchor{0D17}\ProblemN{17}{
	\TextA{Supp $F\subseteq\Rbb,$ and $\forall n\in\Nbp,\,F\cap\Interval{[}{]}{{-n,n}}$ is closed. Prove $F$ is closed.}
}Becs $G=\BigPar{\Rbb\Backslash F}\cup\Interval{(}{)}{{-\infty,-n}}\cup\Interval{(}{)}{{n,\infty}}$ is open for all $n\in\Nbp.$ Let $n>\Sup\Bra{\aMid{a}:a\in\Rbb\Backslash F}.$\parSol{}
By [0.59], $G=\Interval{(}{)}{{-\infty,-n}}\cup\Interval{(}{)}{{n,\infty}}\cup I_1\cup I_2\cup\cdots$ \Par{disj} $\Longrightarrow I_1\cup I_2\cup\cdots=\Rbb\Backslash F.$\PfEnd
\SepLine

\Anchor{0D20}\ProblemN{20}{
	\TextA{Prove $\forall b\in\Rbbn,\,\delta>0,\,B=\Bra{a\in\Rbbn:\Dvert{a-b}\leqslant\delta}$ is closed.\hfill\FontNorm See below [0.46].}
}Asum $\exists\,a_1,a_2,\dots\in B$ convg to $L$ suth $\Dvert{L-b}>\delta.$ \,Then $\forall\varepsilon>0,\exists\,m\in\Nbp,\forall k\geqslant m,$\parSol{}
\uline{$\Dvert{a_k-L}<\varepsilon$}${}\Rightarrow\delta<\Dvert{L-b}\leqslant\Dvert{a_k-L}+\Dvert{a_k-b}<\delta+\varepsilon.$ \,Now $0<\Dvert{L-b}-\delta<\varepsilon.$ \,Ctradic.\PfEnd
\SepLine

\Anchor{0D21}\ProblemN{21}{
	\TextA{Supp $X$ is an open subset of $\Rbb.$ Prove $\exists\,a_k,b_k\in\Rbb$ suth $X=\bigcup_{k=1}^\infty\Interval{[}{]}{a_k,b_k}.$}
}Let $X=\bigcup_{k=1}^\infty\Interval{(}{)}{c_k,d_k},$ where each $d_k<c_{k+1}.$ \,Let $I_k=\Interval{(}{)}{c_k,d_k}.$\parSol{}
Let $a_{1,k},a_{2,k},\dots$ be convg of lim $c_k.$ Simlr, let $b_{1,k},b_{2,k},\dots$ of lim $d_k.$ \,Let $E_{j,k}=\Interval{[}{]}{a_{j,k},b_{j,k}}.$\parSol{}
Now $\bigcup_{k=1}^\infty\bigcup_{j=1}^\infty E_{j,k}=\bigcup_{k=1}^\infty I_k=X.$ \,Rearrange the order of $E_{j,k}$'s.\PfEnd
\SepLine

\Anchor{0D23}\ProblemN{23}{
	\TextA{Supp $F_{\!1},F_{\!2}$ are disj closed subsets of $\Rbb$ suth $U=F_{\!1}\cup F_{\!2}$ is interval. Prove $F_{\!1}$ or $F_{\!2}$ open.}
}If $U$ open, then $F_{\!1}=\emptySet,\,F_{\!2}=\emptySet$ or $\Rbb.$ Now supp $U$ not open $\Rightarrow F_{\!1}$ or $F_{\!2}$ not open.\parSol{}
We show $F_{\!1}$ open $\Longleftrightarrow F_{\!2}$ not open. Note that $F_{\!1}$ open $\Longleftrightarrow F_{\!1}=\emptySet.$ \,Simlr for $F_{\!2}.$\parSol{}
WLOG, asum $F_{\!1},F_{\!2}$ both not open, and $x\in F_{\!1},y\in F_{\!2}$ with $x<y\Rightarrow x,y\in U\supseteq\Interval{[}{]}{x,y}.$\parSol{}
\NOTICE that $T=\Interval{[}{]}{x,y}\cap F_{\!1}$ is a closed subset that has infily many elem. 又 $\sup T<y.$\parSol{}
(I) If $\sup T\notin F_{\!1}.$ Then $\exists$ a convg seq in $T$ with lim $\sup T.$\parSol{}
(II) Othws, $\forall t\in\Interval{(}{]}{\sup T,y},\,t\notin F_{\!1}\Longleftrightarrow t\in F_{\!2}.$ Now $\exists$ a convg seq in $F_{\!2}$ with lim $\sup T\notin F_{\!2}.$\parSol{}
Ctradic the asum $\Rightarrow$ at least one of $F_{\!1},F_{\!2}$ is empty.\PfEnd
\SepLine

%\Anchor{0D24}\ProblemN{24}{
%	\TextA{Supp $G_1,G_2,\dots$ disj open, and $U=\bigcup_{k=1}^\infty G_k$ is interval. Prove at most one $G_k\neq\emptySet.$}
%}Asum all $G_1,G_2,\dots\neq\emptySet\Rightarrow U$ contains infily many elem.\parSol{}
%Becs by [0.59], a union of disj intervals $I_1,I_2,\dots$ is also interval $\Longleftrightarrow$ at most one $I_k\neq\emptySet.$
%Then \,$x\in U\Longleftrightarrow x\in G_1$ xor $x\in G_2$ xor $\cdots,$ which means $\exists\,!\,k,\,x\in G_k.$\parSol{}
%Thus $\exists\,x,\,\forall\delta>0,\,B\Par{x,\delta}\not\subseteq U.$ While $\forall x\in U=\Interval{(}{)}{a,b},\exists\,\delta>0,\,B\Par{x,\delta}\subseteq U.$
%\SepLine

\Anchor{0D25}\ProblemN{25}{
	\TextA{Give an exa of inv \,$f:\Rbb\rightarrow\Rbb\Backslash\Qbb.$}
}Consider a seq disti $a_1,b_1,a_2,b_2,\dots\in\Rbb$ where $\Bra{a_1,a_2,\dots}\in\Qbb$ and each $b_i\in\Rbb\Backslash\Qbb.$\parSol{}
Define \,$\varphi\Par{a_j}=b_{2j-1},\:\varphi\Par{b_k}=b_{2k}\Rightarrow\,\varphi^{-1}\Par{b_i}=\MathLeftBrace{l}{\!a_{\SmallPar{i+1}/2},\,\,\text{if }i\text{ is odd,}\\\!b_{i/2},\hfill\text{if }i\text{ is even.}}$\vspace{6pt}\parSol{}
Let $B=\Bra{b_1,b_2,\dots},\,U=\Qbb\cup B,\,K=\Rbb\Backslash U.$ \,Extend $\varphi\in B^U$ to $\psi\in\;\!\!\Par{K\cup B}{^{K\,\cup\,U}}$ by $\psi\mmid_K=I.$\PfEnd
\SepLine

\Anchor{0D26}\ProblemN{26}{
	\TextA{Supp $E,G\subseteq\Rbbn,$ and $G$ is open. Prove $E+G=\Bra{x+y:x\in E,\:\!y\in G}$ is open.}
}Asum $E+G$ not open $\Longleftrightarrow\Rbbn\Backslash\Par{E+G}$ not closed.\vspace{2pt}\parSol{}
Then $\exists\,a=x+y\in E+G$ suth $\forall\delta>0,\exists\,b\notin E+G$ suth $\Dvert{a-b}{_\infty}<\delta.$\parSol{}
Let \:\!$z=b-x\notin G\Rightarrow\Dvert{y-z}{_\infty}<\delta\Rightarrow z\in B\Par{y,\delta}\subseteq G,\,\exists\,\delta>0.$\PfEnd\vspace{3pt}\parSol{}
\Or $\exists\,a_1,a_2,\dots\notin E+G$ while its lim $L=e+g\in E+G.$ \,又 $\forall x\in E,\,a_k-x\notin G$\parSol{}
$\Rightarrow\lim_{k\rightarrow\infty}\!\Par{a_k-e}=L-e=g\in G.$ \,Thus $\Rbbn\Backslash G$ not closed $\Longleftrightarrow G$ not open.\PfEnd
\SepLine
\ChEnd\pagebreak

\ChDecl{Ch0E}{0$\cdot$E}{}

\vspace{2pt}

\ProblemN{1}{
	\TextA{Prove every convg seq in $\Rbbn$ is bounded.}
}Supp $a_1,a_2,\dots\in\Rbbn$ suth $\forall\varepsilon>0,\exists\,m\in\Nbp,\forall k\geqslant m,\:\aXMid{\Dvert{a_k}{_\infty}-\Dvert{L}{_\infty}}\leqslant\Dvert{a_k-L}{_\infty}<\varepsilon.$\parSol{}
Thus $\Dvert{a_k}{_\infty}\in\Interval{(}{)}{\Dvert{L}{_\infty}\:\!\!-\varepsilon,\:\Dvert{L}{_\infty}\:\!\!+\varepsilon}.$ \,Now asum $\Sup\Bra{\Dvert{a_k}{_\infty}:k\in\Nbp}=\infty.$\vspace{2pt}\parSol{}
Which means $\forall t\in\Rbp,\forall m\in\Nbp,\exists\,k\geqslant m,\,\max\!\Bra{\Dvert{a_1}{_\infty},\dots,\Dvert{a_m}{_\infty}}\leqslant\Dvert{a_k}{_\infty}>t.$ \,Ctradic.\PfEnd
%\vspace{3pt}\parSol{}
%\Or Using coord lim, $a_1,a_2,\cdots\in\Rbbn$ convg $\Longleftrightarrow a_{1,k},a_{2,k},\cdots\in\Rbb$ convg for each $k\in\;\!\!\Bra{1,\dots,n}.$\parSol{}
%Then by (L1), each $a_{1,k},a_{2,k},\cdots$ bounded $\Longleftrightarrow a_1,a_2,\cdots$ bounded.
\SepLine

\BulletPointX\TipsN{1}\,\,\,By def, a seq \:\!$a_1,a_2,\cdots\in\Rbbn$ convg to $L$ $\Longrightarrow$ every subseq convg to $L.$
\SepLine

\ProblemN{3}{
	\TextA{Supp $F\subseteq\Rbbn,$ every seq in $F$ has a convg subseq with lim in $F.$ Prove $F$ is closed bounded.}
}Supp $a_1,a_2,\cdots\in F$ convg with lim $L.$ Becs $\exists$ subseq with lim in $F.$ By \TIPSN{1}, $L\in F.$\parSol{}
Asum $\sup\Bra{\Dvert{a}{_\infty}:a\in F}=\infty\Rightarrow\forall\Delta>0,\exists\,a_1,a_2,\cdots\in F$ with each $\Dvert{a_{k+1}}{_\infty}\geqslant\Dvert{a_k}{_\infty}+\Delta.$\parSol{}
Thus every subseq is unbounded $\Rightarrow$ not convg. Ctradic.\PfEnd
\SepLine

\ProblemB{
	\TextB{Supp $b\in A\subseteq\Rbbm,\,f:A\rightarrow\Rbbn.$ Prove}
	\TextB{[P] \;\!$f$ is continu at $b\Longleftrightarrow\forall b_1,b_2,\cdots\in A$ suth $\lim_{k\rightarrow\infty}b_k=b,$ $\lim_{k\rightarrow\infty}f\Par{b_k}=f\Par{b}.$ [Q]\vspace{1pt}}
}$Q\Rightarrow P:$ \;Supp $\varepsilon>0$ suth $\forall\delta>0,\exists\,a\in A$ with $\Dvert{a-b}{_\infty}<\delta$ and $\BigDvert{\,f\Par{a}-f\Par{b}}{_\infty}\geqslant\varepsilon.$\parSol{}
\Blind{$P\Rightarrow Q:$ \;}Fix a $\delta.$ \,Define $\delta_k=\delta\big/k\Rightarrow\exists\,a_k$ for each. \,Now $\lim_{k\rightarrow\infty}a_k=b.$\parSol{}
\Blind{$P\Rightarrow Q:$ \;}Thus $\forall m,\forall k\geqslant m,\,\BigDvert{\,f\Par{a_k}-f\Par{b}}{_\infty}\geqslant\varepsilon.$ \,Ctradic $Q$.\vspace{4pt}\parSol{}
$P\Rightarrow Q:$ \;Supp $b_1,b_2,\cdots\in A$ suth $\forall\delta>0,\,\exists\,m,\forall k\geqslant m,\,\Dvert{b_k-b}{_\infty}<\delta.$\parSol{}
\Blind{$P\Rightarrow Q:$ \;}Asum $\varepsilon>0$ suth $\forall m,\forall k\geqslant m,\,\BigDvert{\,f\Par{b_k}-f\Par{b}}{_\infty}\geqslant\varepsilon.$ \,Ctradic $P.$\PfEnd 
\SepLine

%\ProblemN{}{
%	\TextA{}
%}
%\SepLine
%
%\ProblemN{}{
%	\TextA{}
%}
%\SepLine

\ProblemN{8}{
	\TextA{Supp $\,f\in\RbbP{\Rbb}$ is bounded continu. Prove $\,f$ uniformly continu.}
}Asum $\exists\,\varepsilon>0,\forall k\in\Nbp,\,\aXMid{\,f\Par{a_k}-f\Par{b_k}}\geqslant\varepsilon$ for some $a_k,b_k\in\Rbb$ suth $\aMid{a_k-b_k}<1\big/k.$\parSol{}
Becs $\,f\Par{a_1},\,f\Par{a_2},\cdots$ bounded $\Rightarrow\exists\,\,f\Par{a_{k_1}},\,f\Par{a_{k_2}},\cdots$ convg to $A.$\parSol{}
Now $\aXMid{A-f\Par{b_{k_j}}}\leqslant\aXMid{A-f\Par{a_{k_j}}}+\aXMid{\,f\Par{a_{k_j}}-f\Par{b_{k_j}}}\Longrightarrow\,f\Par{b_{k_1}},\,f\Par{b_{k_2}},\cdots$ convg to $A.$\PfEnd
\SepLine

\ProblemN{10}{
	\TextA{Supp bounded $A\subseteq\Rbbm$ and uniformly continu \;\!$f\in\RbbP{A}.$ Prove $\,f$ is bounded.}
}Asum $\forall\Delta>0,\exists\,a_1,a_2,\cdots\in A$ suth each $\aXMid{\,f\Par{a_{k+1}}-f\Par{a_k}}\geqslant\aXMid{\,f\Par{a_{k+1}}}-\aXMid{\,f\Par{a_k}}\geqslant\Delta.$\parSol{}
$\exists$ subseq $a_{j_1},a_{j_2},\cdots$ convg to $L\Rightarrow\forall\delta>0,\exists\,j_m,\forall j_x,j_y\in\;\!\!\Bra{\;\!j_m,j_{m+1},\cdots},\:\Dvert{a_{j_x}-a_{j_y}}<\delta.$\parSol{}
$\forall\varepsilon>0,\exists\,\delta>0,\forall a,b\in A$ suth $\Dvert{a-b}<\delta,$ we have $\,f\Par{a}\in\Interval{(}{)}{\,f\Par{b}-\varepsilon,\;f\Par{b}+\varepsilon}.$\parSol{}
Let $b_{k_i}=a_{j_{k_i}}\!.$ \,Becs $\exists\,k_1<k_2<\cdots$ suth $\,f\Par{b_{k_1}},f\Par{b_{k_2}},\cdots$ is monotone.\parSol{}
If incre, then let $k_1$ suth $\,f\Par{b_{k_1}}>0\Rightarrow\,f\Par{b_{k_{i+1}}\!}>f\Par{b_{k_i}\:\!\!}+\Delta.$ Ctradic. Simlr for decre.\PfEnd
\SepLine

%\ProblemN{11}{
%	\TextA{Supp $\,f\in\RbbP{\Rbb}$ is diff and $\;\!f\apostrophe$ is bounded. Prove $\;\!f$ is uniformly continu.}
%}Asum $\exists\,\varepsilon>0,\forall k\in\Nbp,\exists\,a_k,b_k\in\Rbb$ suth $\aMid{a_k-b_k}<1\big/k$ \,while \,$\aXMid{\,f\Par{a_k}-f\Par{b_k}}\geqslant\varepsilon.$\parSol{}
%%$f\apostrophe\Par{a}=\lim_{k\rightarrow\infty}\;\!${\FontSmall\Large$\frac{\,\;f\Par{a\,+\,\Delta_k}\,-\;f\Par{a}\;}{\Delta_k}$}, where $\Delta_1,\Delta_2,\cdots\in\Rbp$ convg to $0.$\parSol{}
%\SepLine

\ProblemN{18}{
	\TextA{Supp $h:\Rbbm\rightarrow\Rbbn.$ Prove $h$ continu $\Longleftrightarrow h^{-1}\Par{G}$ is open for all open $G\subseteq\Rbbn.$}
}Supp $h$ continu and open $G\subseteq\Rbbn.$ We show $\Rbbm\Backslash h^{-1}\Par{G}=\Bra{t\in\Rbbm:h\Par{t}\notin G}$ is closed.\parSol{}
Let $t_1,t_2,\cdots\in\Rbbm\Backslash h^{-1}\Par{G}$ convg to $L\Rightarrow h\Par{t_1},h\Par{t_2},\cdots\in\Rbbn\Backslash G$ convg to $h\Par{L}.$\vspace{2pt}\parSol{}
Supp open $G\subseteq\Rbbn$ suth $\Rbbm\Backslash h^{-1}\Par{G}$ not closed. Asum $h$ continu.\parSol{}
$\exists\,t_1,t_2,\cdots\in\Rbbm\Backslash h^{-1}\Par{G}$ convg to $L\in h^{-1}\Par{G}\Longrightarrow h\Par{t_1},h\Par{t_2},\cdots\in\Rbbn\Backslash G$ convg to $h\Par{L}\notin G.$\PfEnd
\SepLine

%\ProblemN{19}{
%	\TextA{Supp $h:\Rbbm\rightarrow\Rbbn.$ Prove $h$ continu $\Longleftrightarrow h^{-1}\Par{F}$ is closed for all closed $F\subseteq\Rbbn.$}
%}\NOTICE that \,$h$ continu $\Longleftrightarrow\Rbbm\Backslash h^{-1}\Par{F}=h^{-1}\BigPar{\Rbbn\Backslash F}$ is open for all open $\Rbbn\Backslash F\subseteq\Rbbn.$\PfEnd\parSol{}
%%\Or Supp $h$ continu and closed $F\subseteq\Rbbn.$ We show $\Rbbm\Backslash h^{-1}\Par{F}=\Bra{t\in\Rbbm:h\Par{t}\notin F}$ is open.\parSol{}
%%Supp $h\Par{t}\notin F\Rightarrow\exists\,\delta>0,\forall s$ suth $\BigDvert{h\Par{s}-h\Par{t}}{_\infty}<\delta,$ we have $h\Par{s}\notin F.$
%\SepLine

\def\Finf{F_{\:\!\!\!\infty}}
\def\subZero{{_{\:\!0}}}
\ProblemN{22}{
	\TextA{Supp decre seq $F_{\!1}\supsetneq F_{\!2}\supsetneq\cdots$ non-$\emptySet$ closed bounded subsets of $\;\!\Rbbn.$ Prove $\Finf=\bigcap_{k=1}^\infty F_{\!k}\neq\emptySet.$}
}Define $\Dvert{F}{_\infty}=\Sup\Bra{\Dvert{a}{_\infty}:a\in F}$ and $\Dvert{F}\subZero=\Inf\Bra{\Dvert{a}{_\infty}:a\in F}$ for all bounded $F\subseteq\Rbbn.$\parSol{}
Note that $\Dvert{F}{_\infty}=-\infty\Longleftrightarrow\Dvert{F}\subZero=\infty\Longleftrightarrow\Dvert{F}{_\infty}\:\!\!<\Dvert{F}\subZero\Longleftrightarrow F=\emptySet.$\parSol{}
Now $\Dvert{F_{\!1}}{_\infty},\Dvert{F_{\!2}}{_\infty},\cdots$ bounded decre, and $\Dvert{F_{\!1}}\subZero,\Dvert{F_{\!2}}\subZero,\cdots$ bounded incre.\parSol{}
%\Inf\Bra{\Dvert{F_{\!k}}{_\infty}:k\in\Nbp},\Sup\Bra{\Dvert{F_{\!k}}\subZero:k\in\Nbp}
Consider $\Dvert{\Finf}{_\infty}=\lim_{k\rightarrow\infty}\Dvert{F_{\!k}}{_\infty},$ and\;$\Dvert{\Finf}\subZero=\lim_{k\rightarrow\infty}\Dvert{F_{\!k}}\subZero.$\parSol{}
Note that $\Dvert{\Finf}{_\infty}-\Dvert{\Finf}\subZero=\lim_{k\rightarrow\infty}\!{\Sbra{\Dvert{F_{\!k}}{_\infty}-\Dvert{F_{\!k}}\subZero}}\geqslant0.$\parSol{}
\又 For a closed bounded $F,$ $\Dvert{F}{_\infty}=\Dvert{a}$ for some $a\in F,$ \:\!simlr for $\Dvert{F}\subZero.$\PfEnd
\PfEnd
\SepLine

%\ProblemN{23}{
%	\TextA{Supp closed $F\subseteq\Rbb,$ continu $g\in\RbbP{F}.$ Prove $\exists$ continu $h\in\RbbP{\Rbb}$ suth $\forall x\in F,\,g\Par{x}=h\Par{x}.$}
%}
%\SepLine

\ProblemN{26}{
	\TextA{Supp $F\subseteq\Rbbn$ suth every continu $\,f\in\RbbP{F}$ attains a max. Prove $F$ is closed bounded.}
}Let $\,f\Par{a_M}=\Sup\Bra{\,f\Par{a}:a\in F}.$\parSol{}
%$\forall\delta>0,\exists\,m,\forall k\geqslant m,\,\,f\Par{a_k}\in B\BigPar{\,f\Par{a_M},\delta}.$\parSol{}
$\forall k\in\Nbp,\exists\,\delta_k>0,\forall b_k\in F\cap B\Par{a_M,\delta_k},\,\,f\Par{b_k}\in B\BigPar{\,f\Par{a_M},1\big/k}.$\parSol{}
Then $\delta_1,\delta_2,\cdots$ is decre and of lim $0.$ Thus $b_1,b_2,\cdots$ convg to $a_M.$\parSol{}
Asum $\exists\,a_1,a_2,\cdots\in F$ has no subseq convg to some elem in $F.$\parSol{}

\SepLine

\ProblemN{27}{
	\TextA{Supp $\,f\in\RbbP{\Rbb}$ is incre. Prove $\exists$ countable $A\subseteq\Rbb$ suth $\,f\mmid_{\Rbb\text{\small\backslash}A}$ is continu.}
}Asum $\forall$\:\!countable $A\subseteq\Rbb,$\;$\,f\mmid_{\Rbb\text{\small\backslash}A}$ is not continu at some $b\in\Rbb\Backslash A.$\parSol{}
Then $\exists\,\varepsilon>0,\forall\delta>0,\exists\,a\in\Rbb\Backslash A$ suth $\aMid{a-b}<\delta,$\;$\aXMid{\,f\Par{a}-f\Par{b}}\geqslant\varepsilon.$\parSol{}

\SepLine

\ProblemN{29}{
	\TextA{Supp continu $\,f:\Interval{[}{]}{a,b}\rightarrow\Rbb,$ and $t$ is between $\,f\Par{a},\,f\Par{b}.$ Prove $\exists\,c\in\Interval{[}{]}{a,b},\;f\Par{c}=t.$}
}Let $a_0=a,\,b_0=b.$\parSol{}
{\tgbf Step 1}\;\; Pick $c_1\in\Interval{(}{)}{a_0,b_0}.$ If $\,f\Par{c_1}=t,$ then stop.\parSol{}
\Blind{{\tgbf Step 1}\;\; }If $\,f\Par{a_0}<t<f\Par{b_0}.$ Let $\Par{a_1,b_1}=\MathLeftBrace{l}{\!\!\Par{a_0,c_1},\;\text{if }t<f\Par{c_1},\\\!\!\Par{c_1,b_0},\;\text{if }\,f\Par{c_1}<t.}$\vspace{3pt}\parSol{}
\Blind{{\tgbf Step 1}\;\; }If $\,f\Par{b_0}<t<f\Par{a_0}.$ Let $\Par{a_1,b_1}=\MathLeftBrace{l}{\!\!\Par{c_1,b_0},\;\text{if }t<f\Par{c_1},\\\!\!\Par{a_0,c_1},\;\text{if }\,f\Par{c_1}<t.}$\vspace{6pt}\parSol{}
{\tgbf Step m}\; Pick $c_m\in\Interval{(}{)}{a_{m-1},b_{m-1}}.$ If $\,f\Par{c_m}=t,$ then stop.\parSol{}
\Blind{{\tgbf Step m}\; }If $\,f\Par{a_{m-1}}<t<f\Par{b_{m-1}}.$ Let $\Par{a_m,b_m}=\MathLeftBrace{l}{\!\!\Par{a_{m-1},c_m},\;\text{if }t<f\Par{c_m},\\\!\!\Par{c_m,b_{m-1}},\;\text{if }\,f\Par{c_m}<t.}$\vspace{3pt}\parSol{}
\Blind{{\tgbf Step m}\; }If $\,f\Par{b_{m-1}}<t<f\Par{a_{m-1}}.$ Let $\Par{a_m,b_m}=\MathLeftBrace{l}{\!\!\Par{c_m,b_{m-1}},\;\text{if }t<f\Par{c_m},\\\!\!\Par{a_{m-1},c_m},\;\text{if }\,f\Par{c_m}<t.}$\vspace{8pt}\parSol{}
Either we stop at some $m$ and done or we get a seq $\Interval{(}{)}{a_1,b_1}\supsetneq\Interval{(}{)}{a_2,b_2}\supsetneq\cdots$\parSol{}
suth $\forall\varepsilon>0,\exists\,m\in\Nbp,\forall k\geqslant m,$ $b_k-a_k<\varepsilon\Longrightarrow\lim_{k\rightarrow\infty}a_k=\lim_{k\rightarrow\infty}b_k,$ let it be $c\in\Interval{(}{)}{a_0,b_0}.$\parSol{}
Now $\forall\varepsilon>0,\exists\,m\in\Nbp,\forall k\geqslant m,\:\aXMid{t-f\Par{a_k}}<\varepsilon\Longrightarrow\lim_{k\rightarrow\infty}\,f\Par{a_k}=t=f\Par{c}.$\PfEnd
\SepLine

%\ProblemN{30}{
%	\TextA{Prove for every continu \;\!$f:\Rbb\rightarrow\Rbb\Backslash\Qbb,$\;$\,f\Par{\Rbb}=\Bra{c}$ for some $c\in\Rbb\Backslash\Qbb.$}
%}If $f\Par{a}\neq f\Par{b}.$ By Exe\:\!(29), $\forall t\in\Qbb$ suth $t$ between $\,f\Par{a},f\Par{b},$ $\exists\,c\in\Interval{(}{)}{a,b},\;f\Par{c}=t,$ ctradic.\PfEnd
%%\Or Asum $\,f\Par{\Rbb}$ has more than one elem $\Rightarrow\exists$ convg $a_1,a_2,\cdots\in\Rbb$ and convg $b_1,b_2,\cdots\in\Rbb$\parSol{}
%%suth $\lim_{k\rightarrow\infty}\aMid{a_k-b_k}=0$ while $\lim_{k\rightarrow\infty}a_k<\lim_{k\rightarrow\infty}b_k.$ \colorbox{yellow}{TODO}\PfEnd
%\SepLine

\ProblemN{L1}{
	\TextA{Supp $p\in\RbbP{\Rbb}$ is a poly. Prove $p$ continu.}
}Write $p\Par{x}=a_0+a_1x+\dots+a_mx^m$ with $a_m\neq0.$ \,Supp $x_0\in\Rbb.$ Then write\hspace{1pt}:\parSol{}
$\forall\varepsilon>0,\exists\,\delta_M>0,\forall\delta$ suth $\aMid{\delta}<\delta_M,\:\aXMid{p\Par{x_0+\delta}-p\Par{x_0}}=\aXMid{\delta\cdot q\Par{x_0,\delta}}<\varepsilon\Longleftrightarrow \aXMid{q\Par{x_0,\delta}}<\aMid{\varepsilon\big/\delta}$\vspace{2pt}\parSol{}
where \;\!$q\Par{x_0,\delta}=\sum_{k=1}^m\!\BigBigPar{{\sum_{i=1}^k}\,a_k\:\!\mathC_k^i\,\delta^{k-i-1}\:\!x_0^{i}}\Rightarrow\aXMid{q\Par{x_0,\delta}}\leqslant\sum_{k=1}^m\!{\sum_{i=1}^k}\:\!\aMid{\delta}{^{k-i-1}}\aXMid{a_k\:\!\mathC_k^i\:\!x_0^{i}}.$\vspace{3pt}\parSol{}
Let $\aMid{\delta}<1$ suth \:\!$\max\!\Bra{\aMid{a_k\:\!\mathC_k^i\:\!x_0^{i}}:i,k\in\Nbp,\,1\leqslant k\leqslant m,\,1\leqslant i\leqslant k}\cdot\aMid{\delta}\leqslant\aMid{\varepsilon}<\aMid{\varepsilon\big/\delta}.$\PfEnd
\SepLine

\ProblemN{31}{
	\TextA{Supp $p\in\RbbP{\Rbb}$ is a poly with odd deg. Prove $\exists\,b\in\Rbb$ suth $p\Par{b}=0.$}
}Write $p\Par{x}=a_0+a_1x+\dots+a_mx^m$ with $a_m\neq0.$ Supp $a_m>0.$ \Sbra{If $a_m<0,$ then apply to $-p.$}\vspace{3pt}\parSol{}
Now $p\Par{x}=x^m\XSbra{${\small$\Frac{a_0}{\:x^m}+\Frac{a_1}{\,\,x^{m-1}}+\dots+\Frac{\,a_{m-1}\,}{x}+a_m$}$}$ for all $x\in\Rbb\nonzero.$\vspace{3pt}\parSol{}
While $\aXXMid{${\small$\Frac{a_0}{\:x^m}+\Frac{a_1}{\,\,x^{m-1}}+\dots+\Frac{\,a_{m-1}\,}{x}$}$}\leqslant\delta^m\;\!\aMid{a_0}+\delta^{m-1}\aMid{a_1}+\dots+\delta\;\!\aMid{a_{m-1}}\:\!;$\;let $\delta=\aMid{x}.$\vspace{5pt}\parSol{}
Let $x$ be suth $\delta<1$ and $\;\!\max\!\Bra{\aMid{a_0},\aMid{a_1},\dots,\aMid{a_{m-1}}}\cdot\delta\leqslant\aMid{a_m}\big/m.$ \,Thus $p\Par{{-\aMid{x}}}<0<p\Par{\aMid{x}}.$\PfEnd
%By induc on $\deg p=2k+1,\:\!k\in\Nbp.$ (i) $k=0.$ Immed. (ii) $k\geqslant 0.$ Asum true for $k.$\parSol{}
%Let $p=q+r,$ where $q\Par{x}=a_0+a_1x+\dots+a_{\deg q}x^{2k+1},r\Par{x}=a_{\deg q+1}x^{2k+2}+a_{\deg p}x^{2\SmallPar{k+1}+1}.$\parSol{}
%Let $2k+3=m.$ Supp $q\Par{b}=0.$ If $r\Par{b}=0,$ then done. If $r\Par{b}$
\SepLine

%\ProblemN{32}{
%	\TextA{Prove every convg seq in $\Rbbn$ is a Cauchy seq.}
%}Supp $a_1,a_2,\cdots\in\Rbbn$ convg to $L.$ Then $\forall\varepsilon,\exists\,m\in\Nbp,\forall j,k\geqslant m,\;\Dvert{a_j-L}<\varepsilon,\,\Dvert{a_k-L}<\varepsilon.$\parSol{}
%Note that $\Dvert{a_j-a_k}\leqslant\Dvert{a_j-L}+\Dvert{L-a_k}<2\:\!\varepsilon.$
%\SepLine

\ProblemN{33}{
	\TextA{Supp $a_1,a_2,\cdots\in\Rbbn$ is Cauchy seq and $\exists$ subseq convg to $L.$ Prove the seq convg to $L.$}
}Supp subseq $a_{j_1},a_{j_2},\cdots$ convg to $L.$ Then $\forall\varepsilon>0,\exists\,j_m,\forall j_k\in\;\!\!\Bra{\;\!j_m,j_{m+1},\cdots},\:\Dvert{a_{j_k}-L}<\varepsilon,$\vspace{1pt}\parSol{}
and $\exists\,n\in\Nbp,\forall i\geqslant n,\:\Dvert{a_i-L}=\BigDvert{\Par{a_i-a_{j_k}}+\Par{a_{j_k}-L}}\leqslant\Dvert{a_i-a_{j_k}}+\Dvert{a_{j_k}-L}<2\:\!\varepsilon.$\PfEnd
\SepLine

\ProblemN{34}{
	\TextA{Supp closed $F_{\!1}\subseteq\Rbbn$ and closed bounded $F_{\!2}\subseteq\Rbbn.$ Prove $F_{\!1}+F_{\!2}$ is closed.}
}Asum $\exists$ convg $a_1,a_2,\cdots\in F_{\!1}+F_{\!2}$ with lim $L\notin F_{\!1}+F_{\!2}.$ \,Let each $a_k=x_k+y_k,$ and $z_k=L-x_k.$\parSol{}
$\forall\varepsilon>0,\exists\,m\in\Nbp,\forall k\geqslant m,\:\Dvert{a_k-L}=\Dvert{x_k+y_k-L}=\Dvert{y_k-z_k}<\varepsilon.$\parSol{}
Supp subseq \;\!$y_{j_1},y_{j_2},\cdots$ convg to $L_y\in F_{\!2}.$ Then $\forall\varepsilon>0,\exists\,j_m,\forall j_k\in\;\!\!\Bra{\;\!j_m,j_{m+1},\cdots},$\parSol{}
$\Dvert{z_{j_k}-L_y}=\Dvert{\Par{z_{j_k}-y_{j_k}}+\Par{y_{j_k}-L_y}}<\varepsilon.$ \,Now $x_{j_1},x_{j_2},\cdots\in F_{\!1}$ convg to $L-L_y\notin F_{\!1}.$\PfEnd
\SepLine

\ChEnd
\pagebreak

%%%%%%%%%%%%%%%%%%%%%%%%%%%%%%%%%%%%%%%
% 累计用时:思考学习的时长+整理代码的时长(h)
%%%%%%%%%%%%%%%%%%%%%%%%%%%%%%%%%%%%%%%
% 0.A:  1.25  +    0h
% 0.B:  18.25 + 5.25h
% 0.C:  7     +    2h
% 0.D:  7.25  + 1.75h
% 0.E:  6     +  1.5h
%%%%%%%%%%%%%%%%%%%%%%%%%%%%%%%%%%%%%%%