% Copyright (C) 2024 Songbingzhi628. This work is licensed under Creative Commons Attribution-NonCommercial-ShareAlike 4.0 International License.
% Email: 13012057210@163.com

\def\allfiles{}

% Copyright (C) 2024 Songbingzhi628. This work is licensed under Creative Commons Attribution-NonCommercial-ShareAlike 4.0 International License.
% Email: bingzhi2025@163.com

\documentclass[a4paper, 11pt, UTF8]{article}

\usepackage{amssymb,amsmath,amsfonts}
\usepackage{ctex}
\usepackage{setspace}
\usepackage{graphicx}
\usepackage[breaklinks]{hyperref}
\usepackage{bookmark}
\usepackage{fontspec}
\usepackage{unicode-math}
\usepackage{ulem}
\usepackage{xcolor,xargs}
\usepackage{tabularx}
\usepackage{tikz-cd}
\usepackage{Headers/extarrows}

\usepackage[left=1cm,right=1cm,top=1cm,bottom=0cm]{geometry}

\hypersetup{
	bookmarks=true,
	bookmarksnumbered=false,
	bookmarksopen=false,
	bookmarksdepth=2,
	linktoc=all,
	colorlinks=true,
	linkcolor=blue,
	urlcolor=cyan
}

\bookmarksetup{
	numbered=false,
	open=true
}

\parindent 0pt


%{ 关于字体设置
\setmainfont{TeXGyrePagella-Regular}
\setmathfont{TeXGyrePagellaMath-Regular}
\setmathfont[range={\mathcal,\Longleftrightarrow,\Longrightarrow,\Longleftarrow,\Rightarrow,\Leftarrow,\Leftrightarrow,\sqrt,\dagger}]{LatinModernMath-Regular}
%\setmathfont[range={\backslash}]{TeXGyrePagella-Italic}
\setmathfont[range={\mathbb}]{TeXGyreSchola-Regular}
\setmathfont[range={\complement}]{TeXGyrePagella-Bold}
%\setmathfont[range=\mathfrak]{TeXGyrePagella-Regular}
\setmathfont[range={\tilde}]{DejaVuMathTeXGyre.ttf}

\newfontfamily{\tgnr}{TeXGyrePagella-Regular}
\newfontfamily{\tgbf}{TeXGyreSchola-Regular}
\newfontfamily{\tgbfx}{TeXGyrePagella-Bold}
\newfontfamily{\tgbfxx}{FreeSerifBold}
\newfontfamily{\tgsl}{TeXGyrePagella-Italic}
\newfontfamily{\tgsc}{TeXGyrePagella-BoldItalic}
\newfontfamily{\forbra}{MathJax_Size1-Regular.otf}
\newfontfamily{\forbrax}{MathJax_Size2-Regular.otf}
\newfontfamily{\forslash}{MathJax_Size1-Regular.otf}
\newcommand{\XSlash}[1][-1.2pt]{\hspace{#1}\text{\forslash/}}
\newcommand{\Slash}[1][\envFontA]{${#1$/$}$}

\newcommand{\Largenr}[1]{{\Large\tgnr#1}}
\newcommand{\Largesl}[1]{{\Large\tgsl#1}}
\newcommand{\Largebf}[1]{{\Large\tgbf#1}}
\newcommand{\Largebfx}[1]{{\Large\tgbfx#1}}
\newcommand{\Largebfxx}[1]{{\Large\tgbfxx#1}}

\def\envFont{\large}
\def\envFontHuge{\def\envFontA{\huge}\def\envFont{\LARGE}\def\envFontB{\Large}}
\def\envFontLarge{\def\envFontA{\LARGE}\def\envFont{\Large}\def\envFontB{\large}}
\def\envFontDefault{\def\envFontA{\Large}\def\envFont{\large}\def\envFontB{\normalsize}}
\newcommand{\envFontSmall}[1][\small]{\def\envFontA{\Large}\def\envFont{\normalsize}\def\envFontB{#1}}
\newcommand{\TextB}[1]{\envFontLarge\Largesl{#1}\envFontDefault\par\IndentB}
\newcommand{\Text}[1]{\envFontLarge\Largesl{#1}\envFontDefault\par{\,\!}}
\newcommand{\TextA}[1]{\envFontLarge\Largesl{#1}\envFontDefault\par\Blind{\Ptarget\;}}
\def\FontSmall{\normalsize\envFontSmall}
\def\FontNorm{\large\envFontDefault}
\def\FontLarge{\Large\envFontLarge}
%}

%{ 关于排版缩进的便捷指令
%  · {题目内容}\par\IndentB

\def\IndentB{\Blind{\BulletPoint }}

% 用于对齐(a)(b)(c)(d)
%(a) {text} \par\quad\Ha
%... \par\quad\Ha
%{text} \par\quad
\def\Ha{{\Blind{(a) }}}
\def\Hb{{\Blind{(b) }}}
\def\Hc{{\Blind{(c) }}}
\def\Hd{{\Blind{(d) }}}
\def\He{{\Blind{(e) }}}
\def\Hf{{\Blind{(f) }}}

% 用于对齐(i)(ii)(iii)
%(i) {text} \par\quad\Hi
%... \par\quad\Hi
%{text} \par\quad\Endi
%(ii)  {text} \par\quad\Hii
%... \par\quad\Hii
%{text} \par\quad\Endii
%(iii) {text} \par\quad\Hiii
%... \par\quad\Hiii
%{\text}\par\quad
\def\Endi{}
\def\Endii{}
\def\Hi{\Blind{(i) }}
\def\Hii{\Blind{(ii) }}
\def\Hiii{\Blind{(iii) }}

% 用于对齐(I)(II)(III)
\def\EndI{}
\def\EndII{}
\def\EndIII{}
\def\HI{\Blind{(I) }}
\def\HII{\Blind{(II) }}
\def\HIII{\Blind{(III) }}
%}

%{ 关于文本环境
\newcommand{\hMath}[5][-4pt]{#3\hspace{#1}\begin{array}{#2}#5\end{array}\hspace{#1}#4}
\newcommand{\MathLeftBrace}				[2]{\hMath[0pt]{#1}{\left\{}{\right.}{#2}}
\newcommand{\MathRightBrace}			[2]{\hMath[0pt]{#1}{\left.}{\right\}}{#2}}
\newcommand{\MathLeftrightBrace}		[2]{\hMath[0pt]{#1}{\left\{}{\right\}}{#2}}
\newcommand{\MathLeftMid}				[2]{\hMath[0pt]{#1}{\left|}{\right.}{#2}}
\newcommand{\MathRightMid}				[2]{\hMath[0pt]{#1}{\left.}{\right|}{#2}}
\newcommand{\MathLeftrightMid}			[2]{\hMath[0pt]{#1}{\left|}{\right|}{#2}}
\newcommand{\MathLeftrightPare}			[2]{\hMath[0pt]{#1}{\left(}{\right)}{#2}}

\newcommand{\hText}[2][-4pt]{\hMath[#1]{l}{\left.}{\right.}{#2}}
\newcommand{\Par}[2][1pt]{{\text{\forbra\envFontB(\hspace{#1}}}#2{\text{\hspace{#1}\forbra\envFontB)}}}
\newcommandx{\Mid}[3][1=1pt,2=\big]{{\text{#2|}}\hspace{#1}{}#3{}\hspace{#1}{\text{#2|}}}
\newcommand{\cMid}[2][1pt]{\Mid[#1]{#2}}
\newcommand{\aMid}[2][1pt]{\Mid[#1]{#2}}
\newcommand{\aMidsq}[2][1pt]{\Mid[#1]{#2}{\hspace{0.5pt}}{^2}}
\newcommand{\aXMid}[2][1pt]{\Mid[#1][\Big]{#2}}
\newcommand{\aXMidsq}[2][1pt]{\Mid[#1][\Big]{#2}{\hspace{1pt}}{^2}}
\newcommand{\aXXMid}[2][1pt]{\Mid[#1][\bigg]{#2}}
\newcommand{\aXXMidsq}[2][1pt]{\Mid[#1][\bigg]{#2}{\hspace{1pt}}{^2}}
\newcommand{\innerA}[2][]{\aXMid[1pt]{#1\Ang{#2}}}
\newcommand{\innerAsq}[2][]{\aXMidsq[1pt]{#1\Ang{#2}}}
\newcommand{\SmallinnerA}[1]{\aMid[0pt]{\SmallAng{#1}}}
\newcommand{\SmallinnerAsq}[1]{\aMidsq[0pt]{\SmallAng{#1}}}

\newcommand{\SmallAng}[1]{\langle{#1}\rangle}
\newcommand{\Ang}[1]{\big\langle{}#1{}\big\rangle}
\newcommand{\BigAng}[1]{\Big\langle{}#1{}\Big\rangle}
\newcommand{\SmallDvert}[2][0.5pt]{\Vert\hspace{#1}{}#2{}\hspace{#1}\Vert}
\newcommand{\SmallDvertsq}[2][0.5pt]{\Vert\hspace{#1}{}#2{}\hspace{#1}\Vert{\hspace{0.5pt}}{^2}}
\newcommand{\Dvert}[2][1pt]{\big\Vert\hspace{#1}{}#2{}\hspace{#1}\big\Vert}
\newcommand{\Dvertsq}[2][1pt]{\big\Vert\hspace{#1}{}#2{}\hspace{#1}\big\Vert{\hspace{0.5pt}}{^2}}
\newcommand{\BigDvert}[2][1pt]{\Big\Vert\hspace{#1}{}#2{}\hspace{#1}\Big\Vert}
\newcommand{\BigDvertsq}[2][1pt]{\Big\Vert\hspace{#1}{}#2{}\hspace{#1}\Big\Vert{\hspace{0.5pt}}{^2}}

\newcommand{\Bra}[2][\envFont]{{\text{\forbra{#1\{}}}#2{\text{\forbra{#1\}}}}}
\newcommand{\BigBra}[1]{\Bra[\envFontA]{#1}}
\newcommand{\SmallPar}[2][0pt]{{\envFontSmall[\scriptsize]\Par[#1]{#2}}}
\newcommand{\TinyPar}[2][0pt]{(\hspace{#1}{#2}\hspace{#1})}
\newcommand{\BigPar}[2][1pt]{{\text{\envFont\forbra(\hspace{#1}}}#2{\text{\hspace{#1}\envFont\forbra)}}}
\newcommand{\BigBigPar}[2][1pt]{{\text{\envFontA\forbra(\hspace{#1}}}#2{\text{\hspace{#1}\envFontA\forbra)}}}
\newcommand{\XPar}[2][1pt]{{\text{\envFont\forbrax(\hspace{#1}}}#2{\text{\hspace{#1}\envFont\forbrax)}}}
\newcommand{\Sbra}[2][1pt]{{\text{\envFont\forbra[\hspace{#1}}}#2{\text{\envFont\forbra\hspace{#1}]}}}
\newcommand{\BigSbra}[2][1pt]{{\text{\envFontA\forbra[\hspace{#1}}}#2{\text{\envFontA\forbra\hspace{#1}]}}}
\newcommand{\XSbra}[2][0pt]{{\text{\envFont\forbrax[\hspace{#1}}}#2{\text{\envFont\forbrax\hspace{#1}]}}}
\newcommandx{\zeroSubs}[2][1={},2=\Bra]{{#1#2{0}}}
\def\emptySet{\varnothing}

\newcommand{\Interval}[3]{{\text{\envFont\forbra#1\hspace{1pt}}}#3{\text{\envFont\forbra\hspace{1pt}#2}}}
\newcommand{\XInterval}[3]{{\text{\envFont\forbrax#1}}#3{\text{\envFont\forbrax#2}}}

\def\PrePxEnv{\;}

\def\PrePa{{\!\!\PrePxEnv(a)} }
\def\PrePb{{\!\!\PrePxEnv(b)} }
\def\PrePc{{\!\!\PrePxEnv(c)} }
\def\PrePd{{\!\!\PrePxEnv(d)} }
\def\PrePe{{\!\!\PrePxEnv(e)} }

\newcommand{\ProblemX}[3][\Solution]{
	\def\Ptarget{#2}
	{#2} {#3}\envFontDefault\vspace{-16pt}\par
	#1
}
\newcommand{\ProblemBX}[3][\Solution]{
	\def\PrePxEnv{\;}
	\def\Ptarget{\BulletPointX#2}
	{\BulletPointX#2} {#3}\envFontDefault\vspace{-16pt}\par
	#1
}
\newcommand{\ProblemN}[3][\Solution]{
	\def\PrePxEnv{\:}
	\def\Ptarget{\Onumber{#2}}
	{\Onumber{#2}} {#3}\envFontDefault\vspace{-16pt}\par
	#1
}
%\newcommandx{\ProblemNor}[5][1=\Solution,4=\Sbra]{
%	\def\PrePxEnv{\:}
%	\def\Ptarget{{\Onumber{#2}}\;{#4{\OR({\normalsize#3})}}}
%	{\Onumber{#2}} \;{#4{\OR({\normalsize#3})}}{#5}\envFontDefault\vspace{-16pt}\par
%	#1
%}
%\newcommandx{\ProblemNnoor}[5][1=\Solution,4=\Sbra]{
%	\def\PrePxEnv{\:}
%	\def\Ptarget{{\Onumber{#2}}\;{#4{\small#3}}}
%	{\Onumber{#2}} \;{#4{\small#3}}{#5}\envFontDefault\vspace{-16pt}\par
%	#1
%}
\newcommand{\ProblemB}[2][\Solution]{
	\def\PrePxEnv{\;}
	\def\Ptarget{\BulletPoint}
	{\BulletPoint}{#2}\envFontDefault\vspace{-16pt}\par
	#1
}
%\newcommandx{\ProblemBnoor}[4][1=\Solution,3=\Par]{
%	\def\PrePxEnv{\;}
%	\def\Ptarget{{\BulletPoint}#3{{\small#2}}}
%	{\BulletPoint} #3{{\small#2}}{#4}\envFontDefault\vspace{-16pt}\par
%	#1
%}
%\newcommand{\ProblemBor}[3][\Solution]{
%	\def\PrePxEnv{\;}
%	\def\Ptarget{{\BulletPoint} {\OR({\normalsize#2})}}
%	{\BulletPoint} {\OR({\normalsize#2})}{#3}\envFontDefault\vspace{-16pt}\par
%	#1
%}
\newcommand{\AlignEq}[2]{
	\vspace{-25pt}
	\begin{align*}
		#1#2
	\end{align*}
	\vspace{-25pt}
}
%}

%{ 关于常用标识的便捷指令

\newcommand{\Frac}[2]{{\displaystyle{\frac{#1}{#2}}}}
\def\notRightarrow{\Rightarrow\!\!\!\!\!\!\!/\,\,\,\,}%\hspace{-13.5pt}/\hspace{7.5pt}}
\def\notLeftarrow{\Leftarrow\hspace{-13.5pt}/\hspace{7.5pt}}
\def\notLongrightarrow{\Longrightarrow\!\!\!\!\!\!\!\!/\,\,\,\,\,}
\def\notLongleftarrow{\Longleftarrow\!\!\!\!\!\!\!\!/\,\,\,\,\,}
\def\notLongleftrightarrow{\Longleftrightarrow\!\!\!\!\!\!\!\!/\,\,\,\,\,}
\def\Lpf{{($\Leftarrow$)}\,\,}
\def\Rpf{{($\Rightarrow$)}\,\,}
\def\BigLpf{{($\Longleftarrow$)}\,\,}
\def\BigRpf{{($\Longrightarrow$)}\,\,}
\def\HLpf{{\Blind{{\Lpf}}}}
\def\HRpf{{\Blind{{\Rpf}}}}
\def\HBigLpf{{\Blind{{\BigLpf}}}}
\def\HBigRpf{{\Blind{{\BigRpf}}}}

%\def\Nbb{{\mathbb{N}}}
%\def\Zbb{{\mathbb{Z}}}
%\def\Qbb{{\mathbb{Q}}}
%\def\Fbb{{\mathbb{F}}}
\def\Nbb{{\text{\tgbfx N}}}
\def\Zbb{{\text{\tgbfx Z}}}
\def\Qbb{{\text{\tgbfx Q}}}
\def\Fbb{{\text{\tgbfx F}}}
\newcommand{\FbbP}[2][\:]{{\Fbb\!#1}{^{#2}}}
%\def\Rbb{\mathbb{R}}
%\def\Cbb{\mathbb{C}}
\def\Rbb{{\text{\tgbfx R}}}
\def\Cbb{{\text{\tgbfx C}}}
\newcommand{\RbbP}[2][\,]{{\Rbb\!#1}{^{#2}}}
\def\Rbbn{\RbbP{n}}
\def\Rbbm{\RbbP{m}}
\newcommand{\CbbP}[2][\:]{{\Cbb\!#1}{^{#2}}}
%\def\Nbp{{\mathbb{N}^+}}
\def\Nbp{{\text{\tgbfx Z}{^+}}}
\def\Qbp{{\text{\tgbfx Q}{^+}}}
\def\Qbs{{\text{\tgbfx Q}{^*}}}
\def\Rbp{{\text{\tgbfx R}{^+}}}

\newcommand{\Backslash}[1][\big]{#1\backslash}
\def\nonzeroFbb{\Fbb${\envFontA$\backslash$}${\def\envFont{\envFontB}\zeroSubs}}
\newcommandx{\nonzero}[2][1=\big,2=\envFontB]{\Backslash[#1]{\Bra[#2]{0}}}
\def\mmid{\text{\envFontA|}}
\def\d{{\textup{\tgnr d}}}
\def\i{{\textup{\tgnr i}}}

\def\Inf{{\textup{\tgnr inf}\:\!}}
\def\Sup{{\textup{\tgnr sup}}}
\def\Dim{{\textup{\tgnr dim}}} % For dim(...)
\def\Deg{{\textup{\tgnr deg}}} % For deg(...)
\def\Range{{\textup{\tgnr range}}} % For range(...)
\def\Null{{\textup{\tgnr null}}} % For null(...)
%\def\Card{{\textup{\tgnr card}}} % For card(...)
%\def\Spn{{\textup{\tgnr span}}} % For span(...)
%\def\col{{\textup{\tgnr col}}\;\!}
%\def\row{{\textup{\tgnr row}}\:\!}
%\def\Col{{\textup{\tgnr col}}}
%\def\Row{{\textup{\tgnr row}}}
\def\range{{\textup{\tgnr range}}\;\!}
\def\null{{\textup{\tgnr null}}\;\!}
\def\rangep{{\textup{\tgnr range}}\,}
\def\nullp{{\textup{\tgnr null}}\,}
%\def\card{{\textup{\tgnr card}}\;\!}
%\def\rank{{\textup{\tgnr rank}}\;\!}
%\def\Rank{{\textup{\tgnr rank}}}
%\def\spn{{\textup{\tgnr span}}\;\!}
\def\Real{{\textup{\tgnr Re}}\;\!} % For  Re x, Re <...>
\def\Imaginary{{\textup{\tgnr Im}}\;\!} % For  Im x, Im <...>
\def\REAL{\textup{\tgnr Re}} % For  Re(...)
\def\IMAGINARY{\textup{\tgnr Im}} % For Im(...)
%\def\Tr{\textup{\tgnr tr}} % For tr(ST)
%\def\tr{\textup{\tgnr tr}\;\!} % For tr T
%\def\Det{\textup{\tgnr det}\:\!} % For det(...)
%\def\alt{\!\!\textup{\tgnr alt}}
%\def\perm{\textup{\tgnr perm}\;\!} % For perm n
%\def\sign{\textup{\tgnr sign}\;\!} % For sign J
\newcommand{\Span}[2][\Par]{{\textup{\tgnr span}}#1{#2}}
%\newcommand{\Sign}[2][\Par]{{\textup{\tgnr sign}}#1{#2}}
\newcommand{\Lm}[2][\Par]{\mathcal{L}#1{#2}}
%\newcommandx{\LmQxx}[4][2=U,3=X,4=\XSlash]{#1\!\!\:#4{^{#2}_{\!#3}}}
%\def\mM{\mathcal{M}}
%\newcommand{\Mt}[2][\Par]{\mathcal{M}\:\!\!#1{#2}}
%\def\Mneg{\mM^{-1}}
\def\Po{{\mathcal{P}}}
\newcommand{\PoF}[2][\Par]{\Po_{\!\!{#2}}#1{\Fbb}}
\newcommand{\PoR}[2][\Par]{\Po_{\!\!{#2}}#1{\Rbb}}
\newcommand{\PoC}[2][\Par]{\Po_{\!\!{#2}}#1{\Cbb}}
\newcommand{\PoFx}[3][\Par]{\Po_{\!\!{#2}}#1{#3}}
\def\PoFi{\PoF{\,}}
\def\PoRi{\PoR{\,}}
\def\PoCi{\PoC{\,}}
%\newcommand{\dualVn}[1]{V_{\!#1\;\,}\upapostrophe}
%\def\dualVm{V_{\!m\,}\upapostrophe}
\def\mathC{C}
\def\apostrophe{\prime}
\def\upapostrophe{\!\!\!\apostrophe\,}
%\def\rE{\reflectbox{\textit{E}}}
%\def\rmA{\reflectbox{\ensuremath{\mathcal{A}}}}
%\def\rR{\reflectbox{\textit{R}}}
\def\mE{\mathcal{E}}
\def\mA{\mathcal{A}}
\def\mB{\mathcal{B}}
\def\mT{\mathcal{T}}
\def\mS{\mathcal{S}}
\def\mR{\mathcal{R}}
\def\mV{\mathcal{V}}
\def\mC{\mathcal{C}}
\def\mD{\mathcal{D}}
%\def\mK{\mathcal{K}}
\def\mX{\mathcal{X}}
\def\mI{\mathcal{I}}
\def\mQ{\mathcal{Q}}
\def\mG{\mathcal{G}}
%\newcommand{\mEnt}[1]{\mE^{\SmallPar[1pt]{#1}}}
%\newcommand{\mRnt}[1]{\mR^{\SmallPar[1pt]{#1}}}
%\newcommand{\mGnt}[1]{\mG^{\SmallPar[1pt]{#1}}}
%\newcommand{\mQnt}[1]{\mQ^{\SmallPar[1pt]{#1}}}
%\newcommand{\Scom}[2]{\mS_{\!#1}#2}
\def\BulletPoint{{\small\bullet}}
\def\BulletPointX{\BulletPoint \,\hspace{1pt}}
\def\bullpt{{\tiny\bullet}}

\def\又{{\:\!\!又}\,\,}
\def\OR{{\large O{\footnotesize R} }}
\def\Or{{\large O{\footnotesize R.} }}
\def\Solution{{\tgbfx\large S\footnotesize{OLUS:}}\,\,\,}
\def\NOTE{\tgnr\large N{\footnotesize OTE}}
\def\NOTEFOR{{\tgnr\large N{\footnotesize OTE} F{\footnotesize OR}}}
\def\NEWTHEOREM{{\tgnr\large N{\footnotesize EW} T{\footnotesize HEO}}}
\def\NOTICE{{\tgnr\large N{\footnotesize OTICE}\;}}
\def\COMMENT{{\tgnr\large C{\footnotesize OMMENT}\;}}
\def\COROLLARY{{\tgnr\large C{\footnotesize ORO}}}
\def\TIPS{{\tgnr\large T{\footnotesize IPS}}}
\newcommand{\TIPSN}[1]{{\tgnr\large T{\footnotesize IPS {\large(#1)}}}}
\def\Tips{{\tgbfx\large T{\footnotesize IPS}:}}
\newcommand{\TipsN}[1]{{\tgbfx\large T{\footnotesize IPS {\large#1\,}}:}}
\def\IndentTips{{\Blind{\Tips \,\,\,}}}
\newcommand{\IndentTipsN}[1]{{\Blind{\TipsN{#1}\,\,\,}}}
\def\IndentComment{{\Blind{\Comment \,\,\,}}}
\def\IndentCorollary{{\Blind{\Corollary \,\,\,}}}
\def\IndentNote{{\Blind{\Note \,\,\,}}}
\def\IndentSolution{{\Blind{\Solution}}}
\newcommand{\parSol}[1]{\par#1\IndentSolution{}}
\def\parCom{\par\IndentComment{}}
\def\parCor{\par\IndentCorollary{}}
\def\parNot{\par\IndentNote{}}
\def\parExa{\par\Blind{\AExa}{}}
\def\Existns{\Sbra[2pt]{{\tgsl Existns}}}
\def\Uniqnes{\Sbra[2pt]{{\tgsl Uniqnes}}}

\newcommand{\Onumber}[1]{\Largebfxx{#1}\hspace{-2pt}}
\newcommand{\LX}[1]{{\tgbfx{#1.\,}}}
\newcommand{\NoteFor}[1]{\Largebfx{N{\small OTE} F{\small OR} #1:}}
\newcommand{\NoteForSmall}[1]{{\tgbfx\large N{\footnotesize OTE} F{\footnotesize OR} #1:}}
\newcommand{\NewNotation}{{\tgbfx\large N{\footnotesize EW} N{\footnotesize OTA}:}}
\newcommand{\NewTheorem}{{\tgbfx\large N{\footnotesize EW} T{\footnotesize HEO}:}}
\newcommand{\Comment}{{\tgbfx\large C\footnotesize{OMMENT:}}}
\newcommand{\Example}{{\tgbfx\large E\footnotesize{XA:}}}
\newcommand{\Exercise}[1]{{\tgbfx\large E{\footnotesize XE} {#1}\hspace{2pt}:}}
\newcommand{\Corollary}{{\tgbfx\large C{\footnotesize ORO}:}}
\newcommand{\Note}{{\tgbfx\large N{\footnotesize OTE}:}}
\newcommand{\AComm}{\Comment \,\,\,}
\newcommand{\ACoro}{\Corollary \,\,\,}
\newcommand{\AExa}{\Example \,\,\,}
\newcommand{\ANote}{\Note \,\,\,}


\def\ChEnd{\rightline{\Largebfx{E{\small NDED}}}\par\vspace{6pt}}
\newcommand{\PfEnd}[1][-18pt]{{\large\vspace{#1}\par\hfill$\square$\par}}
\newcommand{\PfEndB}[1][-18pt]{{\large\vspace{#1}\par\hfill$\blacksquare$\par}}
\newcommandx{\SepLine}[2][1=0pt,2={}]{{\vspace{-5pt}\par
		#2\tiny \_\,\_\,\_\,\_\,\_\,\_\,\_\,\_\,\_\,\_\,\_\,\_\,\_\,\_\,\_\,\_\,\_\,\_\,\_\,\_\,\_\,\_\,\_\,\_\,\_\,\_\,\_\,\_\,\_\,\_\,\_\,\_\,\_\,\_\,\_\,\_\,\_\,\_\,\_\,\_\,\_\,\_\,\_\,\_\,\_\,\_\,\_\,\_\,\_\,\_\,\_\,\_\,\_\,\_\,\_\,\_\,\_\,\_\,\_\,\_\,\_\,\_\,\_\,\_\,\_\,\_\,\_\,\_\,\_\,\_\,\_\_\,\_\,\_\,\_\,\_\,\_\,\_\,\_\,\_\,\_\,\_\,\_\,\_\,\_\,\_\,\_\,\_\,\_\,\_\,\_\,\_\,\_\,\_\,\_\,\_\,\_\,\_\,\_\,\_\,\_\,\_\,\_\,\_\,\_\,\_\,\_\,\_\,\_\,\_\,\_\,\_\,\_\,\_\,\_\,\_\,\_\,\_\,\_\,\_\,\_\,\_\,\_\,\_\,\_\,\_\,\_\,\_\,\_\,\_\,\_\,\_\,\_\,\_\,\_\,\_\,\_\,\_\,\_\,\_\,\_\,\_}\vspace{5pt}\vspace{#1}\par}
\newcommand{\ChDecl}[3]{{\huge\tgbfxx\hypertarget{#1}{#2}}{#3}{\vspace{5pt}}}
\newcommand\hLk[2]{\hyperlink{#1}{#2}}
\newcommand\Lch[2]{\hLk{Ch#1}{#2}}
\newcommand\TXT[1]{\textup{#1}}
\newcommand{\Blind}[1]{\textcolor{gray!2!yellow!1}{#1}}
%}



\begin{document}

\pagecolor{gray!2!yellow!1}

\footnotesize
$\MathRightMid{r}{$
{\tgbf Copyright \textcopyright\ 2024 Songbingzhi628}$\\$
{\tgbf Email: 13012057210@163.com}$\\[-10pt]\\$
\includegraphics[width=120pt]{./by-nc-sa.png}$\\}$\scriptsize
$\hText{$
This work is licensed under the terms of the CC BY-NC-SA 4.0 International License$\\$ (\url{https://creativecommons.org/licenses/by-nc-sa/4.0}). This license requires that$\\$
reusers give credit to the creator. It allows reusers to distribute, remix, adapt, and$\\$
build upon the material in any medium or format, for noncommercial purposes only.$\\$
If others modify or adapt the material, they must license the modified material under$\\$
identical terms. {\tgsl All images except for `by-nc-sa.png' in this manual are licensed under CC0.}$}$\par\vspace{10pt}
\SepLine[4pt]

{\footnotesize 这是我个人挑战「 {\tgsc Measure, Integration {\&} Real Analysis, by Sheldon Axler} 」的学习笔记,包括课文补注和部分习题。
我先从{\tgsc Supplement}即第0章开始。我当时并没有学过数学分析,以为学完Axler的这个Supplement就能具备所有必要的知识基础。\par\vspace{2pt}
0.B节本来不太要命,但超出课文的补助却让我折戟沉沙——的确,它们不需要硬性知识门槛,可以用集合和数理逻辑来推导$\mathcal{D}$的一切。或许是缺乏Dedekind cut的系统学习,我推导这一切时感到我在亲手缔造一个数学分支;我不是自傲的意思,只是说,\textbf{这非常艰难}。
但我还是坚持下来了;在此过程中我肉眼可见我在数理逻辑上的提升。
}

\pagebreak

\footnotesize
\def\formGap{$\\\;\\$}
\centerline{{\Large A{\small BBREVIATION} T{\small ABLE}}}\vspace{14pt}\par
$\hMath{c}{\left.}{\right.}{$
	{\tgbf\normalsize A B}$\\$
	\begin{tabularx}{0.25\textwidth}{
			| r |
			| >{\raggedright\arraybackslash}X | }
		\hline
		abs&			absolute						\\
		add&			addi(tion)(tive)				\\
		adj&			adjoint							\\
		algo&			algorithm						\\
		arb&			arbitrary						\\
		assoc&			associa(tive)(tivity)			\\
		asum&			assum(e)(ption)					\\
		\hline
		%\end{tabularx}\formGap
		%	{\tgbf\normalsize B}$\\$
		%\begin{tabularx}{0.2\textwidth}{
		%		| r |
		%		| >{\raggedright\arraybackslash}X | }
	%	\hline
	becs&			because							\\
%	bss&			basis							\\
%	bses&			bases							\\
%	$B_V$&			basis of $V$					\\
	\hline
\end{tabularx}\formGap
{\tgbf\normalsize E}$\\$
\begin{tabularx}{0.28\textwidth}{
		| r |
		| >{\raggedright\arraybackslash}X | }
	\hline
	-ec&			-ec(t)(tor)(tion)(tive)		\\
%	eig-&			eigen-						\\
	elem&			element(s)					\\
	ent&			entr(y)(ies)				\\
	equa&			equality					\\
	equiv&			equivalen(t)(ce)			\\
	exa&			example						\\
	exe&			exercise					\\
	exis&			exist(s)(ing)				\\
	existns&		existence					\\
	expo&			exponent					\\
	expr&			expression					\\
	\hline
\end{tabularx}\formGap
{\tgbf\normalsize L}$\\$
\begin{tabularx}{0.2\textwidth}{
		| r |
		| >{\raggedright\arraybackslash}X | }
	\hline
	liney&			linear(ly)				\\
	linity&			linearity				\\
	len&			length					\\
	low-&			lower-					\\
	\hline
\end{tabularx}\formGap
{\tgbf\normalsize R}$\\$
\begin{tabularx}{0.3\textwidth}{
		| r |
		| >{\raggedright\arraybackslash}X | }
	\hline
	recurly&		recursively				\\
	repeti&			repetition(s)			\\
	repres&			represent(s)(ation(s))	\\
	req&			require(s)(d)/requiring	\\
	respectly&		respectively			\\
	restr&			restrict(ion)(ive)(ing)	\\
	rev&			revers(e(s))(ed)(ing)	\\
%	rotat&			rotation				\\
	\hline
\end{tabularx}\formGap
$}\hMath{c}{\left.}{\right.}{$
{\tgbf\normalsize C}$\\$
\begin{tabularx}{0.3\textwidth}{
		| r |
		| >{\raggedright\arraybackslash}X | }
	\hline
%	ch&				characteristic					\\
	closd&			closed under					\\
	coeff&			coefficient						\\
%	col&			column							\\
	combina&		combination						\\
	commu&			commut(es)(ing)(ativity)		\\
	cond&			condition						\\
%	conjug&			conjugat(e)(ing)(ion)			\\
	corres&			correspond(s)(ing)				\\
	conveni&		convenience						\\
	convly&			conversely						\\
	count-&			counter-						\\
	ctradic&		contradict(s)(ion)				\\
	ctrapos&		constrapositive					\\
	\hline
\end{tabularx}\formGap
{\tgbf\normalsize F G H}$\\$
\begin{tabularx}{0.26\textwidth}{
		| r |
		| >{\raggedright\arraybackslash}X | }
	\hline
	factoriz&		factorizaion				\\
	fini&			finite						\\
	finide&			finite-dimensional			\\
	\hline
	%\end{tabularx}\formGap
	%	{\tgbf\normalsize G}$\\$
	%\begin{tabularx}{0.22\textwidth}{
	%		| r |
	%		| >{\raggedright\arraybackslash}X | }
%	\hline
%g-eig-&				generalized eig-		\\
%G disk&				Gershgorin disk			\\
\hline
%\end{tabularx}\formGap
%	{\tgbf\normalsize H}$\\$
%\begin{tabularx}{0.2\textwidth}{
%		| r |
%		| >{\raggedright\arraybackslash}X | }
%	\hline
homo&			homogeneity					\\
hypo&			hypothesis					\\
\hline
\end{tabularx}\formGap
%	{\tgbf\normalsize J}$\\$
%\begin{tabularx}{0.2\textwidth}{
%		| r |
%		| >{\raggedright\arraybackslash}X | }
%	\hline
%\end{tabularx}\formGap
%{\tgbf\normalsize K}$\\$
%\begin{tabularx}{0.2\textwidth}{
%		| r |
%		| >{\raggedright\arraybackslash}X | }
%	\hline
%\end{tabularx}\formGap
{\tgbf\normalsize M N}$\\$
\begin{tabularx}{0.30\textwidth}{
| r |
| >{\raggedright\arraybackslash}X | }
\hline
max&			maxi(mal(ity))(mum)		\\
min&			mini(mal(ity))(mum)		\\
multi&			multipl(e)(icati-on/ve)	\\
%multy&			multiplicity			\\
\hline
%\end{tabularx}\formGap
%	{\tgbf\normalsize N}$\\$
%\begin{tabularx}{0.2\textwidth}{
%		| r |
%		| >{\raggedright\arraybackslash}X | }
%	\hline
%nilp&			nilpotent				\\
non0&			nonzero					\\
nonC&			nonconst				\\
notat&			notation(al)			\\
\hline
\end{tabularx}\formGap
{\tgbf\normalsize S}$\\$
\begin{tabularx}{0.24\textwidth}{
| r |
| >{\raggedright\arraybackslash}X | }
\hline
seq&			sequence				\\
simlr&			similar(ly)				\\
%singval&		singular value			\\
solus&			solution				\\
sp&				space					\\
stmt&			statement				\\
std&			standard				\\
supp&			suppose					\\
surj&			surjectiv(e)(ity)		\\
suth&			such that				\\
%symm&			symmetry				\\
\hline
\end{tabularx}\formGap
$}\hMath{c}{\left.}{\right.}{$
{\tgbf\normalsize D}$\\$
\begin{tabularx}{0.31\textwidth}{
| r |
| >{\raggedright\arraybackslash}X | }
\hline
Ddkd&			Dedekind						\\
def&			definition						\\
deg&			degree							\\
%dep&			dependen(t)(ce)					\\
deri&			derivative(s)					\\
%diag&			diagonal(iza-ble/ility/tion)	\\
diff&			differentia(l)(ting)(tion)		\\
dim&			dimension(al)					\\
disti&			distinct						\\
distr&			distributive propert(ies)(ty)	\\
div&			div(ide)(ision)					\\
\hline
\end{tabularx}\formGap
{\tgbf\normalsize I}$\\$
\begin{tabularx}{0.31\textwidth}{
| r |
| >{\raggedright\arraybackslash}X | }
\hline
id&				identity					\\
immed&			immediately					\\
induc&			induct(ion)(ive)			\\
infily&			infinitely					\\
inje&			injectiv(e)(ity)			\\
inv&			inver(se)(tib-le/ility)		\\
%invar&			invariant					\\
%invard&			invariant under				\\
%invarsp&		invariant subspace			\\
%invarspd&		invariant subspace under	\\
iso&			isomorph(ism)(ic)			\\
%isomet&			isometry					\\
\hline
\end{tabularx}\formGap
{\tgbf\normalsize O P Q}$\\$
\begin{tabularx}{0.2\textwidth}{
| r |
| >{\raggedright\arraybackslash}X | }
\hline
%optor&			operator				\\
othws&			otherwise				\\
orthog&			orthogonal				\\
orthon&			orthonormal				\\
\hline
%\end{tabularx}\formGap
%	{\tgbf\normalsize P}$\\$
%\begin{tabularx}{0.2\textwidth}{
%		| r |
%		| >{\raggedright\arraybackslash}X | }
%	\hline
poly&			polynomial				\\
posi&			positive				\\
prod&			product					\\
\hline
%\end{tabularx}\formGap
%	{\tgbf\normalsize Q}$\\$
%\begin{tabularx}{0.2\textwidth}{
%		| r |
%		| >{\raggedright\arraybackslash}X | }
%	\hline
quad&			quadratic				\\
quotient&		quot					\\
\hline
\end{tabularx}\formGap
{\tgbf\normalsize T U V W X Y Z}$\\$
\begin{tabularx}{0.22\textwidth}{
| r |
| >{\raggedright\arraybackslash}X | }
\hline
%trig&			triangular				\\
%trslate&		translate				\\
%trspose&		transpose				\\
\hline
%\end{tabularx}\formGap
%	{\tgbf\normalsize U}$\\$
%\begin{tabularx}{0.2\textwidth}{
%		| r |
%		| >{\raggedright\arraybackslash}X | }
%	\hline
uniq&			unique					\\
uniqnes&		uniqueness				\\
%unit&			unitary					\\
%up-&			upper-					\\
\hline
%\end{tabularx}\formGap
%	{\tgbf\normalsize V}$\\$
%\begin{tabularx}{0.2\textwidth}{
%		| r |
%		| >{\raggedright\arraybackslash}X | }
%	\hline
val&			value					\\
\hline
%\end{tabularx}\formGap
%	{\tgbf\normalsize W}$\\$
%\begin{tabularx}{0.2\textwidth}{
%		| r |
%		| >{\raggedright\arraybackslash}X | }
%	\hline
-wd&			-ward					\\
-ws&			-wise					\\
wrto&			with respect to			\\
\hline
\end{tabularx}\formGap
%	{\tgbf\normalsize X}$\\$
%\begin{tabularx}{0.2\textwidth}{
%		| r |
%		| >{\raggedright\arraybackslash}X | }
%	\hline
%\end{tabularx}\formGap
%	{\tgbf\normalsize Y}$\\$
%\begin{tabularx}{0.2\textwidth}{
%		| r |
%		| >{\raggedright\arraybackslash}X | }
%	\hline
%\end{tabularx}\formGap
%	{\tgbf\normalsize Z}$\\$
%\begin{tabularx}{0.2\textwidth}{
%		| r |
%		| >{\raggedright\arraybackslash}X | }
%	\hline
%\end{tabularx}\formGap
$}$
\pagebreak

\begin{large}

\newcommand{\BmarkChapter}[2]{\bookmark[level=0,dest=Ch#1#2]{#1.#2}}
\newcommand{\BmarkChapterX}[2]{\bookmark[level=0,dest=Ch#1]{#2}}

\newcommand{\BmarkGeneral}[3]{\bookmark[level=1,dest=#1#2#3]{#3}}

\newcommand{\BM}[1]{\BmarkGeneral{\thisChapter}{\thisSection}{#1}}

\newcommandx{\BMXStart}[2][2=\thisChapter\thisSection]{\bookmark[level=1,dest=Ch#2]{#1}}
\newcommand{\BMX}[3][2]{\bookmark[level=#1,dest=\thisChapter\thisSection#2]{#3}}

\newcommand{\Anchor}[1]{\hypertarget{#1}{}}

\newcommandx{\BmarkI}[4][3=\thisChapter.\thisSection]{
	\def\thisChapter{#1}
	\def\thisSection{#2}
	\bookmark[level=0,dest=Ch#1#2]{#3}
	#4
}

\BmarkI{0}{B}{
    \BM{3}\BM{4}\BM{5}\BM{6}
}

\ChDecl{Ch0B}{0$\cdot$B}{\quad{\ANote $C,D$ are Dedekind cuts.\;\;Numbers used here are always rational.}}

\vspace{4pt}

%LOG: 尽管作者说这一节以后用不到,但这一节还是给我干出心理阴影了。我真是一点没学过实分析。

\BulletPointX Define $\tilde{q}=\Bra{a:a<q},$ and $-\tilde{q}=\widetilde{-q}=\Bra{a:a<-q}.$\TextB{}
Then $\tilde{0}=\Bra{a:a<0}=\Qbb\Backslash{\Qbs}\Rightarrow-\tilde{0}=\Bra{a:a<-b\leqslant0}=\tilde{0}.$\par\vspace{3pt}
\BulletPointX Define $-D=\Bra{a:a<-b,\,b\not\in D}=\Bra{{-a}:-a<-b\Longleftrightarrow a>b,\:b\not\in D}.$\TextB{}
$-\Par{{-D}}=-\Bra{a:a<-b,\,b\notin D}=\Bra{c:c<-a,\,a\geqslant -b,\,\forall b\notin D}=\Bra{c:c<b,\,\forall b\notin D}=D.$\TextB{}
{\FontSmall The last equa is becs (a) $d\notin D\Rightarrow\exists\,b\notin D,\,d\geqslant b,$ and (b) $d\in D\Rightarrow$ if $\exists\,b\notin D$ suth $d\geqslant b,$ then $b\in D,$ ctradic.}\vspace{-2pt}
\SepLine

%\Anchor{0BT1}\ProblemBX{\TipsN{1}}{
%	\TextA{Supp $D_1=\Bra{x\in\Qbb:x<a},\:D_2=\Bra{y\in\Qbb:y<b}.$}
%	\TextA{Prove $D_1+D_2=\Bra{q\in\Qbb:q<a+b}.$}
%}Becs $D_1+D_2=\Bra{x+y:x+y<x+b<a+b,\,x\in D_1,\,y\in D_2}.$\parSol{}
%Supp $x\in D_1.$ Let $\delta=\Par{a-x}\big/2\Rightarrow x<x'=x+\delta<a.$ Let $y=b-\delta\in D_2.$\parSol{}
%Then $x+b=x'+y<a+b.$ Let $\delta'=\Par{a-x'}\big/2\Rightarrow x+b<x''+y<a+b.$\PfEnd\vspace{3pt}\parSol{}
%\Or $D_1+D_2\subseteq\Bra{q\in\Qbb:q<a+b}.$ We show $\forall q\in RHS,\exists\,\Par{x,y}\in D_1\times D_2,\:x+y=q.$\parSol{}
%Let $\delta=\Par{a+b-q}\big/2.$ Let $x=a-\delta\in D_1,\,y=b-\delta\in D_2\Rightarrow x+y=a+b-\delta=q.$\PfEnd
%\SepLine

%\Anchor{0BT2}\ProblemBX{\TipsN{2}}{
%	\TextA{Supp $D$ is Ddkd cut. Prove $\forall b\in\Qbb\Backslash D,\:d<b,\forall d\in D.$}
%}Asum $\exists\,\Par{b,d}\in\Par{\Qbb\Backslash D}\times D,\:d\geqslant b.$ By def of Ddkd cut.\PfEnd
%\SepLine

\Anchor{0BT}\ProblemBX{\Tips}{
%	\PrePa\TextA{If $x\in\Qbb$ suth $\forall d\in D,\,d\leqslant x,$ then $x\in\Qbb\Backslash D.$\hfill\FontNorm\tgnr If not, by def, ctradic.}
%	\PrePb\TextA{If $\forall d\in D,\,\exists\,\varepsilon\in\Qbp,\,d+\varepsilon\in D.$\hfill\FontNorm\tgnr Supp $d<d'\in D\Rightarrow d'-d=\varepsilon\in\Qbp.$}
	\TextA{Prove $\forall\varepsilon>0,\,\exists\,b\not\in D$ \,suth\, $b-\varepsilon\in D.$}
%	\PrePb\TextA{Prove $\forall q\in\tilde{0}\cap\Qbp,\,\exists\,d$}
}Asum $\exists\,\varepsilon>0$ suth $\nexists\,b\not\in D,\,b-\varepsilon\in D\Longleftrightarrow\forall b\not\in D,\,b-\varepsilon\not\in D.$\parSol{}
Then $\Par{b-\varepsilon}-\dots-\varepsilon=b-n\cdot\varepsilon\not\in D$ for any $n\in\Nbp.$\parSol{}
Now $\forall d\in D,\,\exists\,n\in\Nbp$ suth $b-n\cdot\varepsilon<d\Rightarrow b-n\cdot\varepsilon\in D,$ ctradic.\PfEnd
%\ACoro Since $D$ does not contain the largest elem, $\forall\varepsilon\in\Qbp,\,\exists\,b\in\Qbb\Backslash D,\;b-\varepsilon<d$ for some $d\in D.$
\SepLine

\Anchor{0B1}\ProblemN{1}{
	\TextA{Prove {\tgnr\large(a)} $D+\tilde{0}=D,$ \;{\tgnr\large(b)} $-D$ is Dedekind cut, and $D+\Par{{-D}}=\tilde{0}.$}
}(a) $\forall d\in D,\,\exists\,\varepsilon>0,\:d+\varepsilon\in D\Rightarrow\Par{d+\varepsilon}+\Par{{-\varepsilon}}\in D+\tilde{0}.$\vspace{2pt}\parSol{}
(b) Asum $x\in -D$ is the largest elem of $-D\Rightarrow\exists\,b\not\in D,\:x<-b\Rightarrow 0<-b-x.$\parSol{\Hb}
Let $\delta=\Par{{-b-x}}\big/2\Rightarrow0<\delta<-b-x\Rightarrow x<x+\delta<-b.$\parSol{\Hb}
Thus by def, $x+\delta\in -D,$ ctradic the max of $x\in -D.$ \,Hence $-D$ is Ddkd cut.\vspace{3pt}\parSol{\Hb}
$D+\Par{{-D}}=\Bra{x+y:x+y<x-b,\:x\in D,\,b\not\in D}.$\parSol{\Hb}
Supp $a\in\tilde{0}\Rightarrow-a>0.$ By \TIPS, $\exists\,b\not\in D$ \,suth\, $b+a\in D.$\parSol{\Hb}
Note that $b<b-a\not\in D\Rightarrow-b>-b+a\in -D.$ \,Then $\Par{{-b+a}}+\Par{b+a}=2\:\!a<0.$\parSol{\Hb}
Thus $\forall a\in\tilde{0},\,\exists\,b\not\in D,\;d=b+\frac{\:1\:}{2}a\in D,\,c=-b+\frac{\:1\:}{2}a\in-D\Rightarrow c+d=a\in D+\Par{{-D}}.$\PfEnd
%\ACoro $-\Par{{-D}}+\Par{{-D}}=\tilde{0}\Rightarrow -\Par{{-D}}=D,$ by the uniqnes.
%$\Bra{a_1+a_2:a_1,a_2\in D}=D+D=\tilde{0}=\Bra{a:a<0}\Longrightarrow\forall a\in D,\,a+a<0\Longleftrightarrow a<0.$
\SepLine

\Anchor{0B3}\ProblemN{3}{
	\TextA{Show $C\subsetneq D\Longleftrightarrow D-C${\FontNorm ${}=\Bra{d-y:d\in D,\:y>x,\:x\not\in C}$\:\!} posi.\vspace{1pt}}
}(a) $C\subsetneq D\Rightarrow\exists\,x\in D\Backslash C\Rightarrow\exists\,y\in D,\,y>x\Rightarrow\exists\,d\in D,\,d>y\Longleftrightarrow 0<d-y\in D-C.$\vspace{2pt}\parSol{}
(b) $0\in D-C\Rightarrow\exists\,y>x\notin C,\;y\in D\Rightarrow\forall c\in C,\,c<x<y\in D\Rightarrow C\subseteq D.$ \,又 $D-C\neq\tilde{0}.$\PfEnd
\SepLine

%\Anchor{0B'1}\ProblemB{
%	\TextB{Prove $D=-D\Longleftrightarrow D=\tilde{0}.$}
%}$\Bra{a_1+a_2:a_1,a_2\in D}=D+D=\tilde{0}=\Bra{a:a<0}\Rightarrow\forall a\in D,\,a+a<0\Longleftrightarrow a<0.$\parSol{}
%\SepLine

\Anchor{0B5}\ProblemN{5}{
	\TextA{Prove {\tgnr\large(a)} $D$ posi $\Rightarrow -D$ not posi, \;{\tgnr\large(b)} non0 $-D$ not posi $\Rightarrow D$ posi.}
}(a) $0\notin\Bra{a:a<-b,\,b\notin D}\Longleftrightarrow\nexists\,b\notin D,\,0<-b\Longleftrightarrow\forall b\notin D,\,b\geqslant0\Longleftarrow 0\in D.$\vspace{2pt}\parSol{}
(b) Becs $\tilde{0}$ is the largest non posi cuts. Thus $-D\neq\tilde{0}\Rightarrow-D\subsetneq\tilde{0}\Rightarrow\tilde{0}-\Par{{-D}}=D$ posi.\parSol{\Hb}
\Or $\exists\,a<0,\,a\notin -D=\Bra{a:-a>b,\,b\notin D}\Longleftrightarrow\nexists\,b\notin D,\,-a>b\Longleftrightarrow\forall b\notin D,\,0<-a\leqslant b.$\PfEnd
\SepLine

\BulletPointX Define $D^+=\Bra{d\in D:d>0}=D\cap\Qbp.$ \,Then $D^+\neq\emptySet\Longleftrightarrow\Qbb\Backslash\Qbp\subsetneq D\Longleftrightarrow 0\in D\Longleftrightarrow D$ posi.\vspace{2pt}\TextB{}
Define $D^-=\Bra{r\not\in D:r\leqslant 0}=\Par{\Qbb\Backslash D}\cap\Par{\Qbb\Backslash\Qbp}=\Qbb\Backslash\Par{D\cup\Qbp}.$\TextB{}
(a) $D^-=\zeroSubs\Longleftarrow D=\tilde{0}.$ \;Convly, $\Bra{r\notin D:r\leqslant0}=\zeroSubs\Longrightarrow\Qbb\Backslash D=\Qbs.$\TextB{}
(b) $D^-=\emptySet\Longleftrightarrow D\cup\Qbp=\Qbb\Longleftrightarrow\Qbb\Backslash\Qbp\subseteq D\Longleftrightarrow 0\in D\Longleftrightarrow D$ posi. \hfill\ACoro $D$ not posi $\Longleftrightarrow 0\in D^-.$\TextB{}
(c) $\Par{D^-}{^-}=\Bra{r\in D:r\leqslant0}=\Qbb\Backslash D^+.$ \hfill\ACoro $D$ not posi $\Longleftrightarrow\Par{D^-}{^-}=D.$
\SepLine

\BulletPointX $\Par{{-D}}{^+}=\Par{{-D}}\cap\Qbp=\Bra{a:0<a<-b,\:b\not\in D\Longleftrightarrow b\in D^-\nonzero}.$\TextB{}
$\Par{{-D}}{^-}=\Par{\Qbb\Backslash{-D}}\cap\Par{\Qbb\Backslash\Qbp}=\Bra{a:0\geqslant a\geqslant-b,\:\forall b\not\in D}.$
%\vspace{3pt}\TextB{}
%Define $\tilde{1}=\Bra{a:a<1}\Rightarrow\tilde{1}{^+}=\Bra{a:0<a<1},\:\tilde{1}{^-}=\Bra{a:a\leqslant 0}=\tilde{0}\cup\zeroSubs.$\TextB{}
%Then $-\tilde{1}=\Bra{a:-a>1\Longleftrightarrow a<-1}$ not posi.
\SepLine

\BulletPointX For $C,D$ posi, \,define $CD=\Bra{a:a\leqslant cd,\,c\in C^+,\,d\in D^+}=\Bra{cd:c\in C^+,d\in D^+}\cup\Par{\Qbb\Backslash\Qbp}.$\vspace{2pt}\TextB{}
$\Bra{cd:c\in C^+,d\in D^+}=CD\cap\Qbp=\Par{CD}{^+}.$ \;Note that `$a\leqslant cd$' here is equiv to `$a<cd$'.%\vspace{4pt}\par
\SepLine

\BulletPointX For $-C,-D$ posi, \,define $CD=\Par{{-C}}\Par{{-D}}=\Bra{cd:c\in\Par{{-C}}{^+},d\in\Par{{-D}}{^+}}\cup\Par{\Qbb\Backslash\Qbp}.$\TextB{}
$CD=\Bra{0<cd<\Par{{-r}}\Par{{-s}}:r\in C^-\nonzero,\,s\in D^-\nonzero}\cup\Par{\Qbb\Backslash\Qbp}=\Bra{a:a<rs,\,r\in C^-,\,s\in D^-}.$\vspace{2pt}\TextB{}
If $C,-C$ not posi $\Rightarrow C=\tilde{0},$ then with the asum $\tilde{0}D=\tilde{0},$ \,it still holds. \,Simlr for $D,-D$ not posi.
%\BulletPointX {\normalsize The intuitive key point is that the prod of cuts is the cut with the endpoint being the prod of endpoints of cuts.}
\SepLine

\BulletPointX For $D$ posi, \,define $D^{-1}=\Bra{a:a<1\big/b,\;b\notin D}\Rightarrow DD^{-1}=\Bra{a:a\leqslant d\big/b<1,\;b\notin D,\,d\in D^+}=\tilde{1}.$\TextB{}
The last equa holds becs $\forall a\in\tilde{1}\cap\Qbp,\:\exists\,d\in D^+$ suth $b=d\big/a\notin D\Rightarrow d\big/b=a\in DD^{-1}.$\vspace{3pt}\par
\BulletPointX For non0 $D$ not posi, \,define $D^{-1}=\Bra{a:a<1\big/b,\;b\in D^-}\Rightarrow\Par{D^{-1}}{^-}=\Bra{s:0\geqslant s\geqslant 1\big/b,\,\forall b\in D^-}.$\TextB{}
Then $DD^{-1}=\Bra{a:a<rs,\;r\in D^-,\,s\in\Par{D^{-1}}{^-}}=\Bra{a:a<rs,\:\exists\,r\in D^-,\,0\leqslant rs\leqslant r\big/b,\,\forall b\in D^-}.$\TextB{}
Asum $\exists\,a\in\tilde{1}\cap\Qbp,\,\forall r\in D^-,\,s\in\Par{D^{-1}}{^-},\;a\geqslant rs\Longrightarrow a\geqslant r\big/b\Longleftrightarrow r\big/a\leqslant b,\,\forall b\in D^-.$ \;Let $r=0.$
\SepLine

\BulletPointX For $C$ not posi and $D$ posi, we expect that $CD$ not posi. Consider $C$ and $-D$ both not posi.\TextB{}
$CD=-C\Par{{-D}}=-\Bra{a:a<rt,\:r\in C^-,\,t\in\Par{{-D}}{^-}}=\Bra{{-a}:a>b,\,b\geqslant rt,\:\forall r\in C^-,\,t\in\Par{{-D}}{^-}}$\TextB{}
\Blind{$CD$}${}=\Bra{{-a}:a>rt,\:\forall r\in C^-,\,\forall t\text{\;suth\;}0\geqslant t\geqslant -s,\,\forall s\notin D}$\TextB{}
\Blind{$CD$}${}=\Bra{a:a<ru,\:\forall r\in C^-,\,\forall u\text{\;suth\;}0\leqslant u\leqslant s,\,\forall s\notin D}.$\hfill{\FontSmall \BigPar{$r\leqslant 0<s,\;rs\leqslant ru=-rt\leqslant0\leqslant rt\leqslant-rs.$}}\vspace{3pt}\par
\BulletPointX Note the `$0\leqslant u$'. Becs $C^-\neq\emptySet\Rightarrow 0\in C^-.$ If it is to be exactly $CD=\Bra{a:a<0}$, \,then $C^-=\zeroSubs,$\TextB{}
for if not, $\exists\,u>0,$ and $\exists\,r\in C^-\nonzero,$ suth $\exists\,a<ru<0.$ Hence `$0\leqslant u$' is actually `$0<u$'.\vspace{3pt}\par
\BulletPointX `$u\leqslant s$' cannot be abbreviated as in $\Bra{{-a}:a>-rs,\,\forall s\not\in D,\,r\in C^-}=\Bra{{a}:a<rs,\,\forall s\not\in D,\,r\in C^-}.$\TextB{}
`$u\leqslant s$' cannot be `$u<s$', becs here $rs<ru\Rightarrow\exists\,a=rs.$ \,Simlr for `$a<ru$' to be `$a\leqslant ru$' with `$u<s$'.\vspace{4pt}\par
\BulletPointX Note that \,$\Bra{u:0<u\leqslant s,\,\forall s\notin D}=\MathLeftBrace{l}{\!D^+\cup\Bra{{\min\Qbb\Backslash D}},\,\,\text{if it exis,}\\\!D^+,\hfill\text{othws.}}$ \quad Denote it by $D^\oplus=D^\otimes\nonzero.$\vspace{8pt}
\SepLine

\BulletPointX For $C$ not posi and $D$ posi. If $C=\tilde{0},$  then $CD=-C\Par{{-D}}=-\tilde{0}.$ \,Now consider $-C$ and $D$ both posi.\vspace{1pt}\TextB{}
But\, $CD=-\Par{{-C}}D=-\Bra{a:a\leqslant cd,\,\,c\in\Par{{-C}}{^+},\,d\in D^+}\neq\Bra{a:a<-cd,\;\forall c\in\Par{{-C}}{^+},\,d\in D^+}.$\TextB{}
Altho ``$a\leqslant cd$'' is equiv to ``$a<cd$'' so that \,$b\notin\Par{{-C}}D\Rightarrow b\geqslant cd,$ which is actually $b>cd,\,\forall c,d.$\TextB{}
And\, $a<-b<-cd,\,\forall c,d\Rightarrow\forall a,\,\exists\,x$ suth $a<x<-cd,\,\forall c,d.$ \,While {\;\!$a$\:\!} can be the `boundary' in $RHS$.
\SepLine

%\BulletPointX For $C$ posi and $D$ not posi. If $D=\tilde{0},$ then $CD=-\Par{{-C}}\,\tilde{0}=-\tilde{0}.$ \,Now consider $C$ and $-D$ both posi.\vspace{1pt}\TextB{}
%$CD=-\Sbra{C\Par{{-D}}}=-\Bra{a:a<cd,\,c\in C^+,\,d\in\Par{{-D}}{^+}}=\Bra{{-a}:a>b,\,b\geqslant cd,\,\forall c\in C^+,\,d\in\Par{{-D}}{^+}}$\TextB{}
%\Blind{$CD$}${}=\Bra{{a}:a<-cd,\,\forall c\in C^+,\,\forall d\text{\;suth\;}$\uline{$0<d<-s,\,\exists\,s$}${}\in D^-}$\hfill{\FontSmall\BigPar{$sd<{}$\uline{$-cd<0$}}}\TextB{}
%$\Blind{CD}=\Bra{a:a<ct,\,\forall c\in C^+,\,\forall t\text{\;suth\;}s<t<0,\,\exists\,s\in D^-}$\vspace{4pt}\par
%\BulletPointX Note that \,$\Bra{t:s<t<0,\,s\in D^-}=\MathLeftBrace{l}{\!D^-\Backslash\Bra{0,\min\Qbb\Backslash D},\,\,\text{if it exis,}\\\!D^-\nonzero,\hfill\text{othws.}}$\vspace{8pt}
%\SepLine

\BulletPointX $LHS=\Bra{a:a<ru,\;\forall r\in C^-\!,\;\forall u\in D^\oplus},\;\Bra{cs:c\in C,\,s\notin D}=RHS.$\TextB{}
Becs \,$cs\leqslant cu<ru.$ \,We show $LHS\subseteq RHS.$ \,Let $c_1<\cdots<c_n<\cdots\in C$, and $s_1>\cdots>s_m>\cdots\notin D.$\TextB{}
Then \uline{$c_1s_1<\cdots<c_n\:\!s_m<\cdots<ru,\,\forall r,u$ as in $LHS.$ \;Thus $\,a\in LHS\Rightarrow\exists\,a<c_j\:\!s_k.$}\PfEnd
\SepLine

\BulletPointX For $C$ posi and $D$ not posi. If $D=\tilde{0},$ then $CD=-\Par{{-C}}D=-\tilde{0}.$\TextB{}
Now consider $-C$ not posi and $-D$ posi. $\Qbb\Backslash{{-D}}=\Bra{s:s\geqslant -y,\;\forall y\notin D}.$\TextB{}
$CD=\Par{{-C}}\Par{{-D}}=\Bra{a:a<ru,\;\forall r\in\Par{{-C}}{^-},\;\forall u\in\Par{{-D}}{^\oplus}}$\TextB{}
$\Blind{CD}=\Bra{a:a<ru,\;\forall r\text{\;suth\;}\forall x\notin C,\,0\geqslant r\geqslant -x,\;\forall u\text{\;suth\;}0<u\leqslant s,\,\forall s\geqslant -y,\,\forall y\notin D}$\TextB{}
$\Blind{CD}=\Bra{a:a<\Par{{-r}}\Par{{-u}},\;\forall r\text{\;suth\;}\forall x\notin C,\,0\leqslant -r\leqslant x,\;\forall u\text{\;suth\;}y\leqslant -u<0,\,\forall y\notin D}$\TextB{}
$\Blind{CD}=\Bra{a:a<ru,\;\forall r\text{\;suth\;}\forall x\notin C,\,0\leqslant r\leqslant x,\;\forall u\in D^-}.$\vspace{3pt}\par
\BulletPointX Note the `$0\leqslant r$'. Becs $D^-\neq\emptySet\Rightarrow 0\in D^-.$ If it is to be exactly $CD=\Bra{a:a<0},$ then $D^-=\Bra{0},$\TextB{}
for if not, $\exists\,r>0,$ and $\exists\,u\in D^-\nonzero,$ suth $\exists\,a<ru<0.$ Hence `$0\leqslant r$' is actually `$0<r$'.
\SepLine

\BulletPointX We show $-D=\Bra{a:a<-b,\,b\notin D}=\Par{{-\tilde{1}}}D.$\TextB{}
For $D$ posi, $RHS=\Bra{a:a<ru,\,\forall r\text{\;suth\;}{-1}\leqslant r\leqslant0,\,\forall u\in D^\oplus}=\Bra{a:a<-u,\,\forall u\in D^\oplus}\supseteq -D.$\TextB{}
Supp $x$ suth $-b\leqslant x<-u,\,\forall b\notin D,\,\forall u\text{\;suth\;}{-s}\leqslant-u<0,\:\!\forall s\notin D\Longrightarrow -b\leqslant x<-s,\,\forall b,s\notin D.$\TextB{}
For $D$ not posi, $RHS=\Bra{a:a<rb,\,\exists\,r\text{\;suth\;}{-1}\leqslant r\leqslant0,\,b\in D^-}=\Bra{a:a<-b,\;0\geqslant b\notin D}=-D.$
\SepLine\pagebreak

\BulletPointX We show $\tilde{1}\:\!D=D.$ \;For $D$ not posi, immed. \,Othws, \;$\tilde{1}\:\!D=\Bra{a:a\leqslant ij<j,\;0<i<1,\;j\in D^+}\subseteq D.$\vspace{3pt}\TextB{}
Now $\BigPar{\tilde{1}\:\!D}{^+}\subseteq D^+.$ \;又 $\forall d\in D^+,\,\exists\,\varepsilon>0,\:d+\varepsilon\in D^+\Rightarrow d=\Par{d+\varepsilon}\:\!${\Large$\frac{d}{\:d\,+\,\varepsilon\:}$}${}\in\tilde{1}\:\!D.$
\SepLine

\ProblemN{4}{
	\TextA{Supp $B,C,D$ non0 Dedekind cuts. Show $\Par{BC}D=B\Par{CD},$ $B\Par{C+D}=BC+BD.$}
}We discuss in cases.\vspace{8pt}\par\quad
\begin{tabularx}{0.3\textwidth} {
		| >{\centering\arraybackslash}X 
		| >{\centering\arraybackslash}X 
		| >{\centering\arraybackslash}X 
		| >{\centering\arraybackslash}X 
		| >{\centering\arraybackslash}X 
		| >{\centering\arraybackslash}X 
		| >{\centering\arraybackslash}X 
		| >{\centering\arraybackslash}X 
		| >{\centering\arraybackslash}X | }
	\hline
	\Backslash & 1 & 2 & 3 & 4 & 5 & 6 & 7 & 8 \\
	\hline
	B  & + & + & + & + & - & - & - & -\\
	\hline
	C  & + & + & - & - & - & + & - & +\\
	\hline
	D  & + & - & + & - & - & - & + & +\\
	\hline
\end{tabularx}\vspace{8pt}
\par\quad
%(1) $\BigPar{\Par{BC}D}{^+}=\Bra{\Par{bc}d:bc\in\Par{BC}{^+},\,d\in D^+}=\Bra{b\Par{cd}:b\in B^+,\,cd\in\Par{CD}{^+}}=\BigPar{B\Par{CD}}{^+}.$\vspace{2pt}\par\quad
%\Blind{(1)} $B\Par{C+D}=\Bra{a:a\leqslant bx,\;b\in B^+,\,0<x=c+d\in C+D}.$\par\quad
%\Blind{(1)} $BC+BD=\Bra{x:x\leqslant uc,\;u\in B^+,\,c\in C^+}+\Bra{y:y\leqslant vd,\;v\in B^+,\,d\in D^+}$\par\quad
%\Blind{(1)} $\Blind{BC+BD}=\Bra{a:a\leqslant uc+vd,\;u,v\in B^+,\,c\in C^+,\,d\in D^+}.$\hfill Think intuitively. \;Done.\vspace{6pt}\par\quad
(4) $\Par{BC}{^-}=\Bra{r:0\geqslant r\geqslant bc,\,\exists\,c\in C^-,\,\exists\,b\in B^\oplus}.$\par\quad
\Blind{(4)} $\Par{BC}D=\Bra{a:a<rs,\:r\in\Par{BC}{^-},\,s\in D^-}$\par\quad
\Blind{(4)} $\Blind{\Par{BC}D}=\Bra{a:a<bcs,\,\exists\,s\in D^-,\,c\in C^-,\,\exists\,b\in B^\oplus}.$\par\quad
\Blind{(4)} $B\Par{CD}=\Bra{a:a\leqslant bx<bcs,\,\exists\,s\in D^-,\,c\in C^-,\,\exists\,b\in B^+,\text{ and }cs>x\in\Par{CD}{^+}}$\par\quad
\Blind{(4)} $\Blind{B\Par{CD}}=\Bra{a:a<bcs,\,\exists\,s\in D^-,\,c\in C^-,\,\exists\,b\in B^+}.$\vspace{2pt}\par\quad
\Blind{(4)} Note that \,$\Bra{q:q<b,\,b\in B^+}=\Bra{q:q<b,\,\exists\,b\in B^\oplus}.$ \;Done.\vspace{6pt}\par\quad
\Blind{(4)} $B\Par{C+D}=\Bra{a:a<ru,\,\forall r\text{\;suth\;}\forall x\notin B,\,0<r\leqslant x,\;\forall u\text{\;suth\;}0\geqslant u>c+d,\,\forall c\in C,\,d\in D}$\par\quad
\Blind{(4)} $\Blind{B\Par{C+D}}=\Bra{a:a<ru,\,\forall r\in B^\oplus,\;\forall u\text{\;suth\;}0\geqslant u\geqslant c+d,\,\exists\,c\in C^-,\,d\in D^-}$\par\quad
\Blind{(4)} $\Blind{B\Par{C+D}}=\Bra{a:a<r\Par{c+d},\,\forall r\in B^\oplus,\;\forall c\in C^-,\,d\in D^-}.$\par\quad
\Blind{(4)} $BC+BD=\Bra{x:x<pc,\,\forall p\in B^\oplus,\;\forall c\in C^-}+\Bra{y:y<qd,\,\forall q\in B^\oplus,\;\forall d\in D^-}$\par\quad
\Blind{(4)} $\Blind{BC+BD}=\Bra{a:a<pc+qd,\,\forall p,q\in B^\oplus,\;\forall c\in C^-,\,d\in D^-}.$\hfill Done immed.\vspace{6pt}\par\quad
(6) $\Par{BC}{^-}=\Bra{r:0\geqslant r\geqslant bc,\,\exists\,b\in B^-,\,\exists\,c\in C^\oplus}.$ \,$\Par{CD}{^-}=\Bra{r:0\geqslant r\geqslant cd,\,\exists\,d\in D^-,\,\exists\,c\in C^\oplus}.$\par\quad
\Blind{(6)} $\Par{BC}D=\Bra{a:a<rd\leqslant bcd,\,\exists\,d\in D^-,\,b\in B^-,\,\exists\,c\in C^\oplus}.$\par\quad
%\Blind{(6)}\par\quad
\Blind{(6)} $B\Par{CD}=\Bra{a:a<br\leqslant bcd,\,\exists\,b\in B^-,\,d\in D^-,\,\exists\,c\in C^\oplus}.$ \;Done.\vspace{2pt}\par\quad
\Blind{(6)} \Or $\Par{BC}D=\Par{CB}D=C\Par{BD}=C\Par{DB}=\Par{CD}B=B\Par{CD}.$\vspace{6pt}\par\quad
$BC+BD=\Bra{x:x<bc,\,\forall b\in B^-,\,\forall c\in C^\oplus}+\Bra{y:y<bd,\,\exists\,b\in B^-,\,d\in D^-}$\par\quad
$\Blind{BC+BD}=\Bra{a:a<pc+qd,\,\forall p\in B^-,\,\forall c\in C^\oplus,\text{\;and\;}\exists\,q\in B^-,\,d\in D^-\Rightarrow q\geqslant p}.$\vspace{2pt}\par\quad
$\Blind{BC+BD}=\Bra{bc:b\in B,\,c\notin C}+\Bra{a:a<bd,\,\exists\,b\in B^-,\,d\in D^-}$\vspace{-18pt}\par\quad
\qquad\Or\par\quad
$\Blind{BC+BD}=\Bra{pc+a:a<qd,\;\exists\,p\in B,\,q\in B^-,\,c\notin C,\,d\in D^-}$\par\quad
$\Blind{BC+BD}=\Bra{x=pc+\Par{x-pc}\,:\,x-pc<qd,\;\exists\,p\in B,\,q\in B^-,\,c\notin C,\,d\in D^-}$\par\quad
$\Blind{BC+BD}=\Bra{a:a<pc+qd,\;\exists\,c\notin C,\,d\in D^-,\,p\in B,\,q\in B^-\Rightarrow p<q\leqslant0,\text{\,and\;}c\neq0}.$\vspace{4pt}\par\quad
(I) If $C=-D.$ \,Then $B\Par{C+D}=\tilde{0}.$ \,Note that \,$c\in\Qbb\Backslash C=\Qbb\Backslash{{-D}}=\Bra{c:c\geqslant -d,\:\forall d\in D^-}.$\par\quad\HI
Thus \,$pc+qd=pc+\Par{qd-pd}+pd\leqslant0,$ \,becs\, $qd-pd\leqslant0$ \,and\: $pc+pd\leqslant0.$ \,Now immed done.\vspace{3pt}\par\quad
(II) If non0 $C+D$ not posi. Then $B\Par{C+D}=\Bra{a:a<bx,\,\exists\,b\in B^-,\,x>c+d,\,\forall\Par{c,d}\in C\times D}.$\par\quad\HII
\par\quad\HII
\par\quad\HII
\par\quad
(III) If $C+D$ posi. Then $B\Par{C+D}=\Bra{a:a<bx,\,\forall b\in B^-,\,x\in\Par{C+D}{^\oplus}}.$\par\quad\HIII
If $\Par{C+D}{^\oplus}=\Par{C+D}{^+}.$ \,Then $B\Par{C+D}=\Bra{a:a<bc+bd,\,\forall b\in B^-,\,\forall\Par{c,d}\in C^+\times D^+}.$\par\quad\HIII
Othws, $\exists\,s=\min\Qbb\Backslash\Par{C+D}\Rightarrow B\Par{C+D}=\Bra{a:a<bs,\,\forall b\in B^-}.$\par\quad\HIII
\SepLine

\ChEnd
\pagebreak

\ChDecl{Ch0C}{0$\cdot$C}{}

\vspace{2pt}

\Anchor{0C5}\ProblemN{5}{
	\TextA{Supp $a_1,a_2,\dots$ is a seq in $\Qbb,$ and $\Sup\Bra{a_1,a_2,\dots}=\sqrt{2}.$}
	\TextA{Prove $\Sup\Bra{a_{n},a_{n+1},\dots}=\sqrt{2}$ \;for all $n\in\Nbp.$}
}Becs the sup not in seq $\Rightarrow$ infily many disti elem.\parSol{}
$\forall a_i,\,\exists\,a_j,\:a_i<a_j<\sqrt{2}.$ \;For $a_{n+k},$ choose $a_i>a_{n+k}.$ If $i\in\;\!\!\Bra{1,\dots,n},$ then choose $a_j>a_i.$\parSol{}
After at most $\Par{n+1}$ steps, we have $a_{m}$ with $m>n.$ Thus $\forall a_{n+i},\,\exists\,a_{n+j},\:a_{n+i}<a_{n+j}<\sqrt{2}.$\PfEnd
\SepLine

\ProblemB[]{
	\TextB{Supp nonempty $A\subseteq\Rbb.$}
}
\Anchor{0CT1}\ProblemBX{\TipsN{1}}{
	\TextA{Define $-A=\Bra{-a:a\in A}\Rightarrow{-\Par{{-A}}}=A.$ \,Prove $\Sup\Par{{-A}}=-\inf A.$}
}$-b$ is an upper bound of $-A\Longleftrightarrow\forall a\in A,\,-a\leqslant-b\Longleftrightarrow a\geqslant b\Longleftrightarrow b$ is a lower bound of $A.$\parSol{}
Thus \,$-b_M=\Sup\Par{{-A}}\Longleftrightarrow-b_M\leqslant -b\Longleftrightarrow b_M\geqslant b\Longleftrightarrow b_M=\inf A.$\PfEnd
\SepLine

\Anchor{0CT2}\ProblemBX{\TipsN{2}}{
	\TextA{Show if $x\in\Rbb,$ {\tgnr\large(a)} $\sup A>x\Rightarrow\exists\,a\in A,\:a>x,$ \;{\tgnr\large(b)} $\inf A<x\Rightarrow\exists\,a\in A,\:a<x.$}
}(a) $\nexists\,a>x\Longleftrightarrow\forall a\in A,\,a\leqslant x.$ Then by def of sup.\parSol{\Ha}
\Or By (b), $\inf\Par{{-A}}=-\sup A<-x\Rightarrow\exists\,{-a}\in A,\:-a<-x.$ \hfill Simlr for (b).\Blind{\qquad}\PfEnd
%\vspace{4pt}\parSol{}
%(b) $\nexists\,a<x\Longleftrightarrow\forall a\in A,\,a\geqslant x.$ Then by def of inf.\parSol{\Hb}
%\Or By (a), $\sup\Par{{-A}}=-\inf A>-x\Rightarrow\exists\,{-a}\in A,\:-a>-x.$\PfEnd
\SepLine

\Anchor{0C6}\ProblemN{6}{
	\TextA{Supp $A,B\subseteq\Rbb$ has infily many disti elem, so has $A+B=\Bra{a+b:a\in A,b\in B}.$}
	\TextA{Prove $\Sup\Par{A+B}=\sup A+\sup B,\,$ and \,$\Inf\Par{A+B}=\inf A+\inf B.$}
}$\inf A+\inf B\leqslant a+b\leqslant\sup A+\sup B\Rightarrow\Sup\Par{A+B}\leqslant\sup A+\sup B,\,\inf A+\inf B\leqslant\Inf\Par{A+B}.$\vspace{2pt}\parSol{}
$\sup A+\sup B>\Sup\Par{A+B}\Longleftrightarrow\sup A>\Sup\Par{A+B}-\sup B$\parSol{}
$\Longleftrightarrow\exists\,a+\sup B>\Sup\Par{A+B}\Longleftrightarrow\sup B>\Sup\Par{A+B}-a\Longleftrightarrow\exists\,a+b>\Sup\Par{A+B}.$ Ctradic.\vspace{2pt}\parSol{}
%$\inf A<\Inf\Par{A+B}-\inf B\Longleftrightarrow\inf A+\inf B<\Inf\Par{A+B}$\parSol{}
%$\Longleftrightarrow\exists\,a+\inf B<\Inf\Par{A+B}\Longleftrightarrow\inf B<\Inf\Par{A+B}-a\Longleftrightarrow\exists\,b+a<\Inf\Par{A+B}.$ Ctradic.
Simlr for $\Inf\Par{A+B}\in A+B.$ \,\Or Apply to $-A-B,$ becs $\Sup\Par{{-A}}=-\inf A.$\PfEnd
%Then $\forall a\in A,b\in B,\,a+b\in A+B\Rightarrow\exists\,c+d\in A+B,\,c+d>a+b.$\parSol{}
%If $\sup A\in A.$ Then $\sup A+d>a+b.$
%Supp $A,B$ both have infily many elem. Supp sups and infs are not in there sets.\parSol{}
%$\forall a\in A,b\in B,\,$\parSol{}
\SepLine

%\Anchor{0C7}\ProblemN{7}{
%	\TextA{Supp $X\neq\emptySet,\,f,g:X\rightarrow\Rbb$ are functions. \,Prove $\sup_X\!\Par{\,f+g}\leqslant\sup_X f+\sup_X g.$}
%}$ $
%\SepLine

%\Anchor{0C9}\ProblemN{9}{
%	\TextA{Prove the ordered field of rational functions with coeffs in $\Rbb$ does not satisfy Archimedean Property.}
%}
%\SepLine

%\Anchor{0C14}\ProblemN{14}{
%	\TextA{Supp $A,B\subseteq\Interval{[}{]}{{-\infty,\infty}}$ are intervals, and $A\cap B\neq\emptySet.$ \,Prove $A\cup B$ are intervals.}
%}
%\SepLine

%\Anchor{0C15}\ProblemN{15}{
%	\TextA{Supp $I\subseteq\Rbb$ is an interval containing more than one number. Prove $\Inf\Par{I\cap\Qbb}=\inf I.$}
%}$\inf I<\Inf\Par{I\cap\Qbb}\Rightarrow\exists\,x\in I,\,x<\Inf\Par{I\cap\Qbb}.$ While $\exists\,n\in\Nbp,\,1\big/n<x.$\PfEnd
%\SepLine

\Anchor{0C16}\ProblemN{16}{
	\TextA{Supp $\Rbb{_1},\Rbb{_2}$ are complete ordered fields. Define $\varphi_1,\varphi_2$ as in [0.11].\vspace{2pt}}
	\TextA{Define $\mR_1\Par{a}=\Bra{q\in\Qbb:\varphi_1\Par{q}\leqslant a}\Rightarrow\sup_{\mR_1\SmallPar[0.5pt]{a}}\varphi_1=a.$ \;Define $\psi_1\Par{a}=\sup_{\mR_1\SmallPar[0.5pt]{a}}\varphi_2.$\vspace{4pt}}
	\PrePa\TextA{Show $\psi=\psi_1:\Rbb{_1}\rightarrow\Rbb{_2}$ is one-to-one. \;{\tgnr\large(b)} Show $\psi\Par{0}=0,\,\psi\Par{1}=1.$\vspace{2pt}}
	\PrePc\TextA{Show $\psi\Par{a\pm b}=\psi\Par{a}\pm\Par{b},$ and $\psi\Par{ab^{-1}}=\psi\Par{a}\psi\Par{b}{^{-1}}.$\vspace{2pt}}
	\PrePd\TextA{Supp $a\in\Rbb{_1}.$ Show $a>0\Longleftrightarrow\psi\Par{a}>0.$\vspace{2pt}}
}(a) Define $\mR_2\Par{c}=\Bra{q\in\Qbb:\varphi_2\Par{q}\leqslant c}\Rightarrow\sup_{\mR_2\SmallPar[0.5pt]{c}}\varphi_2=c.$\vspace{1pt}\parSol{\Ha}
Define $\psi_2\Par{c}=\sup_{\mR_2\SmallPar[0.5pt]{c}}\varphi_1.$ Then $\psi_2:\Rbb{_2}\rightarrow\Rbb{_1}$ well-defined.\vspace{1pt}\parSol{\Ha}
Note that $\mR_2\BigPar{\psi_1\Par{a}}=\Bra{q\in\Qbb:\varphi_2\Par{q}\leqslant\sup_{\mR_1\SmallPar[0.5pt]{a}}\varphi_2}=\mR_1\Par{a}.$\vspace{1pt}\parSol{\Ha}
Now $\psi_2\BigPar{\psi_1\Par{a}}=\sup_{\mR_1\SmallPar[0.5pt]{a}}\varphi_1=a.$ \;Rev the roles of $\Rbb_1,\Rbb_2.$\vspace{4pt}\parSol{}
(b) Note that $\varphi\Par{q}<0\Longleftrightarrow q<0,$ and $\varphi\Par{0}=0.$\parSol{\Hb}
$\forall q\in\mR_1\Par{1}=\Bra{1\big/m\in\Qbb:m\in\Nbp,\varphi_1\Par{1\big/m}=\Par{1+\dots+1}{^{-1}}\leqslant 1}\cup\Par{\Qbb\Backslash\Qbp},$ \,$\varphi_2\Par{q}\leqslant 1.$\vspace{2pt}\parSol{}
(c) $\mR_1\Par{a\pm b}=\Bra{p\pm q\in\Qbb:\varphi_1\Par{p}\pm\varphi_1\Par{q}\leqslant a\pm b}.$\parSol{\Hc}
$\mR_1\Par{ab^{-1}}=\Bra{pq^{-1}\in\Qbb:\varphi_1\Par{q}\cdot\varphi_1\Par{q}{^{-1}}\leqslant ab^{-1}}.$\vspace{3pt}\parSol{}
(d) $a>0\Longleftrightarrow\exists\,n\in\Nbp,\:1\big/n<a\Longleftrightarrow\psi_1\Par{a}>0.$\PfEnd
\SepLine

\ChEnd
\include{Ch6789-10-11-12}

\end{large}
\end{document}
